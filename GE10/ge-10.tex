\documentclass[12pt]{article}
\usepackage{anysize}
\usepackage[ngerman,english]{babel}
\marginsize{3.5cm}{2.5cm}{1cm}{2cm}

\usepackage[utf8]{inputenc}
\usepackage[english]{babel}
\usepackage{amsmath}
\usepackage{amsthm}
\usepackage{amsfonts}
\usepackage{scalerel,amssymb}  % for mathbb
%---------------------------------------------------------
\usepackage{authblk}
\usepackage{blindtext}
\usepackage{graphicx}
\usepackage[caption=true]{subfig}
\usepackage[
  colorlinks = true,
  citecolor = red,
  % linkcolor = darkblue, % internal references
  % urlcolor = darkblue,
]{hyperref}
\usepackage{cleveref}
\crefname{figure}{Figure}{Figures}
\crefname{equation}{Eq.}{Eqs.}
%---------------------------------------------------------
% \newcommand{\begin{equation}}{\eqbeg} 
% \newcommand{\end{equation}}{\eqend}
\usepackage{bm}
\usepackage{dfcmd}
% \newcommand{\Bvarepsilon}{\bm\varepsilon} 
% \newcommand{\Bx}{\bm x} 
% \newcommand{\Bu}{\bm u} 
% \newcommand{\Bv}{\bm v} 
% \newcommand{\Bg}{\bm g} 
% \newcommand{\Bb}{\bm b} 
\newcommand{\divv}{\text{divv}} 
%---------------------------------------------------------
\usepackage{lineno}
% \linenumbers     

\usepackage{float}
\usepackage{psfrag}

\usepackage{fontawesome}
\usepackage{relsize}

%---------------------------------------------------------
%For \toprule \midrule \bottomrule in table environment
\usepackage{booktabs}


%Page numbering in style 1/3...
\usepackage{lastpage}  
\usepackage{hyperref}
\makeatletter
\renewcommand{\@oddfoot}{\hfil 
% Aachen, November $04^{th}$, 2021 \hspace{300pt} 
NumPDE $\cdot$ GE10 $\cdot$ WS21/22 
\hspace{280pt} 
\thepage/\pageref{LastPage}\hfil}
\makeatother
%------------------------------------------------------------------------------

\begin{document}
\begin{center}
	\section*{Global Exercise - 10}
\end{center}
\begin{center}
	Tuan Vo
\end{center}
\begin{center}
	$15^{\text{th}}$ December 2021
\end{center}
%------------------------------------------------------------------------------
\section{Breaking time for an arbitrary flux function}
Consider arbitrary convex scalar equation of conservation laws 
\begin{equation}
	u_{t} + f(u)_{x} = 0, 
\end{equation}
where $f(u)$ is convex, i.e. $f''(u) > 0$. Then, smooth flux function $f(u)$ leads to
\begin{equation}
	u_{t} + f'(u) u_{x} = 0.
\end{equation}
Arbitrarily consider one characteristic line passing the point with initial condition in space and time $x_{0}(t=0)$. 
% Likewise, the neighborhood of this point reads $x_{0} + \delta x$.\\
The first characteristic line passing point $(x_{0_{1}}, 0)$ reads
\begin{equation}\label{eq:firstline}
	x =  x_{0_{1}} + f'(u_{0}(x_{0_{1}}))\,t                                                              
\end{equation}
The second characteristic line passing point $(x_{0_{2}}, 0)$  reads
\begin{equation}\label{eq:secondline}
	x =  x_{0_{2}} + f'(u_{0}(x_{0_{2}}))\,t                                                              
\end{equation}
% \begin{align}\label{eq:secondline}
% 	x & = (x_{0} + \delta x) + f'(u_{0}(x_{0} + \delta x))\,t                                       \notag \\
% 	  & = (x_{0} + \delta x) + f'(u_{0}(x_{0}) + u'_{0}(x_{0})\delta x + \mathcal{O}(\delta x^2))t \notag  \\
% 	  & = (x_{0} + \delta x) + f'(u_{0}(x_{0}) + u'_{0}(x_{0})\delta x + \mathcal{O}(\delta x^2))t
% \end{align}
Equalizing the two characteristic lines from \eqref{eq:firstline} and \eqref{eq:secondline} yields
% \begin{align*}
% 	x_{0} + f'(u_{0}(x_{0})) \, t = 
% 	(x_{0} + \delta x) + f'(u_{0}(x_{0}) + u'_{0}(x_{0})\delta x + \mathcal{O}(\delta x^2)) \, t
% \end{align*}
\begin{align}
	x_{0_{1}} + f'(u_{0}(x_{0_{1}}))\,t = x_{0_{2}} + f'(u_{0}(x_{0_{2}}))\,t  
\end{align}
which leads to
\begin{equation}
	t=-\frac{x_{0_{1}} - x_{0_{2}}}{f'(u_{0}(x_{0_{1}})-f'(u_{0}(x_{0_{2}})}
\end{equation}
The breaking time requires the following condition
\begin{align}
	t & =-\frac{1}{\displaystyle \frac{f'(u_{0}(x_{0_{1}})-f'(u_{0}(x_{0_{2}})}{x_{0_{1}} - x_{0_{2}}}} \notag \\
	  & =-\frac{1}{\displaystyle \frac{d}{dx}f'(u_{0}(x))}
\end{align}
The \emph{first} breaking time yields
\begin{equation}
	\therefore\quad
	\boxed{
		T_{b}=-\frac{1}{\displaystyle \min_{x\in\mathbb{R}} \frac{d}{dx}f'(u_{0}(x))}
	}
\end{equation}
\clearpage
\begin{example}
	Compute the breaking time of Burgers' equation given different initial conditions as follows.
\end{example}
\inputfig{floats/characsin}{characsin}
\clearpage
%------------------------------------------------------------------------------
\section{Riemann's problem}
\begin{example}
	Non-zero initial condition applied on $u_{L}$ and $u_{R}$.
\end{example}
Consider the two cases as follows
\begin{align}
	\begin{cases}
		\begin{aligned}
			u_{L} = \alpha, \qquad x<0, \\
			u_{R} = \beta, \qquad x>0,   
		\end{aligned}
	\end{cases}
\end{align}
and
\begin{align}
	\begin{cases}
		\begin{aligned}
			u_{L} = \beta, \qquad x<0, \\
			u_{R} = \alpha, \qquad x>0,   
		\end{aligned}
	\end{cases}
\end{align}
where $\alpha,\beta \in \mathbb{R}\backslash\left\{0\right\}$.
\inputfig{floats/charac12182181}{charac12182181}
\clearpage
\begin{example}
	Zero initial condition applied either on $u_{L}$ or $u_{R}$.
\end{example}
Consider the two cases as follows
\begin{align}
	\begin{cases}
		\begin{aligned}
			u_{L} = \alpha, \qquad x<0, \\
			u_{R} = 0, \qquad x>0,   
		\end{aligned}
	\end{cases}
\end{align}
and
\begin{align}
	\begin{cases}
		\begin{aligned}
			u_{L} = 0, \qquad x<0, \\
			u_{R} = \alpha, \qquad x>0,   
		\end{aligned}
	\end{cases}
\end{align}
where $\alpha \in \mathbb{R}\backslash\left\{0\right\}$.
\inputfig{floats/charac02082080}{charac02082080}
% %------------------------------------------------------------------------------
% \section{Riemann's problem}
% %------------------------------------------------------------------------------
% \section{Krushkov}
\clearpage
%------------------------------------------------------------------------------
\section{Total variation (TV)}
(Useful later on for structure of FVM, i.e. Total-variation-diminishing (TVD))
\begin{example}
	Let $f$ be defined as follows
	\begin{align}
		f:
		\begin{cases}
			\mathbb{R} \rightarrow \mathbb{R}, \\
			x \mapsto
			f(x) := 
			\begin{cases}
				x, \quad   & \quad  0<x<1,                 \\
				2, \quad   & \quad  1\leq x < 2,           \\
				1+x, \quad & \quad  2\leq x < 4,           \\
				0, \quad   & \quad \text{everywhere else}.
			\end{cases}
		\end{cases}
	\end{align}
	Find the total variation of $f$.
	
\end{example}
\inputfig{floats/totalvariation}{totalvariation}

The total variation of $f$ is computed as follows
\begin{equation}
	\begin{aligned}
		TV(f)   = \int_{0}^1|f^{'}|dx                                                                
		 & + \lim_{h\to 0} \int_{1-h}^1\frac{|f(x+h) - f(x)|}{h}dx
		+\int_{1}^2|f^{'}|dx                                                                    \\
		 & + \lim_{h\to 0} \int_{2-h}^{2}\frac{|f(x+h) - f(x)|}{h}dx  +  \int_{2}^{4}|f^{'}|dx  \\
		 & + \lim_{h\to 0} \int_{4-h}^{4}\frac{|f(x+h) - f(x)|}{h}dx                           
		= 1 + 1 + 0 + 1 + 2 + 5  = 10 \notag
	\end{aligned}
\end{equation}
\begin{equation}
	\therefore\quad
	\boxed{
		TV(f) = 10.
	}
\end{equation}
%------------------------------------------------------------------------------
\begin{example}
	Let $f$ be defined as follows
	\begin{align}
		f:
		\begin{cases}
			\mathbb{R} \rightarrow \mathbb{R}, \\
			x \mapsto
			f(x) := 
			\begin{cases}
				-x+1, \quad & \quad  0<x<1,                 \\
				2, \quad    & \quad  1\leq x < 2,           \\
				-x+7, \quad & \quad  2\leq x < 4,           \\
				0, \quad    & \quad \text{everywhere else}.
			\end{cases}
		\end{cases}
	\end{align}
	Find the total variation of $f$.
\end{example}
\inputfig{floats/totalvariation_inv}{totalvariation_inv}

The total variation of $f$ is computed as follows
\begin{equation}
	\begin{aligned}
		TV(f)  = \dots
		%  & = \int_{0}^1|f^{'}|dx                                                               \\
		%  & + \lim_{h\to 0} \int_{1-h}^1\frac{|f(x+h) - f(x)|}{h}dx
		% +\int_{1}^2|f^{'}|dx                                                                   \\
		%  & + \lim_{h\to 0} \int_{2-h}^{2}\frac{|f(x+h) - f(x)|}{h}dx  +  \int_{2}^{4}|f^{'}|dx \\
		%  & + \lim_{h\to 0} \int_{4-h}^{4}\frac{|f(x+h) - f(x)|}{h}dx                           \\
		%  & = 1 + 1 + 2 + 3 + 2 + 3 = 12.
	\end{aligned}
\end{equation}
\newline
\newline
\newline
\newline
\newline
\newline
\newline
\newline
\newline
\begin{equation}
	\therefore\quad
	\boxed{
		TV(f) = 12
	}
\end{equation}
%------------------------------------------------------------------------------
\begin{example}
	Let $f$ be defined as follows
	\begin{align}
		f:
		\begin{cases}
			\mathbb{R} \rightarrow \mathbb{R}, \\
			x \mapsto
			f(x) := 
			\begin{cases}
				-x+1, \quad           & \quad  0<x<1,                 \\
				2, \quad              & \quad  1\leq x < 2,           \\
				\sin(-\pi x)+4, \quad & \quad  2\leq x < 4,           \\
				0, \quad              & \quad \text{everywhere else}.
			\end{cases}
		\end{cases}
	\end{align}
	Find the total variation of $f$.
\end{example}
\inputfig{floats/totalvariation_sin}{totalvariation_sin}

The total variation of $f$ is computed as follows
\begin{equation}
	\begin{aligned}
		TV(f)  = \dots
		% 	 & = \int_{0}^1|f^{'}|dx                                                               \\
		% 	 & + \lim_{h\to 0} \int_{1-h}^1\frac{|f(x+h) - f(x)|}{h}dx
		% 	+\int_{1}^2|f^{'}|dx                                                                   \\
		% 	 & + \lim_{h\to 0} \int_{2-h}^{2}\frac{|f(x+h) - f(x)|}{h}dx  +  \int_{2}^{4}|f^{'}|dx \\
		% 	 & + \lim_{h\to 0} \int_{4-h}^{4}\frac{|f(x+h) - f(x)|}{h}dx                           \\
		% 	 & = 1 + 1 + 2 + 0 + 2 + 1 + 2 + 1 + 4 = 14.
	\end{aligned}
\end{equation}
\newline
\newline
\newline
\newline
\newline
\newline
\newline
\newline
\newline
\begin{equation}
	\therefore\quad
	\boxed{
		TV(f) = 14
	}
\end{equation}
%------------------------------------------------------------------------------
% \section{Domain of dependence}
\clearpage
%------------------------------------------------------------------------------
\section{Manipulating conservation laws}
\emph{Manipulation of conservation laws} by transforming the differential form into another differential equation, 
which appears lately as equivalently as the original one, \textbf{may not} result in an equivalent differential
equation with respect to weak solutions.
\begin{example}
	Manipulating conservation laws
\end{example}
Consider the Burgers' equation
\begin{equation}\label{eq:Burgers}
	u_{t} + \left(\frac{1}{2}u^2\right)_{x} = 0.
\end{equation}
\emph{Manipulation} of this PDE by multiplying \eqref{eq:Burgers} by $2u$ we obtain
\begin{equation}\label{eq:Burgers1}
	2uu_{t} + 2u\left(\frac{1}{2}u^2\right)_{x} = 0,
\end{equation}
which can be recast into
\begin{equation}\label{eq:Burgers2}
	w_{t} + \left(\frac{2}{3}w^{\frac{3}{2}}\right)_{x} = 0
\end{equation}
where $w:=u^2$, and the flux function become $\displaystyle f(w) = \frac{2}{3}w^{\frac{3}{2}}$. 
The solutions to \eqref{eq:Burgers} and \eqref{eq:Burgers2}, respectively, read
\begin{align}
	u(x,t) & = g_{0}(x-ut),       \\
	w(x,t) & = g_{0}(x-w^{1/2}t),
\end{align}
which are precisely the same. Note in passing that $w^{1/2} = u$. Although \eqref{eq:Burgers} and \eqref{eq:Burgers2}
have the same solution, their weak solutions are different from each other,
i.e. by considering the \emph{Riemann}'s problem with $u_{L}>u_{R}$
together with checking the \emph{Rankine-Hugoniot} condition, we obtain unique weak solution 
with a shock travelling at speeds computed as follows
\begin{equation}
	s_{\text{Burgers}} = \frac{[f(u)]}{[u]} = \dots
\end{equation}
Likewise, for the \emph{manipulated Burgers} we obtain
\begin{equation}
	s_{\text{manipulated Burgers}} = \frac{[f(w)]}{[w]} = \dots
\end{equation}
\clearpage
%------------------------------------------------------------------------------
\section{Linear hyperbolic systems}
\begin{example}
	Consider the natural 1D second order wave equation 
	\begin{equation}
		u_{tt} = \alpha^2 u_{xx},\quad x\in\mathbb{R}, 
	\end{equation}
	with initial condition
	\begin{align}
		\begin{cases}
			\begin{aligned}
				u(x,0)     & = u_{0}(x), \\
				u_{t}(x,0) & = u_{1}(x). \\
			\end{aligned}
		\end{cases}
	\end{align}
\end{example}
Approach: By introducing 
\begin{align}
	\begin{cases}
		\begin{aligned}
			v & = u_{x}, \\
			w & = u_{t},
		\end{aligned}
	\end{cases}
\end{align}
one obtains 
\begin{align}
	\begin{pmatrix} v\\ w\end{pmatrix}_{t}
	+
	\begin{pmatrix} -w \\ -\alpha^2 v \end{pmatrix}_{x}
	= 0,
\end{align}
which can be recast into
\begin{align}
	\begin{pmatrix} v\\ w\end{pmatrix}_{t}
	+
	\begin{pmatrix} 0 & -1 \\ -\alpha^2 & 0 \end{pmatrix}
	\begin{pmatrix} v\\ w\end{pmatrix}_{x}
	= 0.
\end{align}
The initial condition becomes
\begin{align}
	\begin{cases}
		\begin{aligned}
			v(x,0) & = u'_{0}(x), \\
			w(x,0) & = u_{1}(x).  \\
		\end{aligned}
	\end{cases}
\end{align}
%------------------------------------------------------------------------------
\clearpage
\begin{example}
	Consider the \textbf{linearized shallow water equations}
	\begin{align}
		\begin{pmatrix} u\\ \varphi\end{pmatrix}_{t}
		+
		\begin{pmatrix} \bar{u} & 1 \\ \bar{\varphi} & \bar{u} \end{pmatrix}
		\begin{pmatrix} u \\ \varphi\end{pmatrix}_{x}
		= 0,
	\end{align}
	where $\bar{u}, \bar{\varphi} \in \mathbb{R}$ are given, 
	and sufficient initial conditions for $u$ and $\varphi$ are provided.
\end{example}

% % Integral form
% % \begin{equation}
% % 	\frac{d}{dt}\int_{\Omega}u\,d\Omega 
% % 	= -\int_{\partial\Omega}f(u)\cdot n\,d\partial\Omega
% % \end{equation}
% % Differential form
% % \begin{equation}
% % 	u_{t} + \nabla\cdot f(u) = 0 
% % \end{equation}
% \emph{Master balance principle}: due to the fact that the mathematical structure of the fundamental balance relations,
% namely of mass, linear momentum, moment of momentum, energy and entropy, is in principle identical, they can be 
% formulated within the concise shape of a master balance. To begin with, let $\Psi$ and $\BPsi$ be volume specific
% scalar and vector value densities of a physical quantity to be balanced, respectively. Then, the general balance relations take 
% the global \textbf{integral form} as follows 
% \begin{align}
% 	\frac{d}{dt}\int_{\Omega} \Psi \,d\Omega  & = 
% 	-\int_{\partial\Omega}\Bphi\cdot\Bn\,d\partial\Omega 
% 	+ \int_{\Omega}\sigma\,d\Omega
% 	+ \int_{\Omega}\hat{\Psi}\,d\Omega             \\
% 	\frac{d}{dt}\int_{\Omega} \BPsi \,d\Omega & = 
% 	-\int_{\partial\Omega}\BPhi\cdot\Bn\,d\partial\Omega 
% 	+ \int_{\Omega}\Bsigma\,d\Omega
% 	+ \int_{\Omega}\hat{\BPsi}\,d\Omega
% \end{align}
% where $(\Bphi\cdot\Bn)$ and $(\BPhi\cdot\Bn)$ describe action at the vicinity; $\sigma$ and 
% $\Bsigma$ present for action from a distance; and $\hat{\Psi}$ and $\hat{\BPsi}$ for production or nucleation.
% \begin{example}
% 	Conservation of mass: A derivation from integral to differential form.
% \end{example}
% \inputfig{floats/massx1x2}{massx1x2}
% Let $x$ represent the distance along the tube and let $\rho(x,t)$ be the mass density of the 
% fluid at point $x$ and time $t$. Then, the total mass $M$ in the section 
% $[x_{1},x_{2}]$ at time $t$ is defined as follows
% \begin{equation}
% 	M := \int_{x_{1}}^{x_{2}} \rho(x,t)\,dx.
% \end{equation}
% Assume that the walls of the tube are impermeable and mass is neither created nor destroyed, 
% this total mass $M$ in this interval $[x_1,x_2]$ varies over time only due to the fluid flowing
% across at the boundaries $x=x_{1}$ and $x=x_{2}$. 
% % Besides, the flow rate (also called \emph{rate of flow}, or \emph{flux}) is denoted as $f(\rho)$.
% Now let $v(x,t)$ be the measured velocity of flow at point $x$ and time $t$.
% Then, the flow rate (also called \emph{rate of flow}, or \emph{flux})
% passing the end points of the interval $[x_{1},x_{2}]$ at time $t$ is given by
% \begin{equation}
% 	\rho(x_{1},t)v(x_{1},t)-\rho(x_{2},t)v(x_{2},t).
% \end{equation}
% The rate of change of mass in $[x_{1},x_{2}]$ given by the difference of fluxes at 
% $x_1$ and $x_2$ flow rate has to be equal with the time rate of change of total mass
% \begin{equation}\label{eq:relation0}
% 	\boxed{
% 		\frac{d}{dt}\int_{x_{1}}^{x_{2}} \rho(x,t)\,dx 
% 		% \stackrel{!}{=} f(\rho(x_{1},t))-f(\rho(x_{2},t))
% 		= \rho(x_{1},t)v(x_{1},t)-\rho(x_{2},t)v(x_{2},t)
% 	}
% \end{equation}
% which is the \textbf{integral form} of the conservation law. Next, note that the following relation holds
% \begin{equation}\label{eq:relation1}
% 	\int_{x_{1}}^{x_{2}} \frac{\partial}{\partial x}f(\rho)\,dx 
% 	= f(\rho(x_{2},t))-f(\rho(x_{1},t)).
% \end{equation}
% Substitution of \eqref{eq:relation1} into \eqref{eq:relation0} leads to
% \begin{equation}\label{eq:relation2}
% 	\frac{d}{dt}\int_{x_{1}}^{x_{2}} \rho(x,t)\,dx 
% 	= -\int_{x_{1}}^{x_{2}}\frac{\partial}{\partial x}f(\rho)\,dx.
% \end{equation}
% If $\rho(x,t)$ is sufficiently smooth, time derivative on the LHS of \eqref{eq:relation2}
% can be brought inside the integral. Then, an arrangement of \eqref{eq:relation2} yields
% \begin{equation}\label{eq:relation3}
% 	\int_{x_{1}}^{x_{2}}
% 	\left(\frac{\partial}{\partial t}\rho(x,t) + \frac{\partial}{\partial x}f(\rho)\right)\,dx 
% 	= 0.
% \end{equation}
% Since \eqref{eq:relation3} holds for any choice of $x_{1}$ and $x_{2}$, the integrant must be vanished
% everywhere. Therefore, one obtains the \textbf{differential form} of the conservation law
% \begin{equation}\label{eq:deri}
% 	\therefore \quad
% 	\boxed{\rho_{t} + f(\rho)_{x} = 0}
% \end{equation}
% % \begin{enumerate}
% % 	\item gas dynamics
% % 	\item oil-water mixtures
% % 	\item plasmas
% % 	\item shallow water flows
% % 	\item meteology
% % 	\item traffic
% % 	\item ...
% % \end{enumerate}
% % \begin{definition}
% % 	Cauchy problem
% % \end{definition}
% % Cauchy problem\\
% % Riemann problem\\
% % \begin{equation}
% % 	u_{t} + f(u)_{x} = 0 
% % \end{equation}
% \begin{example}
% 	Different definitions of flux function yields different applications
% \end{example}
% \begin{enumerate}
% 	\item Advection equation: \eqref{eq:deri} with flux function
% 	      $\displaystyle  f(\rho) =  a\rho, \ a\in \mathbb{R^+}$, taking form
% 	      \begin{equation}
% 		      \boxed{
% 			      \rho_{t} + a\rho_{x} = 0
% 		      }
% 	      \end{equation}
% 	\item \emph{Inviscid} Burgers' equation: \eqref{eq:deri} with flux function
% 	      $\displaystyle  f(\rho) =  \frac{1}{2}\rho^2$, yielding
% 	      \begin{equation}
% 		      \boxed{
% 			      \rho_{t} + \rho\rho_{x} = 0
% 		      }
% 	      \end{equation}
% 	      which is used to illustrate the distortion of waveform in simple waves.
% 	      Meanwhile, \emph{viscous} Burgers' equation is a scalar parabolic PDE, taking the form
% 	      $$\rho_t + \rho\rho_{x} = \varepsilon\rho_{xx}$$
% 	      which is the simplest differential model for a fluid flow.
% 	\item Lighthill-Whitham-Richards (LWR) equation: \eqref{eq:deri} with flux function
% 	      $$\displaystyle  f(\rho) =  u_{max}\,\rho\left(1-\frac{\rho}{\rho_{max}}\right)$$
% 	      which is used to model traffic flow; $u_{max}$ is the given maximal speed of vehicle to be allowed, 
% 	      $\rho_{max}$ the given maximal density of vehicle; taking the form
% 	      \begin{equation}
% 		      \rho_{t} + \left(u_{max}-\frac{2u_{max}}{\rho_{max}}\rho\right)\rho_{x} = 0
% 	      \end{equation}
% 	\item (Typical) Buckley-Leverett petroleum equation: \eqref{eq:deri} with flux function
% 	      $$\displaystyle f(\rho) =  \frac{\rho^2}{\rho^2 + (1-\rho)^2\mu_{\text{water}}/\mu_{\text{oil}}}$$
% 	      which is a one-dimensional model for a two-phase flow; $\rho$ stands for water saturation
% 	      and takes the value in the interval $[0,1]$.
% 	\item \emph{Euler} equations of gas dynamics in 1D
% 	      %   \begin{align}
% 	      %       \begin{cases}
% 	      % 	      \begin{aligned}
% 	      % 		      \rho_{t}     & + (\rho v)_x & = 0 \\
% 	      % 		      (\rho v)_{t} & + (\rho v^2 + p)_x & = 0 \\
% 	      % 		      E_{t}        & + (v(E+p))_x & = 0
% 	      % 	      \end{aligned}
% 	      %       \end{cases}
% 	      %   \end{align}
% 	      \begin{align}
% 		      \begin{pmatrix} \rho\\\rho v \\ E \end{pmatrix}_{t}
% 		      +
% 		      \begin{pmatrix} \rho v\\\rho v^2 + p \\ v(E+p) \end{pmatrix}_{x}
% 		      =
% 		      \begin{pmatrix} 0\\ 0 \\ 0 \end{pmatrix}
% 	      \end{align}
% 	      $\rightarrow$ The pressure $p$ should be specified as a given function of 
% 	      mass density $\rho$, linear momentum $\rho v$ and/or energy $E$ in order to 
% 	      fulfill the \emph{closure} problem.
% 	      Such additional equation is called \emph{equation of state} (normally in fluid), or 
% 	      \emph{constitutive equation} (normally in solid). 
% 	\item \emph{Euler} equations of gas dynamics in 2D
% 	      \begin{align}
% 		      \begin{pmatrix} \rho\\ \rho u\\ \rho v \\ E \end{pmatrix}_{t}
% 		      +
% 		      \begin{pmatrix} \rho u\\ \rho u^2 + p \\ \rho uv \\ u(E+p) \end{pmatrix}_{x}
% 		      + 
% 		      \begin{pmatrix} \rho v\\ \rho uv \\ \rho v^2 + p \\ v(E+p) \end{pmatrix}_{y}
% 		      =
% 		      \begin{pmatrix} 0 \\ 0 \\ 0 \\ 0 \end{pmatrix}
% 	      \end{align}
% \end{enumerate}
% % \begin{recall}
% % 	Convexity
% % \end{recall}
% \begin{definition}
% 	Cauchy problem and Riemann problem
% \end{definition}
% Let $u$ be a conserved unknown quantity to be modelled and defined as follows
% \begin{align}\label{eq:conservedquantity}
% 	u:
% 	\begin{cases}
% 		\mathbb{R}\times\mathbb{R}^+ \rightarrow \mathbb{R}, \\
% 		(x,t) \mapsto u(x,t).
% 	\end{cases}
% \end{align}
% \emph{Cauchy's problem} is simply the pure initial value problem (IVP), e.g. 
% find a function $u$ holding \eqref{eq:conservedquantity} that is a solution of
% $\eqref{eq:cauchy}_1$ satisfying the initial condition $\eqref{eq:cauchy}_2$:
% \begin{equation}\label{eq:cauchy}
% 	\boxed{
% 		\begin{aligned}
% 			\partial_{t}u + \partial_{x}f(u) & = 0,        \\
% 			u(x,0)                           & = u_{0}(x).
% 		\end{aligned}
% 	}
% \end{equation}
% \emph{Riemann's problem} is simply the conservation law together with particular initial data
% consisting of two constant states separated by a single discontinuity, e.g.
% find a function $u$ holding \eqref{eq:conservedquantity} that is a solution of
% $\eqref{eq:riemann}_1$ satisfying the initial condition $\eqref{eq:riemann}_2$:
% \begin{equation}\label{eq:riemann}
% 	\boxed{
% 		\begin{aligned}
% 			\partial_{t}u + \partial_{x}f(u) & = 0,        \\
% 			u(x,0)                           & = u_{0}(x)=
% 			\begin{cases}
% 				u_{L}, \quad x<0, \\
% 				u_{R}, \quad x>0.
% 			\end{cases}
% 		\end{aligned}
% 	}
% \end{equation}
% % \begin{definition}
% % 	Hyperbolic
% % \end{definition}
% % Let $\Omega$ be 
% % \inputfig{floats/comparehyperbolic}{comparehyperbolic}

% Let $f:\mathbb{R}\rightarrow\mathbb{R}$ be a $C^1$ function. We consider the \emph{Cauchy} problem
% \begin{align}\label{eq:charac}
% 	\frac{\partial u}{\partial t} + \frac{\partial}{\partial x}f(u) & = 0,        \qquad  x\in\mathbb{R}, t>0, \\
% 	u(x,0)                                                          & = u_{0}(x), \qquad  x\in\mathbb{R}.
% \end{align}
% By setting $a(u) = f'(u)$, and letting $u$ be a classical solution of \eqref{eq:charac},
% one obtains from \eqref{eq:charac} the \emph{non-conservative} form as follows
% \begin{equation}
% 	\frac{\partial u}{\partial t} + a(u)\frac{\partial u}{\partial x} = 0.
% \end{equation}
% The characteristic curves associated with \eqref{eq:charac} are defined as the integral curves
% of the differential equation
% \begin{equation}
% 	\frac{dx}{dt} = a(u(x(t),t)).
% \end{equation}
% \begin{proposition}
% 	Assume that u is a smooth solution of the Cauchy problem. The characteristic curves 
% 	are straight lines along which u is constant.
% \end{proposition}
% \begin{proof}
% 	Consider a characteristic curve passing through the point $(x_{0},0)$, i.e.
% 	a solution of the ordinary differential system
% 	\begin{align}
% 		\begin{cases}\displaystyle
% 			\frac{dx}{dt} & = a(u(x(t),t)), \\
% 			x(0)          & = x_{0}.
% 		\end{cases}
% 	\end{align}
% 	It exists at least on a small time interval $[0,t_{0})$. 
% 	Along such a curve, u is constant since
% 	\begin{align*}
% 		\frac{d}{dt}u(x(t),t) 
% 		 & = \frac{\partial u}{\partial t}(x(t),t)
% 		+ \frac{\partial u}{\partial x}(x(t),t) \frac{dx}{dt}(t)                                    \\
% 		 & =\left(\frac{\partial u}{\partial t} + a(u)\frac{\partial u}{\partial x}\right)(x(t),t) 
% 		= 0.
% 	\end{align*}
% 	Therefore, the characteristic curves are straight lines whose constant slopes depend
% 	on the initial data. As a result, the characteristic straight line passing through
% 	the point $(x_{0},0)$ is defined by the equation
% 	\begin{equation}
% 		\boxed{
% 			x = x_{0} + ta(u_{0}(x_{0}))
% 		}
% 	\end{equation}
% \end{proof}
% \begin{example}
% 	Determine characteristics of the inviscid Burgers' equation with initial conditions given as
% 	\begin{equation}
% 		u(x,0) = u_{0}(x) =
% 		\begin{cases}
% 			\begin{aligned}
% 				 & 1,   & x\leq 0,        &                        \\
% 				 & 1-x, & 0\leq x \leq 1, & \quad x\in \mathbb{R}, \\
% 				 & 0,   & x\geq 1.        & 
% 			\end{aligned}
% 		\end{cases}
% 	\end{equation}
% \end{example}
% By using the method of characteristics, the solution can be solved up to the time when
% those characteristics first intersect with each other, i.e. so-called breaking time or shock.
% Since the $f'(u) = u$ in the inviscid Burgers' equation, it yields the characteristics 
% passing through the point $(x_{0},t)$
% \begin{equation}
% 	x = x_{0} + t\,u_{0}(x_{0})
% \end{equation}
% which leads to
% \begin{equation}
% 	x(x_{0},t) = 
% 	\begin{cases}
% 		\begin{aligned}
% 			 & x_{0}+t,          & x_{0}\leq 0,          \\
% 			 & x_{0}+t(1-x_{0}), & 0\leq x_{0} \leq 1,   \\
% 			 & x_{0},            & x_{0}\geq 1.         
% 		\end{aligned}
% 	\end{cases}
% \end{equation}
% (Sketch)
% %------------------------------------------------------------------------------
% % \section{General strategy for solving the first-order PDE}
% % Supposed 
% %------------------------------------------------------------------------------
% % \section{Rankine-Hugoniot (RH) jump condition}

% % Characteristic curves
% % \begin{equation}
% % 	\dot{x} = f'(u)\\
% % \end{equation}
% % \begin{align*}
% % 	\frac{d}{dt}u(x(t),t) 
% % 	= \frac{\partial u}{\partial t}(x(t),t)
% % 	+ \frac{\partial u}{\partial x}(x(t),t) \frac{dx}{dt}(t)
% % \end{align*}


% % \inputfig{floats/multistructures}{multistructures}

\end{document}

