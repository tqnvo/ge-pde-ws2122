\documentclass[12pt]{article}
\usepackage{anysize}
\usepackage[ngerman,english]{babel}
\marginsize{3.5cm}{2.5cm}{1cm}{2cm}

\usepackage[utf8]{inputenc}
\usepackage[english]{babel}
\usepackage{amsmath}
\usepackage{amsthm}
\usepackage{amsfonts}
\usepackage{scalerel,amssymb}  % for mathbb
%---------------------------------------------------------
\usepackage{authblk}
\usepackage{blindtext}
\usepackage{graphicx}
\usepackage[caption=true]{subfig}
\usepackage[
  colorlinks = true,
  citecolor = red,
  % linkcolor = darkblue, % internal references
  % urlcolor = darkblue,
]{hyperref}
\usepackage{cleveref}
\crefname{figure}{Figure}{Figures}
\crefname{equation}{Eq.}{Eqs.}
%---------------------------------------------------------
% \newcommand{\begin{equation}}{\eqbeg} 
% \newcommand{\end{equation}}{\eqend}
\usepackage{bm}
\usepackage{dfcmd}
% \newcommand{\Bvarepsilon}{\bm\varepsilon} 
% \newcommand{\Bx}{\bm x} 
% \newcommand{\Bu}{\bm u} 
% \newcommand{\Bv}{\bm v} 
% \newcommand{\Bg}{\bm g} 
% \newcommand{\Bb}{\bm b} 
\newcommand{\divv}{\text{divv}} 
%---------------------------------------------------------
\usepackage{lineno}
% \linenumbers     

\usepackage{float}
\usepackage{psfrag}

\usepackage{fontawesome}
\usepackage{relsize}

%---------------------------------------------------------
%For \toprule \midrule \bottomrule in table environment
\usepackage{booktabs}


%Page numbering in style 1/3...
\usepackage{lastpage}  
\usepackage{hyperref}
\makeatletter
\renewcommand{\@oddfoot}{\hfil 
% Aachen, November $04^{th}$, 2021 \hspace{300pt} 
PDE $\cdot$ GE15 $\cdot$ WS21/22 
\hspace{280pt} 
\thepage/\pageref{LastPage}\hfil}
\makeatother
%------------------------------------------------------------------------------

\begin{document}
\begin{center}
	\section*{Global Exercise - 15}
\end{center}
\begin{center}
	Tuan Vo
\end{center}
\begin{center}
	$02^{\text{nd}}$ February 2022
\end{center}

%------------------------------------------------------------------------------
\section{Monotone $\rightarrow$ $L_1$-Contracting $\rightarrow$ TVD $\rightarrow$ Mon.Pre.}
\begin{example}
	abc
\end{example}
Roe'solver:\\
P.L.Roe [1981], 
\href{https://www.sciencedirect.com/science/article/pii/0021999181901285}
{Approximate Riemann solvers, parameter vectors, and difference schemes}.

\clearpage
%------------------------------------------------------------------------------
\section{A small note about monotonicity: \emph{Upwind} scheme}
% \begin{example}
% 	abcdef
% \end{example}
\inputfig{floats/monotoneUW}{monotoneUW}

\clearpage
%------------------------------------------------------------------------------
\section{Total variation diminishing (TVD)}
\begin{example}
	abc
\end{example}

\clearpage
%------------------------------------------------------------------------------
\section{Limiter}
\begin{example}
	Examine the $1^{st}$-order-converged LF and the $2^{nd}$-order-converged LW.
\end{example}
\inputfig{floats/LFvsLW}{LFvsLW}

\clearpage
%------------------------------------------------------------------------------
\section{Review Riemann's problem and Godunov's solver}
\begin{example}
	abc
\end{example}

\clearpage
\begin{example}
	Examine $u_{L} > u_{R}$.
\end{example}
\inputfig{floats/RiemannAllCasesLR}{RiemannAllCasesLR}
Schock solution

\clearpage
\begin{example}
	Examine $u_{L} < u_{R}$.
\end{example}
\inputfig{floats/RiemannAllCasesRL}{RiemannAllCasesRL}
Rarefaction solution


% %------------------------------------------------------------------------------
% \section{A small remark for consistent numerical flux function used
%   in Upwind scheme - FVM}
% \begin{example}
% 	Determine consistent numerical flux function for Upwind scheme.
% \end{example}
% Recall the conservation form used for FVM reads
% \begin{align}\label{eq:consevationformgeneralUW}
% 	u_{j}^{n+1} = u_{j}^{n} + \frac{\Delta t}{\Delta x}
% 	\left(
% 	\widetilde{f}_{j-1/2}^{n} - \widetilde{f}_{j+1/2}^{n}
% 	\right).
% \end{align}
% Meanwhile, the Upwind scheme with point-to-the-left stencils reads
% \begin{equation}\label{eq:UWform}
% 	\boxed{
% 		u_{j}^{n+1} = 
% 		u_{j}^{n}
% 		+\frac{\Delta t}{\Delta x}
% 		\left(
% 		f(u_{j-1}^{n}) - f(u_{j}^{n})
% 		\right).
% 	}
% \end{equation}
% Next, we would like to write \eqref{eq:UWform} in terms of \eqref{eq:consevationformgeneralUW}. 
% Since the two formulations are already identical, the numerical flux function
% $\widetilde{f}_{j-1/2}^{n}$ and $\widetilde{f}_{j+1/2}^{n}$
% are recognized directly as follow
% \begin{align}
% 	\widetilde{f}_{j-1/2}^{n}
% 	 & = \widetilde{f}_{j-1/2}^{n} \left(u_{j-1}^{n}, u_{j}^{n}\right)
% 	= f(u_{j-1}^{n}),                                                   \label{eq:consistencycheck11} \\
% 	\widetilde{f}_{j+1/2}^{n}
% 	 & = \widetilde{f}_{j+1/2}^{n} \left(u_{j}^{n}, u_{j+1}^{n}\right)
% 	= f(u_{j}^{n}).  \label{eq:consistencycheck12}
% \end{align}
% Then, the consistency of numerical flux functions
% are checked as follows
% \begin{align}
% 	\widetilde{f}_{j-1/2}^{n} \left( \beta, \beta\right) = f(\beta), \quad \checkmark \\
% 	\widetilde{f}_{j+1/2}^{n} \left( \beta, \beta\right) = f(\beta), \quad \checkmark 
% \end{align}
% which is automatically satisfied $\forall \beta \in \mathbb{R}$ in case of constant flow. 
% Likewise, in case of the Upwind scheme with point-to-the-right stencils,
% we obtain the same structure, as follows
% \begin{equation}\label{eq:UWform2}
% 	\boxed{
% 		u_{j}^{n+1} = 
% 		u_{j}^{n}
% 		+\frac{\Delta t}{\Delta x}
% 		\left(
% 		f(u_{j}^{n}) - f(u_{j+1}^{n})
% 		\right),
% 	}
% \end{equation}
% which is the Upwind scheme with point-to-the-right stencils. 
% Next, since the formulation \eqref{eq:UWform2} is already identical 
% with \eqref{eq:consevationformgeneralUW}, 
% the numerical flux function
% $\widetilde{f}_{j-1/2}^{n}$ and $\widetilde{f}_{j+1/2}^{n}$
% are recognized directly as follow
% \begin{align}
% 	\widetilde{f}_{j-1/2}^{n}
% 	 & = \widetilde{f}_{j-1/2}^{n} \left(u_{j-1}^{n}, u_{j}^{n}\right)
% 	= f(u_{j}^{n}),                                                 \label{eq:consistencycheck21} \\
% 	\widetilde{f}_{j+1/2}^{n}
% 	 & = \widetilde{f}_{j+1/2}^{n} \left(u_{j}^{n}, u_{j+1}^{n}\right)
% 	= f(u_{j+1}^{n}). \label{eq:consistencycheck22}
% \end{align}
% Checking the consistency of \eqref{eq:consistencycheck21} and \eqref{eq:consistencycheck22}
% is similar with \eqref{eq:consistencycheck11} and \eqref{eq:consistencycheck12}, as follows
% \begin{align}
% 	\widetilde{f}_{j-1/2}^{n} \left( \beta, \beta\right) = f(\beta), \quad \checkmark \\
% 	\widetilde{f}_{j+1/2}^{n} \left( \beta, \beta\right) = f(\beta), \quad \checkmark 
% \end{align}
% which is automatically satisfied $\forall \beta \in \mathbb{R}$ in case of constant flow. 
% \inputfig{floats/a4casesupwind}{a4casesupwind}

% \clearpage
% %------------------------------------------------------------------------------
% \begin{example}
% 	Comparison of numerical solutions between left-pointing and right-pointing stencils
% 	of Upwind scheme in case of $a>0$. 
% \end{example}
% \inputfig{floats/UWLRstencils}{UWLRstencils}

% \clearpage
% %------------------------------------------------------------------------------
% \section{A small remark for consistent numerical flux function used
%   in Lax-Friedrichs scheme - FVM}
% \begin{example}
% 	Determine consistent numerical flux function for Lax-Friedrichs scheme.
% \end{example}
% Recall the conservation form used for FVM reads
% \begin{align}\label{eq:consevationformgeneral}
% 	u_{j}^{n+1} = u_{j}^{n} + \frac{\Delta t}{\Delta x}
% 	\left(
% 	\widetilde{f}_{j-1/2}^{n} - \widetilde{f}_{j+1/2}^{n}
% 	\right).
% \end{align}
% Meanwhile, the Lax-Friedrichs scheme reads
% \begin{align}\label{eq:LFform}
% 	u_{j}^{n+1} = 
% 	\frac{1}{2}
% 	\left(
% 	u_{j-1}^{n} + u_{j+1}^{n}
% 	\right)
% 	-\frac{\Delta t}{2\Delta x}
% 	\left(
% 	f(u_{j+1}^{n}) - f(u_{j-1}^{n})
% 	\right).
% \end{align}
% Next, we would like to write \eqref{eq:LFform} in terms of \eqref{eq:consevationformgeneral}. 
% The derivation is done by some algebraic manipulations, as follows
% \begin{align}
% 	u_{j}^{n+1} & = 
% 	\frac{1}{2}
% 	\left(
% 	u_{j-1}^{n} + u_{j+1}^{n}
% 	\right)
% 	-\frac{\Delta t}{2\Delta x}
% 	\left(
% 	f(u_{j+1}^{n}) - f(u_{j-1}^{n})
% 	\right)                                    \notag                             \\
% 	            & =
% 	u_{j}^{n}
% 	+ \frac{1}{2} u_{j-1}^{n}  - \frac{1}{2}u_{j}^{n}
% 	+ \frac{1}{2}u_{j+1}^{n}  - \frac{1}{2}u_{j}^{n}
% 	-\frac{\Delta t}{2\Delta x} f(u_{j+1}^{n}) 
% 	+\frac{\Delta t}{2\Delta x} f(u_{j-1}^{n}) \notag                             \\	
% 	            & =
% 	u_{j}^{n}
% 	+ \frac{\Delta t}{\Delta x}
% 	\left(
% 	\frac{\Delta x}{2\Delta t}u_{j-1}^{n} 
% 	-\frac{\Delta x}{2\Delta t}u_{j}^{n} 
% 	+\frac{\Delta x}{2\Delta t}u_{j+1}^{n} 
% 	-\frac{\Delta x}{2\Delta t}u_{j}^{n} \right. \notag                           \\
% 	            & \qquad \qquad \qquad \qquad \qquad \qquad \qquad \qquad \qquad 
% 	\left.
% 	-\frac{1}{2}f(u_{j+1}^{n}) 
% 	+\frac{1}{2}f(u_{j-1}^{n}) 
% 	\right)                                     \notag                            \\
% 	            & =
% 	u_{j}^{n}
% 	+ \frac{\Delta t}{\Delta x}
% 	\left(
% 	\underbrace{\left(
% 	\frac{
% 		\Delta x}{2\Delta t}u_{j-1}^{n}
% 	-\frac{\Delta x}{2\Delta t}u_{j}^{n}
% 	+\frac{1}{2}f(u_{j-1}^{n}) 
% 	\right)}_{=:\widetilde{f}_{j-1/2}^{n}}
% 	\right.
% 	\notag                                                                        \\
% 	            & \qquad \qquad \qquad \qquad \qquad
% 	\left. -
% 	\underbrace{\left(
% 	\frac{\Delta x}{2\Delta t}u_{j}^{n}
% 	-\frac{\Delta x}{2\Delta t}u_{j+1}^{n}
% 	+\frac{1}{2}f(u_{j+1}^{n}) 
% 	\right)}_{=:\widetilde{f}_{j+1/2}^{n}}
% 	\right)\notag                                                                 \\
% 	            & =
% 	u_{j}^{n}
% 	+ \frac{\Delta t}{\Delta x}
% 	\left(
% 	\widetilde{f}_{j-1/2}^{n}
% 	-\widetilde{f}_{j+1/2}^{n}
% 	\right),
% \end{align}
% which confirms that the \emph{Lax-Friedrichs} scheme given at \eqref{eq:LFform}
% is able to be written in the \emph{conservation form}
% with the numerical flux function recognized as follows
% \begin{align}
% 	\widetilde{f}_{j-1/2}^{n} & =  
% 	\widetilde{f}_{j-1/2}^{n} \left(u_{j-1}^{n}, u_{j}^{n}\right)
% 	=
% 	\frac{\Delta x}{2\Delta t}\left(u_{j-1}^{n} - u_{j}^{n} \right)
% 	+\frac{1}{2}f(u_{j-1}^{n})      \label{eq:flux1} \\
% 	\widetilde{f}_{j+1/2}^{n} & =  
% 	\widetilde{f}_{j-1/2}^{n} \left(u_{j}^{n}, u_{j+1}^{n}\right)
% 	=
% 	\frac{\Delta x}{2\Delta t}\left(u_{j}^{n} - u_{j+1}^{n} \right)
% 	+\frac{1}{2}f(u_{j+1}^{n}) \label{eq:flux2}
% \end{align}
% However, these numerical flux functions $\widetilde{f}(\cdot,\cdot)$
% are not consistent with the original flux function $f(\cdot)$, which 
% can be checked for the case of constant flow, as follows
% \begin{align}
% 	\eqref{eq:flux1} \Leftrightarrow
% 	\widetilde{f}_{j-1/2}^{n} \left(\beta, \beta\right) & =  
% 	\frac{\Delta x}{2\Delta t}\left(\beta - \beta \right)
% 	+\frac{1}{2}f(\beta)                                                
% 	=\frac{1}{2}f(\beta) \quad \lightning                     \\
% 	\eqref{eq:flux2} \Leftrightarrow
% 	\widetilde{f}_{j+1/2}^{n} \left(\beta, \beta\right) & =  
% 	\frac{\Delta x}{2\Delta t}\left(\beta - \beta \right)
% 	+\frac{1}{2}f(\beta) 
% 	=\frac{1}{2}f(\beta) \quad \lightning
% \end{align}
% which have confirmed that 
% $\widetilde{f}_{j\pm 1/2}^{n} \left(\beta, \beta\right) \neq f(\beta),\, \forall \beta \in \mathbb{R}$.
% Therefore, a modified version is required for these numerical flux functions,
% such that they become consistent with $f(\cdot)$, and simultaneously, 
% the summation of these two fluxes do not lead to any change in the \emph{Lax-Friedrichs} scheme.
% The consistent numerical flux functions for \emph{Lax-Friedrichs} scheme reads
% \begin{align}
% 	\widetilde{f}_{j-1/2}^{n} & =  
% 	\widetilde{f}_{j-1/2}^{n} \left(u_{j-1}^{n}, u_{j}^{n}\right)
% 	=
% 	\frac{\Delta x}{2\Delta t}\left(u_{j-1}^{n} - u_{j}^{n} \right)
% 	+\frac{1}{2}
% 	\left(f(u_{j-1}^{n}) + f(u_{j}^{n}) \right),      \label{eq:flux1corr} \\
% 	\widetilde{f}_{j+1/2}^{n} & =  
% 	\widetilde{f}_{j-1/2}^{n} \left(u_{j}^{n}, u_{j+1}^{n}\right)
% 	=
% 	\frac{\Delta x}{2\Delta t}\left(u_{j}^{n} - u_{j+1}^{n} \right)
% 	+\frac{1}{2}
% 	\left( f(u_{j+1}^{n}) + f(u_{j}^{n}) \right), \label{eq:flux2corr}
% \end{align}
% which are obtained by adding the term $1/2f(u_{j}^{n})$ to both 
% $\widetilde{f}_{j-1/2}^{n}$ and $\widetilde{f}_{j+1/2}^{n}$. 
% Note in passing that the subtraction sign between these two fluxes will 
% cancel out this extra term, as shown in \eqref{eq:consevationformgeneral}.
% Next, the consistent property is checked as follows
% \begin{align}
% 	\eqref{eq:flux1corr} \Leftrightarrow
% 	\widetilde{f}_{j-1/2}^{n} \left(\beta, \beta\right) & =  
% 	\frac{\Delta x}{2\Delta t}\left(\beta - \beta \right)
% 	+ \frac{1}{2}\left(f(\beta)+f(\beta)\right)  
% 	=f(\beta) \quad \checkmark                                \\
% 	\eqref{eq:flux2corr} \Leftrightarrow
% 	\widetilde{f}_{j+1/2}^{n} \left(\beta, \beta\right) & =  
% 	\frac{\Delta x}{2\Delta t}\left(\beta - \beta \right)
% 	+\frac{1}{2} \left(f(\beta)+f(\beta)\right)  
% 	=f(\beta) \quad \checkmark
% \end{align}
% which have confirmed that 
% $\widetilde{f}_{j\pm 1/2}^{n} \left(\beta, \beta\right) = f(\beta),\, \forall \beta \in \mathbb{R}$
% for constant flow. Hence, the numerical flux functions $\widetilde{f}_{j\pm 1/2}^{n}$
% for the \emph{Lax-Friedrichs} scheme take the following formulation 
% \begin{equation}
% 	\therefore\quad
% 	\boxed{
% 		\begin{aligned}
% 			\widetilde{f}_{j-1/2}^{n} \left(u_{j-1}^{n}, u_{j}^{n}\right)
% 			 & =
% 			\frac{\Delta x}{2\Delta t}\left(u_{j-1}^{n} - u_{j}^{n} \right)
% 			+\frac{1}{2}
% 			\left(f(u_{j-1}^{n}) + f(u_{j}^{n}) \right) \\
% 			\widetilde{f}_{j+1/2}^{n} \left(u_{j}^{n}, u_{j+1}^{n}\right)
% 			 & =
% 			\frac{\Delta x}{2\Delta t}\left(u_{j}^{n} - u_{j+1}^{n} \right)
% 			+\frac{1}{2}
% 			\left(f(u_{j}^{n}) + f(u_{j+1}^{n}) \right)
% 		\end{aligned}
% 	}
% \end{equation}
% or the generalized form used for FVM implementation reads
% \begin{equation}
% 	\therefore\quad
% 	\boxed{
% 		F\left(u_L, u_R \right) =
% 		\frac{1}{2} \left(f(u_L) + f(u_R) \right)
% 		+\frac{\Delta x}{2\Delta t}\left(u_L - u_R \right)
% 	}
% \end{equation}

% \clearpage
% %------------------------------------------------------------------------------
% \begin{example}
% 	Crucialness of conservation form and consistent numerical flux fcn.
% \end{example}
% \inputfig{floats/LFconsistency}{LFconsistency}

% % %------------------------------------------------------------------------------
% % \pagebreak
% % \section{Godunov's method}
% % \begin{example}
% % 	Summary
% % \end{example}
% % $\rightarrow$ One-sided method cannot be used for system, i.e. mixed sign of eigenvalue causes difficulty.\\
% % $\rightarrow$
% % %------------------------------------------------------------------------------
% % \section{Approximate Riemann Solvers}
% % \begin{example}
% % 	Linearized Riemann solvers - Roe Solver.
% % \end{example}

% % \pagebreak
% % \begin{example}
% % 	Local Lax-Friedrichs flux.
% % \end{example}

% %------------------------------------------------------------------------------
% \clearpage
% \section{Approximate Riemann solvers (cont.)}
% \begin{example}
% 	Roe's solver.
% \end{example}
% Nonlinear \emph{Riemann}'s problem
% \begin{align}\label{eq:nonlinearRP}
% 	U_{t} + F(U)_x = 0,
% 	\quad\text{with}\quad
% 	U(x,0) = 
% 	\begin{cases}
% 		U_{L},\quad x<0, \\
% 		U_{R},\quad x>0,
% 	\end{cases}
% \end{align}
% which is written in quasi-linear form as follows
% \begin{align}
% 	U_{t} + A(U)U_x = 0,
% 	\quad\text{with}\quad
% 	U(x,0) = 
% 	\begin{cases}
% 		U_{L},\quad x<0, \\
% 		U_{R},\quad x>0,
% 	\end{cases}
% \end{align}
% where $A(U)$ is the non-constant Jacobian matrix. 

% Recall the conservation form used for FVM reads
% \begin{align}
% 	U_{j}^{n+1} = U_{j}^{n} + \frac{\Delta t}{\Delta x}
% 	\left(
% 	\widetilde{F}_{j-1/2}^{n} - \widetilde{F}_{j+1/2}^{n}
% 	\right).
% \end{align}

% \emph{Godunov}'s solver takes the intercell numerical flux function 
% \begin{align}
% 	\widetilde{F}_{j-1/2}^{n}
% 	 & = \widetilde{F}_{j-1/2}^{n}(U_{L},U_{R})
% 	= F_{j-1/2}^{n}(U^{*}(U_{L},U_{R})),        \\
% 	\widetilde{F}_{j+1/2}^{n}
% 	 & = \widetilde{F}_{j+1/2}^{n}(U_{L},U_{R})
% 	= F_{j+1/2}^{n}(U^{*}(U_{L},U_{R})),
% \end{align}
% where $U^{*}(U_{L},U_{R})$ is the exact \emph{Riemann}'s solution 
% at the interface between cells.

% Instead of solving exactly the nonlinear \emph{Riemann}'s problem above
% for every single space-time increment in FVM,
% which turns out to be costly and not efficient in general, 
% \emph{Roe} proposed solving the following system
% \begin{align}\label{eq:Roesolver}
% 	U_{t} + \widehat{A}(U_{L},U_{R})U_x = 0,
% 	\quad\text{with}\quad
% 	U(x,0) = 
% 	\begin{cases}
% 		U_{L},\quad x<0, \\
% 		U_{R},\quad x>0,
% 	\end{cases}
% \end{align}
% where $\widehat{A}(U_{L},U_{R})$ is essentially a constant Jacobian matrix. 
% The linearized system \eqref{eq:Roesolver} easily to be solved exactly is known as
% \emph{Approximate Riemann Solver}. 
% Besides, the correlation between 
% \eqref{eq:Roesolver} and \eqref{eq:nonlinearRP} is 
% guaranteed by three conditions applied on matrix $\widehat{A}(U_{L},U_{R})$
% proposed by \emph{Roe} as follows
% \begin{enumerate}
% 	\item Hyperbolicity: real eigenvalues
% 	      $\widehat{\lambda}_{p} = \widehat{\lambda}_{p}(U_{L},U_{R})$ required.
% 	\item Consistency with the exact Jacobian matrix $\widehat{A}(U,U) = A(U)$
% 	\item Conservation across discontinuities $F(U_{R})-F(U_{L}) = \widehat{A}(U_{L},U_{R})(U_{L}-U_{R})$
% \end{enumerate}

% \clearpage
% Matrix $\widehat{A}(U_{L},U_{R})$
% $\rightarrow$
% Eigenvalues $\widehat{\lambda}_{p}(U_{L},U_{R})$
% $\rightarrow$
% Eigenvectors $\widehat{r}_{p}(U_{L},U_{R})$

% Consider 
% \begin{align}
% 	U_{R} - U_{L}
% 	= \sum_{p=1}^{m}
% 	\widehat{\alpha}_{p}(U_{L},U_{R})
% 	\widehat{\lambda}_{p}(U_{L},U_{R})
% \end{align}

% Next, interfacial values $U_{j+1/2}(x/t)$ along $t-$axis, i.e. $x/t = 0$, 
% take the following equality 
% \begin{align}
% 	U_{j+1/2}(0) = U_{L} + 
% 	\sum_{\widehat{\lambda}_{p} \leq 0}
% 	\widehat{\alpha}_{p}(U_{L},U_{R})
% 	\widehat{\lambda}_{p}(U_{L},U_{R}) \\
% 	U_{j+1/2}(0) = U_{R} - 
% 	\sum_{\widehat{\lambda}_{p} \leq 0}
% 	\widehat{\alpha}_{p}(U_{L},U_{R})
% 	\widehat{\lambda}_{p}(U_{L},U_{R})
% \end{align}

% Besides, according to Definition II.$21$ from lecture note,
% the corresponding numerical flux function reads
% % \begin{align}
% % 	\widetilde{F}U_{L},U_{R}} & = F_{L}
% % \end{align}

% \clearpage
% %------------------------------------------------------------------------------
% \begin{example}
% 	Local Lax-Friedrichs (LLF) flux function.
% \end{example}

% Recall that the \emph{Roe}'s solver takes the following forms of numerical flux functions
% \begin{align*}
% 	\widetilde{F}_{j+1/2}\left(U_{L},U_{R}\right)
% 	 & = F(U_L) + \widehat{A}\left(U_{L},U_{R}\right)^{-}\left(U_{R} - U_{L}\right), \\
% 	\widetilde{F}_{j+1/2}\left(U_{L},U_{R}\right)
% 	 & = F(U_R) - \widehat{A}\left(U_{L},U_{R}\right)^{+}\left(U_{R} - U_{L}\right).
% \end{align*}
% By summing and taking the average we obtain another form of 
% numerical flux function in \emph{Roe}'s solver as follows
% \begin{align}\label{eq:Roefluxnewform}
% 	\therefore\quad
% 	\boxed{
% 		\widetilde{F}^{(R)}_{j+1/2}(U_{L},U_{R}) = 
% 		\frac{1}{2}\left(F(u_{L})+F(u_{R})\right) 
% 		- \frac{1}{2}\left| \widehat{A} \right|\left(U_{R}-U_{L}\right)
% 	}
% \end{align}
% Note in passing that we have used the following equality in the expression \eqref{eq:Roefluxnewform}
% \begin{align}
% 	\widehat{A}^{-}(U_{L},U_{R})-\widehat{A}^{+}(U_{L},U_{R})
% 	= -\left|\widehat{A}\right|(U_{L},U_{R}).
% \end{align}
% Besides, the local Lax-Friedrichs flux function take the following form 
% \begin{align}\label{eq:lLFfluxform}
% 	\therefore\quad
% 	\boxed{
% 		\widetilde{F}^{(LLF)}_{j+1/2}(U_{L},U_{R}) = 
% 		\frac{1}{2}\left(F(u_{L})+F(u_{R})\right) 
% 		- \frac{1}{2} \lambda^{\text{max}} \left(U_{R}-U_{L}\right)
% 	}
% \end{align}
% Note in passing that we have used an approximation for 
% $\left| \widehat{A} \right|$ in \eqref{eq:Roefluxnewform}
% and applied it for $\lambda^{\text{max}}$ in \eqref{eq:lLFfluxform}, as follows

% \clearpage
% \begin{example}
% 	Examine Local Lax-Friedrichs (LLF) versus \emph{Roe}'s solver.
% \end{example}
% \inputfig{floats/LLFvsRoe}{LLFvsRoe}

% \clearpage
% %------------------------------------------------------------------------------
% % \begin{example}
% % 	Determin a Roe matrix for the one-dimensional shallow water equations.
% % \end{example}
% %------------------------------------------------------------------------------
% % \begin{example}
% % 	Determin a Roe matrix for the one-dimensional isothermal equations of gas dynamics.
% % \end{example}
% \begin{example}
% 	Derivation of Harten-Lax-van Leer (HLL).
% \end{example}

% \begin{enumerate}
% 	\item Recall that the two left-and right-sided formulae
% 	      for the numerical flux functions based on \emph{Roe}'s solver 
% 	      are written as follows
% 	      \begin{align}
% 		      \widetilde{F}_{j+1/2}\left(U_{L},U_{R}\right)
% 		       & = F(U_L) + \widehat{A}\left(U_{L},U_{R}\right)^{-}\left(U_{R} - U_{L}\right) \\
% 		      \widetilde{F}_{j+1/2}\left(U_{L},U_{R}\right)
% 		       & = F(U_R) - \widehat{A}\left(U_{L},U_{R}\right)^{+}\left(U_{R} - U_{L}\right)
% 	      \end{align}
% 	      By first taking summation and then averaging the above two expressions we arrive at
% 	      \begin{align}
% 		      \widetilde{F}_{j+1/2}\left(U_{L},U_{R}\right)
% 		      =\frac{1}{2}\left(F(U_{L})+F(U_{R})\right)
% 		      -\frac{1}{2}\left|\widehat{A}\right|\left(U_{R}-U_{L}\right).
% 	      \end{align}
% 	      Note in passing that
% 	      $\widehat{A}^{-}(U_{L},U_{R})-\widehat{A}^{+}(U_{L},U_{R}) = -\left|\widehat{A}\right|(U_{L},U_{R})$.	      

% 	\item The integral formulation applied to the domain
% 	      $[x_{L},x_{R}]\times[t_{n},t_{n+1}]$ and together
% 	      with the condition $x_{L}<0<x_{R}$ reads
% 	      \begin{align}\label{eq:integralform}
% 		      \int_{x_{L}}^{x_{R}}U(x,t_{n+1})\,dx
% 		       & =\int_{x_{L}}^{x_{R}}U(x,t_{n})\,dx \notag \\
% 		       & \quad 
% 		      +\int_{t_{n}}^{t_{n+1}}F(U(x_{L},t))\,dt  
% 		      -\int_{t_{n}}^{t_{n+1}}F(U(x_{R},t))\,dt.
% 	      \end{align}
% 	      \inputfig{floats/roeflux}{roeflux}
% 	      Since the solution is piece-wise constant, these integrals are computed as follows
% 	      \begin{align*}
% 		      \int_{t_{n}}^{t_{n+1}}F(U(x_{L},t))\,dt & = \cdots,                   \\
% 		      \int_{t_{n}}^{t_{n+1}}F(U(x_{R},t))\,dt & = \cdots,                   \\
% 		      \int_{x_{L}}^{x_{R}}U(x,t_{n})\,dx      & = \cdots + \cdots,          \\
% 		      \int_{x_{L}}^{x_{R}}U(x,t_{n+1})dx      & = \cdots + \cdots + \cdots.
% 	      \end{align*}
% 	      Next, by substituting these four integrals back into \eqref{eq:integralform}
% 	      we obtain the following relation
% 	      \begin{align}\label{eq:ustarfinal}
% 		      U^{*}
% 		       & =\frac{\left|\lambda_{L}\right|U_{L}+\lambda_{R}U_{R}+F(U_{L})-F(U_{R})}{\left|\lambda_{L}\right|+\lambda_{R}}.
% 	      \end{align}

% 	\item The decomposition with the two waves are given by
% 	      \begin{align}
% 		      U_{R}-U_{L}=\sum_{p=1}^{N}\alpha_{p}\widehat{r}_{p}=
% 		       & \left(U_{R}-U^{\star}\right)+\left(U^{\star}-U_{L}\right),
% 	      \end{align}
% 	      and hence we arrive at the following expression
% 	      \begin{align}\label{eq:givendecomposition}
% 		      \sum_{p=1}^{N}\alpha_{p}\left|\widehat{\lambda}_{p}\right|\widehat{r}_{p}
% 		       & =\lambda_{R}\left(U_{R}-U^{\star}\right)+\left|\lambda_{L}\right|\left(U^{\star}-U_{L}\right),
% 	      \end{align}
% 	      which leads to the following expression
% 	      \begin{align}\label{eq:FtildeHLL}
% 		      \widetilde{F}_{j+1/2}\left(U_{L},U_{R}\right) & 
% 		      =\frac{1}{2}\left(F(U_{L})+F(U_{R})\right)-\frac{\lambda_{R}}{2}\left(U_{R}-U^{\star}\right)-\frac{\left|\lambda_{L}\right|}{2}\left(U^{\star}-U_{L}\right).
% 	      \end{align}
% 	      Then, by using \eqref{eq:givendecomposition}, inserting \eqref{eq:ustarfinal}
% 	      into  \eqref{eq:FtildeHLL}, and
% 	      applying some more algebraic manipulations we arrive at
% 	      the HLL numerical flux function, where $\lambda_{L}<0<\lambda_{R}$, as follows
% 	      \begin{align}
% 		      \widetilde{F}_{j+1/2}\left(U_{L},U_{R}\right) & 
% 		      =\frac{\left|\lambda_{L}\right|F(U_{R})+\lambda_{R}F(U_{L})}
% 		      {\left|\lambda_{L}\right|+\lambda_{R}}+\frac{\lambda_{R}\left|\lambda_{L}\right|}
% 		      {\left|\lambda_{L}\right|+\lambda_{R}}\left(U_{L}-U_{R}\right).
% 	      \end{align}

% 	\item When all waves travel to the right, we obtain
% 	      \begin{align}
% 		      \lambda_{L,R}>0 
% 		      \rightarrow U^{*} = U_{L}
% 		      \rightarrow F(U^{*})=F(U_{L}).
% 	      \end{align}
% 	      When all waves travel to the left, we obtain
% 	      \begin{align}
% 		      \lambda_{L,R}<0 
% 		      \rightarrow U^{*}=U_{R}
% 		      \rightarrow F(U^{*})=F(U_{R}).
% 	      \end{align}
% 	      Finally, we obtain a complete HLL numerical flux function as follows 
% 	      \begin{equation}\label{eq:HLLsolver}
% 		      \therefore\quad
% 		      \boxed{
% 		      \widetilde{F}_{j+1/2}^{\text{(HLL)}}\left(U_{L},U_{R}\right) =
% 		      \begin{cases}
% 			      F(U_{R}), \quad (\lambda_{L,R}<0), \\
% 			      \displaystyle
% 			      \frac{\left|\lambda_{L}\right|F(U_{R})+\lambda_{R}F(U_{L})}
% 			      {\left|\lambda_{L}\right|+\lambda_{R}}
% 			      +\frac{\lambda_{R}\left|\lambda_{L}\right|}
% 			      {\left|\lambda_{L}\right|+\lambda_{R}}
% 			      \left(U_{L}-U_{R}\right),          \\
% 			      \qquad \qquad \qquad \qquad \qquad \qquad \qquad 
% 			      (\lambda_{L}<0<\lambda_{R}),       \\
% 			      F(U_{L}), \quad (\lambda_{L,R}>0).
% 		      \end{cases}
% 		      }
% 	      \end{equation}
% \end{enumerate}

% \clearpage
% \begin{example}
% 	Examine \emph{Roe}'s solver versus Harten-Lax-van Leer (HLL).
% \end{example}
% \inputfig{floats/RoevsHLL}{LLFvsHLL}

% \section{Conservation form - Finite Volume Method (cont.)}
% \begin{example}
% 	Derivation of conservation form.
% \end{example}
% \inputfig{floats/massx1x2_u}{massx1x2_u}
% Starting point is the \textbf{integral form} of conservation law. Recall
% the integral formulation of conservation law
% for a scalar conserved quantity $u(x,t)$ over domain $[x_1,x_2]$, herein, reads
% \begin{equation}\label{eq:integralform1}
% 	\boxed{
% 		\frac{d}{dt}\int_{x_{1}}^{x_{2}} u(x,t)\,dx = f(u(x_{1},t)) - f(u(x_{2},t)).
% 	}
% \end{equation}
% Integration both side of \eqref{eq:integralform1} over temporal interval $[t_1,t_2]$ yields
% \begin{align}
% 	\left.\left( \int_{x_{1}}^{x_{2}} u(x,t)\,dx\right) \right|_{t=t_1}^{t=t_2}
% 	= \int_{t_1}^{t_2} \left( f(u(x_{1},t)) - f(u(x_{2},t)) \right)dt,
% \end{align}
% which leads to \textbf{\textit{another} integral form} of the conservation law, as follows
% \begin{equation}\label{eq:integralform2}
% 	\boxed{
% 		\int_{x_{1}}^{x_{2}} u(x,t_2)\,dx = 
% 		\int_{x_{1}}^{x_{2}} u(x,t_1)\,dx
% 		+ \int_{t_1}^{t_2} f(u(x_{1},t)) \,dt
% 		- \int_{t_1}^{t_2} f(u(x_{2},t)) \,dt.
% 	}
% \end{equation}
% The integral form of conservation law shown in
% \eqref{eq:integralform2} will be now used 
% to derive the conservation form for finite volume method (FVM).
% Then, the derivation is performed by taking into consideration of
% the spatial interval $[x_{j-1/2},x_{j+1/2}]$
% and temporal interval $[t_n,t_{n+1}]$ instead of $[x_1,x_2]$ and $[t_1,t_2]$, respectively,
% as follows
% \begin{align}
% 	[x_1,x_2] & \rightarrow [x_{j-1/2},x_{j+1/2}], \\
% 	[t_1,t_2] & \rightarrow [t_n,t_{n+1}].
% \end{align}
% Hence, the integral form in \eqref{eq:integralform2} becomes
% \begin{align}\label{eq:integralform3}
% 	\int_{x_{j-1/2}}^{x_{j+1/2}} u(x,t_{n+1})\,dx & = 
% 	\int_{x_{j-1/2}}^{x_{j+1/2}} u(x,t_{n})\,dx \notag                                                     \\
% 	                                              & \qquad + \int_{t_{n}}^{t_{n+1}} f(u(x_{j-1/2},t)) \,dt
% 	- \int_{t_{n}}^{t_{n+1}} f(u(x_{j+1/2},t)) \,dt.
% \end{align}
% \inputfig{floats/conservationform}{conservationform}

% Next, consideration of the cell average formulation, which,
% by definition, reads
% \begin{align}
% 	\bar{u}_{j}^{n}	:= 
% 	\frac{1}{\Delta x} \int_{x_{j-1/2}}^{x_{j+1/2}} u(x,t_{n})\,dx,
% \end{align}
% and multiplication of both sides of \eqref{eq:integralform3} by $1/\Delta x$
% yield the following relation
% \begin{align}\label{eq:integralform4}
% 	\bar{u}_{j}^{n+1} = \bar{u}_{j}^{n}
% 	+ \frac{1}{\Delta x}
% 	\left(
% 	\int_{t_{n}}^{t_{n+1}} f(u(x_{j-1/2},t)) \,dt
% 	- \int_{t_{n}}^{t_{n+1}} f(u(x_{j+1/2},t)) \,dt
% 	\right).
% \end{align}
% Moreover, by defining the \textbf{numerical flux functions} as follows
% \begin{align}
% 	\tilde{f}_{j-1/2}^{n} := \frac{1}{\Delta t}\int_{t_{n}}^{t_{n+1}} f(u(x_{j-1/2},t)) \,dt, \\
% 	\tilde{f}_{j+1/2}^{n} := \frac{1}{\Delta t}\int_{t_{n}}^{t_{n+1}} f(u(x_{j+1/2},t)) \,dt,
% \end{align}
% the expression \eqref{eq:integralform4} finally becomes the \textbf{conservation form}
% \begin{equation}
% 	\therefore\quad
% 	\boxed{
% 		\bar{u}_{j}^{n+1} = \bar{u}_{j}^{n}
% 		+ \frac{\Delta t}{\Delta x}
% 		\left(
% 		\tilde{f}_{j-1/2}^{n}
% 		-\tilde{f}_{j+1/2}^{n}
% 		\right).
% 	}
% \end{equation}
% Likewise, a generalization to nonlinear \emph{systems} takes the following form
% \begin{equation}
% 	\therefore\quad
% 	\boxed{
% 		\bar{U}_{j}^{n+1}
% 		= \bar{U}_{j}^{n}
% 		+ \frac{\Delta t}{\Delta x}
% 		\left(
% 		\tilde{F}_{j-1/2}^{n}
% 		-\tilde{F}_{j+1/2}^{n}
% 		\right).
% 	}
% \end{equation}

% %------------------------------------------------------------------------------
% % \pagebreak
% % \begin{example}
% % 	Lipschitz continuity.
% % \end{example}
% % \begin{figure}[ht]
% % 	\centering
% % 	%{10} means speed
% % 	\animategraphics[scale=0.3,every=1,autoplay,loop]{10}{animations/sqn_lipschitz/lipschitz-}{1}{51}
% % 	% \movie[height = 0.6\textwidth,width = 0.8\textwidth]{}{movies/lipschitz/lipschitz.mpg}
% % 	\caption{Lipschitz continuity (\emph{Tip: Open the pdf file in Okular to see animation}).}
% % \end{figure}

% \pagebreak
% \begin{example}
% 	Derivation of the conservative form of upwind scheme.
% 	Examine its numerical flux function: consistent?
% \end{example}
% Recall the conservation form
% \begin{equation}
% 	\bar{u}_{j}^{n+1} = \bar{u}_{j}^{n}
% 	+ \frac{\Delta t}{\Delta x}
% 	\left(
% 	\tilde{f}_{j-1/2}^{n}
% 	-\tilde{f}_{j+1/2}^{n}
% 	\right).
% \end{equation}

% %------------------------------------------------------------------------------
% \pagebreak
% \begin{example}
% 	Derivation of the conservative form of Lax-Friedrichs scheme. 
% 	Examine its numerical flux function: consistent?
% \end{example}

% Approach:
% Recall the conservation form
% \begin{equation}
% 	\bar{u}_{j}^{n+1} = \bar{u}_{j}^{n}
% 	+ \frac{\Delta t}{\Delta x}
% 	\left(
% 	\tilde{f}_{j-1/2}^{n}
% 	-\tilde{f}_{j+1/2}^{n}
% 	\right).
% \end{equation}
% The generalization of the \emph{Lax-Friedrichs} scheme to nonlinear systems 
% takes the following form
% \begin{align}
% 	u_{j}^{n+1} = 
% 	\frac{1}{2} \left(u_{j-1}^{n} + u_{j+1}^{n}\right)
% 	+ \frac{\Delta t}{2\Delta x}
% 	\left(f(u_{j+1}^n - f(u_{j-1}^n)\right)
% \end{align}

% %------------------------------------------------------------------------------
% \pagebreak
% \begin{example}
% 	Derivation of the conservative form of Lax-Wendroff scheme. 
% 	Examine its numerical flux function: consistent?
% 	Consider the nonlinear scalar conservation law given as follows
% 	\begin{align*}
% 		U_{t} + F(U)_x = 0.
% 	\end{align*}
% 	Write the \emph{Lax-Wendroff} scheme for the above equation in \emph{conservation form}. 
% 	Show that the numerical flux function is consistent.
% \end{example}

% % Recall the conservation form for vectorial unknown quantity:
% % \begin{equation}
% % 	\bar{U}_{j}^{n+1} = \bar{U}_{j}^{n}
% % 	+ \frac{\Delta t}{\Delta x}
% % 	\left(
% % 	\tilde{F}_{j-1/2}^{n}
% % 	-\tilde{F}_{j+1/2}^{n}
% % 	\right).
% % \end{equation}

% Approach:
% \begin{enumerate}
% 	\item Conservation form: A numerical scheme applied for the conservation law
% 	      $U_t + F(U)_x = 0$ is said to be conservative 
% 	      if it can be written in the conservation form, as follows
% 	      \begin{align}
% 		      \bar{U}_{j}^{n+1} = \bar{U}_{j}^{n}
% 		      + \frac{\Delta t}{\Delta x}
% 		      \left(
% 		      \tilde{F}_{j-1/2}^{n}
% 		      -\tilde{F}_{j+1/2}^{n}
% 		      \right),
% 	      \end{align}
% 	      where $\tilde{F}_{j-1/2}^n$ and $\tilde{F}_{j+1/2}^n$
% 	      are called interfacial fluxes. 
% 	      %   These fluxes are given by 
% 	      %   \begin{align}
% 	      %       \tilde{f}_{j+1/2}^n & = \tilde{f}\left( u_{j-m+1}^n, \hdots, u_{j+m}^n \right), \\
% 	      %       \tilde{f}_{j-1/2}^n & = \tilde{f}\left( u_{j-m}^n, \hdots, u_{j+m-1}^n \right).
% 	      %   \end{align}
% 	      Next, by taking into consideration for the case of \emph{Lax-Wendroff} scheme 
% 	      for the nonlinear conservation law equation,
% 	      the conservation form is derived as follows
% 	      \begin{align}
% 		      U_j^{n+1} & = U_j^n 
% 		      - \frac{\Delta t}{2\Delta x} \left( F(U_{j+1}^n) - F(U_{j-1}^n) \right) \notag \\
% 		                & \quad + \frac{(\Delta t)^2}{2\left(\Delta x\right)^2}
% 		      \left(
% 		      A^2 (U_j^n, U_{j+1}^n) (U_{j+1}^n - U_j^n ) 
% 		      - A^2 (U_{j-1}^n, U_{j}^n) (U_{j}^n - U_{j-1}^n ) 
% 		      \right),
% 	      \end{align}
% 	      which is rewritten by grouping relative terms together, as follows
% 	      \begin{align}
% 		      U_j^{n+1} 
% 		       & = U_j^n - 
% 		      \frac{\Delta t}{\Delta x}
% 		      \left(  
% 		      \frac{1}{2} \left( F(U_j^n) + F(U_{j+1}^n) \right)  
% 		      - 
% 		      \frac{\Delta t}{2\Delta x}  A^2 (U_j^n, U_{j+1}^n) (U_{j+1}^n - U_j^n ) 
% 		      \right)       \notag \\ 
% 		       & 
% 		      + 
% 		      \frac{\Delta t}{\Delta x}
% 		      \left(  
% 		      \frac{1}{2} \left( F(U_{j-1}^n) + F(U_{j}^n) \right)
% 		      - \frac{\Delta t}{2\Delta x} A^2 (U_{j-1}^n, U_{j}^n) (U_{j}^n - U_{j-1}^n ) 
% 		      \right).
% 	      \end{align}
% 	      Then, this form can be written in terms of conservation form, as follows
% 	      \begin{align}
% 		      U_j^{n+1} 
% 		      = U_j^n 
% 		      - 
% 		      \frac{\Delta t}{\Delta x}
% 		      \left(
% 		      \tilde{F}_{j+1/2}^n - \tilde{F}_{j-1/2}^n
% 		      \right),
% 	      \end{align}
% 	      where the numerical flux functions are recognized as
% 	      \begin{align*}
% 		      \begin{split}
% 			      \tilde{F}_{j+1/2}^n(U_j^n, U_{j+1}^n)  
% 			      %   &= \tilde{F}(U_j^n, U_{j+1}^n)
% 			      &=  \frac{1}{2}
% 			      \left( F(U_j^n) + F(U_{j+1}^n) \right)
% 			      - \frac{\Delta t}{2\Delta x}
% 			      A^2 (U_j^n, U_{j+1}^n) (U_{j+1}^n - U_j^n ), \\
% 			      \tilde{F}_{j-1/2}^n(U_{j-1}^n, U_{j}^n)  
% 			      %   & = \tilde{F}(u_{j-1}^n, u_{j}^n) 
% 			      & =\frac{1}{2}
% 			      \left( F(U_{j-1}^n) + F(U_{j}^n) \right)
% 			      - \frac{\Delta t}{2\Delta x}
% 			      A^2 (U_{j-1}^n, U_{j}^n) (U_{j}^n - U_{j-1}^n ).
% 		      \end{split}
% 	      \end{align*}
% 	\item Consistent numerical flux function:\\
% 	      The conservative numerical flux function is then given by
% 	      \begin{align}
% 		      \tilde{F}(V, W)  =  \frac{1}{2}
% 		      \left( F(V) + F(W) \right)
% 		      - \frac{\Delta t}{2 \Delta x}
% 		      A^2 (V,W)(V-W).
% 	      \end{align}
% 	      When $V=W$, we obtain the following relation
% 	      \begin{align}
% 		      \tilde{F}(V,W) = \tilde{F}(W,W) = F(W),
% 	      \end{align}
% 	      which confirms that the numerical flux function $\tilde{F}$ is consistent 
% 	      with the continuous flux function $F$.  
% \end{enumerate}

% %------------------------------------------------------------------------------
% \pagebreak
% \begin{example}
% 	% Derivation of the conservative form of the two-step Lax-Wendroff scheme. 
% 	% Examine its numerical flux function: conservative? consistent?
% 	Consider the two-step \emph{Lax-Wendroff} scheme as follows
% 	% \begin{align}
% 	% 	u^{n+1/2}_{j+1/2} & 
% 	% 	= \frac{1}{2} \left(u_{j+1}^n + u_j^n \right) -
% 	% 	\frac{\Delta t}{2\Delta x}
% 	% 	\left( f\left(u_{j+1}^n\right) -f\left(u_j^n\right) \right), \\
% 	% 	u_j^{n+1}         & 
% 	% 	= u_j^n -
% 	% 	\frac{\Delta t}{\Delta x}
% 	% 	\left(
% 	% 	f\left(u_{j+1/2}^{n+1/2}\right)
% 	% 	-f\left(u_{j-1/2}^{n+1/2}\right)
% 	% 	\right).
% 	% \end{align}

% 	\begin{enumerate}
% 		\item[\text{Step-1:}] First, update the half-advanced-in-time point $U^{n+1/2}_{j+1/2}$
% 		      by computing
% 		      \begin{align}
% 			      U^{n+1/2}_{j+1/2}
% 			      = \frac{1}{2} \left(U_{j+1}^n + U_j^n \right) -
% 			      \frac{\Delta t}{2\Delta x}
% 			      \left( F\left(U_{j+1}^n\right) -F\left(U_j^n\right) \right),
% 		      \end{align}
% 		\item[\text{Step-2:}] Then, update the fully-advanced-in-time point $U^{n+1}_{j}$
% 		      by computing
% 		      \begin{align}
% 			      U_j^{n+1}         
% 			      = U_j^n -
% 			      \frac{\Delta t}{\Delta x}
% 			      \left(
% 			      F\left(U_{j+1/2}^{n+1/2}\right)
% 			      -F\left(U_{j-1/2}^{n+1/2}\right)
% 			      \right).
% 		      \end{align}
% 	\end{enumerate}
% 	Examine the two-step Lax-Wendroff scheme whether 
% 	it is conservative, i.e. the scheme can be written 
% 	in the \emph{conservation form}. 
% 	In case it is conservative, determining the numerical flux function.
% 	% Examine also if its numerical flux function is consistent.
% \end{example}

% Idea:
% \begin{equation}
% 	\boxed{
% 		\begin{array}{c}
% 			\text{One way to avoid using Jacobian matrix arising in flux function,} \\
% 			\text{when dealing with linear hyperbolic system $U_{t} + AU_{x} = 0$.}
% 		\end{array}
% 	}
% \end{equation}

% Recall the conservation form for vectorial unknown quantity:
% \begin{equation}
% 	\bar{U}_{j}^{n+1} = \bar{U}_{j}^{n}
% 	+ \frac{\Delta t}{\Delta x}
% 	\left(
% 	\tilde{F}_{j-1/2}^{n}
% 	-\tilde{F}_{j+1/2}^{n}
% 	\right).
% \end{equation}

% Approach:
% The two-step \emph{Lax-Wendroff} scheme for the conservation law
% $u_t + f(u)_x=0$ is given by
% \begin{align*}
% 	u^{n+1/2}_{j+1/2} & = \frac{1}{2}
% 	\left( u_{j+1}^n +u_j^n \right)
% 	- \frac{\Delta t}{2\Delta x} \left(
% 	f\left(u_{j+1}^n\right) - f\left(u_j^n\right) \right), \\
% 	u_j^{n+1}         & = 
% 	u_j^n - \frac{\Delta t}{\Delta x} \left( f\left(u_{j+1/2}^{n+1/2}\right)
% 	- f\left(u_{j-1/2}^{n+1/2}\right) \right).
% \end{align*}
% This scheme can also be written in the conservation form
% \begin{align*}
% 	u_j^{n+1} 
% 	= u_j^n - \frac{\Delta t}{\Delta x}
% 	\left( f_{i + 1/2}^n - f_{i - 1/2}^n \right),
% \end{align*}
% where
% \begin{align*}
% 	\begin{split}
% 		f_{j+1/2}^n 
% 		& = f 
% 		\left( 
% 		\frac{1}{2} \left ( u_{j+1}^n + u_j^n \right)
% 		- \frac{\Delta t}{2\Delta x}
% 		\left( f\left(u_{j+1}^n\right) - f(u_j^n) \right ) \right), \\
% 		f_{j-1/2}^n 
% 		& = f 
% 		\left( 
% 		\frac{1}{2} \left ( u_{j}^n + u_{j-1}^n \right)
% 		- \frac{\Delta t}{2\Delta x} \left( f(u_{j}^n) - f(u_{j-1}^n) \right) 
% 		\right).
% 	\end{split}
% \end{align*}
% The conservative flux function is given by 
% \begin{align*}
% 	F(u,v) = f
% 	\left( 
% 	\frac{1}{2}(u+v)
% 	- \frac{\lambda}{2} \left( f(v) - f(u) \right) 
% 	\right).
% \end{align*}
% When $v=u$, we obtain the relation
% \begin{align*}
% 	F(u,u)=f(u),
% \end{align*}
% which shows that the numerical flux function is 
% consistent with the continuous flux function.

% %------------------------------------------------------------------------------
% \pagebreak
% \begin{example}
% 	% Derivation of the conservative form of MacCormack scheme. 
% 	% Examine its numerical flux function: conservative? consistent?
% 	Consider the \emph{MacCormack} scheme $(*)$ given by
% 	\begin{enumerate}
% 		\item[\text{Step-1:}] First, update the half-advanced-in-time point $u^{n+1/2}_{j}$ by computing
% 		      \begin{align}
% 			      u^{n+1/2}_{j}  = 
% 			      u_j^n - \frac{\Delta t}{\Delta x}
% 			      \left(
% 			      f\left(u_{j+1}^n\right)
% 			      -f\left(u_j^n\right)
% 			      \right),
% 		      \end{align}
% 		\item[\text{Step-2:}] Then, update the fully-advanced-in-time point $u^{n+1}_{j}$ by computing
% 		      \begin{align}
% 			      u_j^{n+1}      = 
% 			      \frac{1}{2} \left(u_j^n + u_j^{n+1/2} \right)
% 			      -\frac{\Delta t}{2\Delta x}
% 			      \left( f\left(u_j^{n+1/2}\right)
% 			      -f\left(u_{j-1}^{n+1/2}\right)
% 			      \right).
% 		      \end{align}
% 	\end{enumerate}
% 	% \begin{align*}
% 	% 	u^{n+1/2}_{j} & = 
% 	% 	u_j^n -\lambda \left(
% 	% 	f\left(u_{j+1}^n\right)
% 	% 	-f\left(u_j^n\right) \right) \\
% 	% 	u_j^{n+1}     & = 
% 	% 	\frac{1}{2} \left(u_j^n + u_j^{n+1/2} \right)
% 	% 	-\frac{\lambda}{2}
% 	% 	\left( f\left(u_j^{n+1/2}\right)
% 	% 	-f\left(u_{j-1}^{n+1/2}\right)
% 	% 	\right)
% 	% \end{align*}
% 	% with $\lambda :=\Delta t/\Delta x$.
% 	Examine the MacCormack scheme whether it is conservative, i.e.
% 	the scheme can be written in the \emph{conservation form}. 
% 	Examine also if its numerical flux function is consistent.
% \end{example}

% $(*)$ R.W.MacCormack [1969]:
% \href{https://arc.aiaa.org/doi/10.2514/2.6901}
% {The effects of viscosity in hypervelocity impact cratering.}\\

% Idea:
% \begin{equation}
% 	\boxed{
% 		\begin{array}{c}
% 			\text{Another way, beside the two-step Lax-Wendroff scheme,}              \\
% 			\text{to avoid using Jacobian matrix arising in computing flux function,} \\
% 			\text{when dealing with linear hyperbolic system $U_{t} + AU_{x} = 0$.}
% 		\end{array}
% 	}
% \end{equation}

% Recall the conservation form for vectorial unknown quantity:
% \begin{equation}
% 	\bar{U}_{j}^{n+1} = \bar{U}_{j}^{n}
% 	+ \frac{\Delta t}{\Delta x}
% 	\left(
% 	\tilde{F}_{j-1/2}^{n}
% 	-\tilde{F}_{j+1/2}^{n}
% 	\right).
% \end{equation}

% Approach:

% In case of \emph{MacCormack} scheme we arrive at the conservative flux function as follows
% \begin{align}
% 	F(u,v) = \frac{1}{2}
% 	\left( 
% 	f(v) + f\left ( u - \frac{\Delta t}{\Delta x} \left( f(v) - f(u) \right ) \right )
% 	\right).
% \end{align}
% The above flux function is consistent since
% \begin{align}
% 	F(u,v) = \frac{1}{2}
% 	\left( 
% 	f(v) + f\left ( u - \frac{\Delta t}{\Delta x} \left( f(v) - f(u) \right ) \right)
% 	\right)
% 	= 
% \end{align}

% %------------------------------------------------------------------------------
% \pagebreak
% \section{Discontinuous solution}

% %------------------------------------------------------------------------------
% \pagebreak
% \section{Godunov linear systems}

% %------------------------------------------------------------------------------
% \pagebreak
% \section{Theory of high resolution}

% \section{A remark about the derivation from coupled to decoupled form of linear hyperbolic systems}
% \begin{enumerate}
% 	\item Case 1: $W:=R^{-1}U$ as the scheme shown in exercise
% 	      \begin{equation*}
% 		      \boxed{
% 			      \begin{aligned}
% 				      U_{t} + AU_{x} = 0                     \\
% 				      U_{t} + R\Lambda R^{-1} U_{x} = 0      \\
% 				      R^{-1}U_{t} + \Lambda R^{-1} U_{x} = 0 \\
% 				      W_{t} + \Lambda W_{x} = 0                    
% 			      \end{aligned}
% 		      }
% 	      \end{equation*}
% 	      where the matrix $A$ is diagonalizable with a transformation matrix $R \in \mathbb{R}^{N\times N}$ in the form 
% 	      $$A=R\Lambda R^{-1}.$$

% 	\item Case 2: $W:=TU$ as the scheme shown in lecture note
% 	      \begin{equation*}
% 		      \boxed{
% 			      \begin{aligned}
% 				      U_{t} + AU_{x} = 0                \\
% 				      U_{t} + T^{-1}\Lambda T U_{x} = 0 \\
% 				      TU_{t} + \Lambda TU_{x} = 0       \\
% 				      W_{t} + \Lambda W_{x} = 0                      
% 			      \end{aligned}
% 		      }
% 	      \end{equation*}
% 	      where the matrix $A$ is diagonalizable with a transformation matrix $T \in \mathbb{R}^{N\times N}$ in the form 
% 	      $$A= T^{-1} \Lambda T.$$
% \end{enumerate}
% $\rightarrow$ Note in passing that both schemes result in the same solution.\\ 
% $\rightarrow$ We have just to be consistent with which scheme to follow.

% \pagebreak
% %------------------------------------------------------------------------------
% \section{Correlation between Domain of dependence (DoD) and Courant-Friedrichs-Lewy (CFL) condition}
% \begin{example}\label{eq:example1}
% 	Examine the numerical domain of dependence 
% 	and the corresponding CFL condition of the One-sided method,
% 	where the stencils point to the left.
% \end{example}
% \inputfig{floats/DomainOfDependence}{DomainOfDependence}
% As it can be seen from \rFig{DomainOfDependence}, 
% the numerical value computed at point $A$ depends essentially on 
% computed initial conditions laying between point $D$ and $E$.
% \begin{enumerate}
% 	\item Perspective of indical subscription:\\
% 	      Line $(l_1)$ passsing point $A(j,n)$ and $B(j-1,n-1)$
% 	      has the following form
% 	      \begin{align}
% 		      (l_1):\quad 
% 		      \tau & = \tau_A + \frac{\tau_B-\tau_A}{\xi_B-\xi_A}\left(\xi-\xi_A\right) \notag \\
% 		      \Leftrightarrow
% 		      \tau & = n + \frac{(n-1)-n}{(j-1)-j} \left(\xi-j\right) \notag                   \\
% 		      \Leftrightarrow
% 		      \tau & = n + \frac{-1}{-1} \left(\xi-j\right),
% 	      \end{align}
% 	      where $\tau$ is the indical variable corresponding to $t$,
% 	      and $x$ the indical variable to $x$.
% 	      Hence, line $(l_1)$ passing line $x$ with index $\tau=0$ at point $D$ leads 
% 	      to the following relation
% 	      \begin{align}\label{eq:idlinel1onesided}
% 		      \xi = j - n
% 		      \Leftrightarrow 
% 		      x_{\xi} = x_{j-n}
% 		      \Leftrightarrow 
% 		      x_{\xi} = x_{j}-n\Delta x
% 		      \Leftrightarrow 
% 		      x_{\xi} - x_j =-n\Delta x.
% 	      \end{align}
% 	      Likewise, line $(l_2)$ passing line $x$ with index $\tau=0$ at point $E$
% 	      leads to the following relation
% 	      \begin{align}\label{eq:idlinel2onesided}
% 		      x_{\xi} - x_{j} = 0.
% 	      \end{align}
% 	      Therefore, by combining \eqref{eq:idlinel1onesided}
% 	      and \eqref{eq:idlinel2onesided} we arrive at the numerical domain of dependence
% 	      for the One-sided method in terms of indical perspective
% 	      \begin{align}\label{eq:Ddod}
% 		      \mathcal{D}_{\Delta t}\left(x_j,t_n\right)
% 		      =  \left\{ x_{\xi}\, \Big| -n\Delta x \leq x_{\xi} - x_j \leq 0 \right\}.
% 	      \end{align}
% 	      Next, by using the $CFL$ number $\nu := a\Delta t/\Delta x$ we obtain the 
% 	      following equality
% 	      \begin{align}\label{eq:equalityD}
% 		      -n\Delta x 
% 		      = -n\Delta t\frac{a\Delta x}{a \Delta t}
% 		      \stackrel{(CFL)}{=}
% 		      -n\Delta t\frac{a}{\nu}
% 		      = -\frac{at_n}{\nu}.
% 	      \end{align}
% 	      Then, by substituting \eqref{eq:equalityD} into \eqref{eq:Ddod}
% 	      with limit consideration we obtain the entire set of the numerical
% 	      domain of dependence, as follows 
% 	      \begin{equation}\label{eq:numedod}
% 		      %   \lim_{\Delta t \rightarrow 0}
% 		      \boxed{
% 			      \mathcal{D}_{\Delta t}\left(x_j,t_n\right)
% 			      =  \left\{ x \, \Big| -\frac{at_n}{\nu} \leq x - x_j \leq 0 \right\}
% 		      }.
% 	      \end{equation}
% 	      Besides, the analytical domain of dependence for the linear advection PDE reads
% 	      \begin{align}\label{eq:anadod}
% 		      \mathcal{D}\left(x_j,t_n\right)
% 		      =  \left\{ x \, \Big|\, x = x_{j} - at_n \right\} .
% 	      \end{align}
% 	      Furthermore, the CFL condition enforces that
% 	      \begin{align}\label{eq:CFLconditionenforced}
% 		      \mathcal{D}\left(x_j,t_n\right) \subset \mathcal{D}_{\Delta t}\left(x_j,t_n\right),
% 	      \end{align}
% 	      which implies that characteristics should lie with the triangular zone 
% 	      under the line $(l_1)$ and $(l_2)$, as shown in \rFig{DomainOfDependence}.
% 	      Therefore, substitution of \eqref{eq:anadod} into \eqref{eq:numedod} yields
% 	      the CFL condition applied on the linear advection equation, as follows
% 	      \begin{align}
% 		      -\frac{at_n}{\nu} \leq \left(x_{j} - at_n\right) - x_j \leq 0 
% 		       & \Leftrightarrow
% 		      -\frac{at_n}{\nu} \leq - at_n \leq 0,
% 	      \end{align}
% 	      which, equally, leads to the CFL condtion
% 	      \begin{equation}\label{eq:CFLsol1}
% 		      \therefore\quad
% 		      \boxed{
% 			      0 \leq \nu \leq 1
% 			      \Leftrightarrow 
% 			      0 \leq \Delta t \leq \frac{\Delta x}{a}.
% 		      }
% 	      \end{equation}
% 	      Herein, the CFL condition \eqref{eq:CFLsol1} leads to 
% 	      constraint on the time step $\Delta t$ for the case when $a > 0$.
% 	      Note in passing that $\nu$ is non-negative.

% 	\item Perspective of fixed-point value:\\
% 	      Line $(l_1)$ passsing point $A(x_j,t_n)$ and $B(x_{j-1},t_{n-1})$
% 	      has the following form
% 	      \begin{align}
% 		      (l_1):\quad 
% 		      t = t_A + \frac{t_B-t_A}{x_B-x_A}\left(x-x_A\right)
% 		       & \Leftrightarrow
% 		      t = t_n + \frac{t_{n-1}-t_{n}}{x_{j-1}-x_{j}} \left(x-x_{j}\right) \notag \\
% 		       & \Leftrightarrow
% 		      t = t_n + \frac{-\Delta t}{-\Delta x} \left(x-x_{j}\right).
% 	      \end{align}
% 	      Hence, line $(l_1)$ passing line $t=0$ at point $D$ leads to the relation
% 	      \begin{align}\label{eq:linel1onesided}
% 		      x = x_j - \frac{t_n \Delta x}{\Delta t}
% 		      \Leftrightarrow 
% 		      x - x_j = -\frac{t_n \Delta x}{\Delta t}.
% 	      \end{align}
% 	      Likewise, line $(l_2)$ passing line $t=0$ at point $E$ leads to the relation
% 	      \begin{align}\label{eq:linel2onesided}
% 		      x - x_{j} = 0.
% 	      \end{align}
% 	      Therefore, combination of \eqref{eq:linel1onesided}
% 	      and \eqref{eq:linel2onesided} leads to the numerical domain of dependence
% 	      for the One-sided method in terms of fixed-point value
% 	      \begin{equation}\label{eq:method2valuedconsi}
% 		      \boxed{
% 			      \mathcal{D}_{\Delta t}\left(x_j,t_n\right)
% 			      =  \left\{ x\, \Big| -\frac{t_n \Delta x}{\Delta t} \leq x - x_j \leq 0 \right\}.
% 		      }
% 	      \end{equation}
% 	      Besides, the analytical domain of dependence for the linear advection PDE,
% 	      as given by \eqref{eq:anadod}, reads
% 	      \begin{align}\label{eq:anadod2}
% 		      \mathcal{D}\left(x_j,t_n\right)
% 		      =  \left\{ x \, \Big|\, x = x_{j} - at_n \right\} .
% 	      \end{align}
% 	      Then, by taking into consideration of requirement of the CFL condition, we obtain the 
% 	      following relation
% 	      \begin{align}\label{eq:resultsCFL2}
% 		      -\frac{t_n \Delta x}{\Delta t} \leq \left(x_{j} - at_n\right) - x_j \leq 0,
% 	      \end{align}
% 	      which we have substituted \eqref{eq:anadod2} into \eqref{eq:method2valuedconsi}.
% 	      Herein, the relation \eqref{eq:resultsCFL2} enforcing CFL condition
% 	      on the time step $\Delta t$
% 	      \begin{equation}
% 		      \therefore\quad
% 		      \boxed{
% 			      0 \leq \Delta t \leq \frac{\Delta x}{a},
% 		      }
% 	      \end{equation}
% 	      which is similar to \eqref{eq:CFLsol1}.
% \end{enumerate}

% \pagebreak
% %------------------------------------------------------------------------------
% \begin{example}\label{eq:example2}
% 	Examine the numerical domain of dependence 
% 	and the corresponding CFL condition of the One-sided method, 
% 	where the stencils point to the right.
% \end{example}
% \inputfig{floats/DomainOfDependenceTTR}{DomainOfDependenceTTR}

% Similarly, by following steps done in Example \ref{eq:example1} we obtain the following summary:

% \begin{enumerate}
% 	\item Point $E$ in terms of fixed-point value satisfying
% 	      \begin{align}
% 		      (l_2):\quad t = t_n + \frac{t_{n-1}-t_n}{x_{j+1}-x_j}\left(x-x_{j}\right)
% 	      \end{align}
% 	\item Numerical domain of dependence reads
% 	      \begin{align}
% 		      \mathcal{D}_{\Delta t}\left(x_j,t_n\right)
% 		      =  \left\{ x \, \Big|\, 0 \leq x - x_j \leq -\frac{at_n}{\nu} \right\}.
% 	      \end{align}

% 	\item Analytical domain of dependence reads
% 	      \begin{align}
% 		      \mathcal{D}\left(x_j,t_n\right)
% 		      =  \left\{ x \, \Big|\, x = x_{j} - at_n \right\} .
% 	      \end{align}

% 	\item CFL condition reads
% 	      \begin{equation}
% 		      \therefore\quad
% 		      \boxed{
% 			      \Delta t \geq \frac{\Delta x}{a}.
% 		      }
% 	      \end{equation}
% \end{enumerate}
% Note in passing that the advection velocity $a$ in this Example \ref{eq:example2} is $a<0$.

% \pagebreak
% %------------------------------------------------------------------------------
% \begin{example}\label{eq:example3}
% 	Examine the numerical domain of dependence 
% 	and the corresponding CFL condition for a three-point scheme.
% \end{example}
% \inputfig{floats/DomainOfDependence3pts}{DomainOfDependence3pts}

% Similarly, by following steps done in Example \ref{eq:example1} we obtain the following summary:

% \begin{enumerate}
% 	\item Point $D$ and $E$ in terms of fixed-point value satisfying
% 	      \begin{align}
% 		      (l_1):\quad t & = t_n + \frac{t_{n-1}-t_n}{x_{j-1}-x_j}\left(x-x_{j}\right) \\
% 		      (l_2):\quad t & = t_n + \frac{t_{n-1}-t_n}{x_{j+1}-x_j}\left(x-x_{j}\right)
% 	      \end{align}
% 	\item Numerical domain of dependence reads
% 	      \begin{align}
% 		      \mathcal{D}_{\Delta t}\left(x_j,t_n\right)
% 		      =  \left\{ x \, \Big|\, \left| x - x_j \right| \leq \frac{at_n}{\nu} \right\}.
% 	      \end{align}

% 	\item Analytical domain of dependence reads
% 	      \begin{align}
% 		      \mathcal{D}\left(x_j,t_n\right)
% 		      =  \left\{ x \, \Big|\, x = x_{j} - at_n \right\} .
% 	      \end{align}

% 	\item CFL condition reads
% 	      \begin{equation}
% 		      \therefore\quad
% 		      \boxed{
% 			      \left| \frac{a \Delta t}{\Delta x} \right| \leq 1.
% 		      }
% 	      \end{equation}
% \end{enumerate}

% \pagebreak
% %------------------------------------------------------------------------------
% \begin{example}
% 	Examine the numerical domain of dependence of \emph{Lax-Friedrichs} method.
% \end{example}
% \inputfig{floats/DomainOfDependenceLF}{DomainOfDependenceLF}

% Similarly, by following steps done in Example \ref{eq:example1},
% or the same as \ref{eq:example3}
% we obtain the following summary:

% \begin{enumerate}
% 	\item Point $D$ and $E$ in terms of fixed-point value satisfying
% 	      \begin{align}
% 		      (l_1):\quad t & = t_n + \frac{t_{n-1}-t_n}{x_{j-1}-x_j}\left(x-x_{j}\right) \\
% 		      (l_2):\quad t & = t_n + \frac{t_{n-1}-t_n}{x_{j+1}-x_j}\left(x-x_{j}\right)
% 	      \end{align}
% 	\item Numerical domain of dependence reads
% 	      \begin{align}
% 		      \mathcal{D}_{\Delta t}\left(x_j,t_n\right)
% 		      =  \left\{ x \, \Big|\, \left| x - x_j \right| \leq \frac{at_n}{\nu} \right\}.
% 	      \end{align}

% 	\item Analytical domain of dependence reads
% 	      \begin{align}
% 		      \mathcal{D}\left(x_j,t_n\right)
% 		      =  \left\{ x \, \Big|\, x = x_{j} - at_n \right\} .
% 	      \end{align}

% 	\item CFL condition reads
% 	      \begin{equation}
% 		      \therefore\quad
% 		      \boxed{
% 			      \left| \frac{a \Delta t}{\Delta x} \right| \leq 1.
% 		      }
% 	      \end{equation}
% \end{enumerate}

% \pagebreak
% % %------------------------------------------------------------------------------
% \section{von Neumann stability analysis}
% % \begin{example}
% % 	Perform the \emph{von Neumann} stability analysis for \emph{Lax-Friedrichs} scheme
% % 	\begin{align}\label{scheme-01}
% % 		u_j^{n+1} = 
% % 		\frac{1}{2}
% % 		\left( u_{j+1}^n + u_{j-1}^n \right)
% % 		- \frac{\nu}{2}\left( u_{j+1}^n - u_{j-1}^n \right),
% % 	\end{align}
% % 	where $v:=a\Delta t/\Delta x$ is the \emph{CFL} number.
% % 	Determine if the numerical scheme is unstable, conditionally stable, or 
% % 	unconditionally stable.
% % \end{example}
% % Approach:
% % Let $\displaystyle \bar{u}_j^n$, $u_j^n$ and $e_j^n$ be the exact solution,
% % computed solution, and the error of the computed solution of the above numerical method,
% % respectively. Then, the computed solution can be written as follows
% % \begin{align}\label{scheme-01-error}
% % 	u_j^n  = \bar{u}_j^n + e_j^n.
% % \end{align}
% % Substitution of the expression \eqref{scheme-01-error} into \eqref{scheme-01} leads to
% % \begin{align}\label{scheme-01-2}
% % 	\eqref{scheme-01}
% % 	 & \stackrel{\eqref{scheme-01-error}}{\Leftrightarrow}
% % 	\bar{u}_j^{n+1} + e_j^{n+1}
% % 	= \bar{u}_j^n + e_j^n
% % 	- \frac{c}{2} \left(
% % 	\left(\bar{u}_{j+1}^n + e_{j+1}^n\right)
% % 	- \left(
% % 		\bar{u}_{j-1}^n + e_{j-1}^n
% % 		\right)
% % 	\right)                                                \notag \\
% % 	 & \Leftrightarrow
% % 	\left(
% % 	\bar{u}_j^{n+1} - \bar{u}_j^n
% % 	+ \frac{c}{2} \left( \bar{u}_{j+1}^n - \bar{u}_{j-1}^n\right)
% % 	\right)
% % 	+ \left(
% % 	e_j^{n+1} - e_j^n + \frac{c}{2} \left( e_{j+1}^n - e_{j-1}^n \right) \right)
% % 	= 0.
% % \end{align}
% % %   \begin{align*}
% % %       \left(
% % %       \bar{u}_i^{j+1} - \bar{u}_i^j
% % %       + \frac{c}{2} \left( \bar{u}_{i+ 1}^j - \bar{u}_{i - 1 }^j\right )
% % %       \right)
% % %       + \left( e_i^{j+1} - e_i^j + \frac{c}{2} \left( e_{i+ 1}^j - e_{i - 1 }^j \right ) \right) = 0
% % %   \end{align*}
% % Since $\bar{u}_i^j$ is the exact solution, $\bar{u}_i^j$ also satisfies the equation
% % for the numerical method. Therefore, the equation \eqref{scheme-01-2} is now reduced to
% % \begin{align}\label{scheme1-err-eqn}
% % 	e_j^{n+1} - e_j^n
% % 	+ \frac{c}{2} \left( e_{j+1}^n - e_{j-1}^n \right) = 0
% % \end{align}
% % Herein, this relation \eqref{scheme1-err-eqn} implies that
% % the error of the computed solution
% % also satisfies the difference equation. Next,
% % by using the \emph{von Neumann} stability analysis,
% % the error $e_j^n$ can be represented as follows
% % \begin{align}\label{eq:vonneumannstabil}
% % 	e_j^{n} = A^n e^{\mathbf{i} j \phi }
% % \end{align}
% % where $A^n$ is the amplitude at time step $n$,
% % and $\mathbf{i}$ is the imaginary unit satisfying
% % the equation $\mathbf{i}^2 = -1$.
% % Then, by substituting the expression \eqref{eq:vonneumannstabil}
% % into the error equation \eqref{scheme1-err-eqn} we arrive at
% % \begin{align}\label{eq:vonneumannstabil2}
% % 	\eqref{scheme1-err-eqn}
% % 	 & \stackrel{\eqref{eq:vonneumannstabil}}{\Leftrightarrow}
% % 	A^{n+1} e^{\mathbf{i} j \phi } - A^{n} e^{\mathbf{i} j \phi }
% % 	+\frac{c}{2} \left(
% % 	A^{n} e^{\mathbf{i} (j+1) \phi }
% % 	- A^{n} e^{\mathbf{i} (j-1) \phi }
% % 	\right)                                                   \notag \\
% % 	 & \Leftrightarrow
% % 	A^{n+1} = A^n \left(
% % 	1 - \frac{c}{2} \left ( e^{\mathbf{i} \phi} - e^{- \mathbf{i} \phi} \right)
% % 	\right)  .
% % \end{align}
% % Next, we define the amplification factor $G$ as
% % $\displaystyle G := A^{n+1}/A^n$.
% % Furthermore, in order to make sure the scheme to be stable, we have to show that $|G| \leq 1$.
% % For the present scheme, the amplification factor $G$ is recognized as
% % \begin{align*}
% % 	G = \frac{A^{j+1}}{A^j}
% % 	\stackrel{\eqref{eq:vonneumannstabil2}}{=}
% % 	1 - \frac{c}{2} \left( e^{\mathbf{i} \phi} - e^{- \mathbf{i} \phi} \right)
% % 	= 1 -  \mathbf{i} c\sin \phi,
% % \end{align*}
% % which leads to the following expression
% % \begin{align*}
% % 	|G| = \sqrt{G G^*} = \sqrt{ 1 +  c^2\sin^2 \phi} \geq 1 \quad \forall\,c> 0.
% % \end{align*}
% % Therefore, the forward in time and central difference in space scheme is
% % unconditionally unstable, or simply, \textbf{unstable}.
% % For the Lax - Friedrich scheme, the amplification factor is given by 
% % \begin{align}
% % 	g(\xi) =  \cos \xi - \mathbf{i} \nu \sin\xi \quad 
% % \end{align}
% % which leads to
% % \begin{align}
% % 	|g(\xi)| = 
% % 	\sqrt{\cos^2(\xi)   + \nu^2 \sin^2(\xi)} \leq 1 ,
% % 	\quad \text{for} \quad 0 < \nu \leq 1.
% % \end{align}
% % Therefore, the \emph{Lax-Friedrichs} scheme is 
% % \emph{conditionally stable} for $0<\nu\leq 1$. 

% \begin{example}
% 	von Neumann stability analysis for Upwind method.
% \end{example}
% \inputfig{floats/vonNeumannstabil}{vonNeumannstabil}

% \begin{example}
% 	Summary consistency $+$ stability $\Rightarrow$ Convergence
% \end{example}

% \pagebreak
% %------------------------------------------------------------------------------
% \section{Conservation form - Finite Volume Method}
% \begin{example}
% 	Derivation of conservation form.
% \end{example}

% % \begin{example}
% % 	Consider the \emph{Burgers}' equation
% % 	\begin{align*}
% % 		u_t + \left(\frac{1}{2}u^2\right)_x = 0.
% % 	\end{align*}
% % 	\begin{enumerate}
% % 		\item Show that the upwind scheme for this version of \emph{Burgers}' equation as follows
% % 		      \begin{align*}
% % 			      u_j^{n+1} = u_j^n + \frac{\Delta t}{\Delta x} \left(
% % 			      \frac{1}{2}(u_{j-1}^n)^2 - \frac{1}{2}(u_j^n)^2 \right)
% % 		      \end{align*}
% % 		      \emph{can} be represented in \emph{conservation form}.

% % 		\item If we rewrite the above \emph{Burgers}'
% % 		      equation as
% % 		      \begin{align*}
% % 			      u_t + uu_x = 0,
% % 		      \end{align*}
% % 		      we can assume it as a linear advection problem with speed $u$. Prove that
% % 		      the upwind scheme for the latter equation as
% % 		      \begin{align*}
% % 			      u_j^{n+1} = u_j^n + \frac{u_j^n \Delta t}{\Delta x} (u_{j-1}^n-u_j^n)
% % 		      \end{align*}
% % 		      \emph{cannot} be represented in \emph{conservation form}.
% % 	\end{enumerate}
% % \end{example}
% % Approach:
% % \begin{enumerate}
% % 	\item If we take the numerical flux
% % 	      \begin{align*}
% % 		      \tilde{F}_{j+1/2}(u_j^n, u_{j+1}^n) = f(u_j^n) = \frac{1}{2} (u_j^n)^2,
% % 	      \end{align*}
% % 	      then the corresponding numerical scheme can be rewritten in conservation form
% % 	      \begin{align*}
% % 		      u_j^{n+1} = u_j^n + \frac{\Delta t}{\Delta
% % 			      x}(\tilde{F}_{j-1/2}(u_{j-1}^n, u_j^n) - \tilde{F}_{j+1/2}(u_j^n,
% % 		      u_{j+1}^n)).
% % 	      \end{align*}

% % 	\item
% % 	      \begin{proof}
% % 		      Let us take a look at the derivatives of $H_{\Delta t}$ by
% % 		      $u_{j-1}^n, u_j^n$ and $u_{j+1}^n$ as follows
% % 		      \begin{align*}
% % 			      x & = \partial_{u_{j-1}^n} H_{\Delta t} = \frac{u_j^n \Delta t}{\Delta x}, \\
% % 			      y & = \partial_{u_j^n} H_{\Delta t} = 1 + \frac{u_{j-1}^n \Delta
% % 				      t}{\Delta x} - \frac{2u_j^n \Delta t}{\Delta x},                           \\
% % 			      z & = \partial_{u_{j+1}^n} H_{\Delta t} = 0.
% % 		      \end{align*}
% % 		      Summation of the term $x|_{j+1}$, the term $y|_j$, and the term $z|_{j-1}$
% % 		      must be equal to $1$ in order to satisfy the conservation form.
% % 		      However, we observe that the summation of these three terms does not satisfy
% % 		      the equality to $1$, as follows
% % 		      \begin{align*}
% % 			      x|_{j+1} + y|_j + z|_{j-1}
% % 			       & = 1 + \frac{\Delta t}{\Delta x} (u_{j+1}^n - 2u_j^n + u_{j-1}j^n) \\
% % 			       & \ne 1.
% % 		      \end{align*}
% % 		      Besides, if we would like to check this property for the first scheme,
% % 		      we obtain the following relations
% % 		      \begin{align*}
% % 			      x & = \frac{\Delta t}{\Delta x}u_{j-1}^n,  \\
% % 			      y & = 1 - \frac{\Delta t}{\Delta x} u_j^n, \\
% % 			      z & = 0,
% % 		      \end{align*}
% % 		      which leads to the summation of these three terms,
% % 		      i.e. the term $x|_{j+1}$, the term $y|_j$, and the term $z|_{j-1}$,
% % 		      equal to $1$, as follows
% % 		      \begin{align*}
% % 			      x|_{j+1} + y|_j + z|_{j-1} = 1.
% % 		      \end{align*}
% % 	      \end{proof}
% % \end{enumerate}



% % %------------------------------------------------------------------------------
% % \section{Comparison: Order of consistency}

% % %------------------------------------------------------------------------------
% % \section{Godunov}

% % %------------------------------------------------------------------------------
% % \section{TVD}

% % %------------------------------------------------------------------------------
% % \section{Fourier transformation}

% % %------------------------------------------------------------------------------
% % \section{Linearity - Oder of consistency - Stability}

% % %------------------------------------------------------------------------------
% % \section{Time step methods: Comparison}
% % \begin{enumerate}
% % 	\item Upwind left - One-sided method
% % 	\item Upwind right - One-sided method
% % 	\item Lax-Friedrichs
% % 	\item Lax-Wendroff
% % 	\item Leapfrog
% % 	\item 
% % \end{enumerate}

% % %------------------------------------------------------------------------------
% % \section{Remark about Rarefaction wave solution}
% % \begin{example}
% % 	Show that for a general convex scalar problem 
% % 	\begin{align*}
% % 		u_t + f(u)_x = 0	
% % 	\end{align*}
% % 	with initial condition
% % 	in form of \emph{Riemann}'s problem 
% % 	\begin{align*}
% % 		u(x,t=0) = 
% % 		\begin{cases}
% % 			u_L, \quad x<0, \\
% % 			u_R, \quad x>0,
% % 		\end{cases}
% % 		\quad\text{where}\quad
% % 		u_L<u_R,
% % 	\end{align*}
% % 	the rarefaction wave solution
% % 	is given by 
% % 	\begin{align*}
% % 		u(x,t) =
% % 		\begin{cases} 
% % 			u_L,    & \quad x/t < f'(u_L),                 \\
% % 			v(x/t), & \quad f'(u_L) \leq x/t \leq f'(u_R), \\
% % 			u_R,    & \quad x/t > f'(u_R),
% % 		\end{cases}
% % 	\end{align*}
% % 	where $v(\xi)$ is the solution to $f'(v(\xi)) = \xi$.
% % \end{example}

% % \begin{checkexample}
% % 	Let matrix $A$ be given as 
% % 	\begin{align*}
% % 		A = \begin{pmatrix} 2 &-1 \\ -1 &2 \end{pmatrix}
% % 	\end{align*}
% % 	then the diagonalization with transformation matrix $R$ and $T$ goes as follows
% % 	\begin{align*}
% % 		A = R\Lambda R^{-1} = \begin{pmatrix} 2 &-1 \\ -1 &2 \end{pmatrix}
% % 	\end{align*}
% % \end{checkexample}

% % \begin{example}
% % 	Let $u$ be a scalar-valued function in $C^1(\mathbb{R}\times\mathbb{R^+})$ and defined as follows 
% % 	\begin{align}
% % 		u:
% % 		\begin{cases}
% % 			\mathbb{R}\times\mathbb{R^+} \rightarrow \mathbb{R}, \\
% % 			(x,t) \mapsto u(x,t).                                
% % 		\end{cases}
% % 	\end{align}
% % 	Consider the natural 1D second-order wave equation 
% % 	\begin{equation}\label{eq:wave1D}
% % 		u_{tt} = \alpha^2 u_{xx},\quad \forall\alpha\in\mathbb{R}, 
% % 	\end{equation}
% % 	with initial conditions given 
% % 	\begin{align}
% % 		\begin{cases}
% % 			\begin{aligned}
% % 				u(x,0)     & = u_{0}(x), \\
% % 				u_{t}(x,0) & = u_{1}(x). \\
% % 			\end{aligned}
% % 		\end{cases}
% % 	\end{align}
% % 	Find solution to \eqref{eq:wave1D} by means of linear hyperbolic system of conservation laws.
% % \end{example}

% % Approach: By introducing two new variables called $v$ and $w$ defined as follows
% % \begin{align}\label{eq:vwdefined}
% % 	\begin{cases}
% % 		\begin{aligned}
% % 			v & := u_{x}, \\
% % 			w & := u_{t},
% % 		\end{aligned}
% % 	\end{cases}
% % \end{align}
% % we obtain the following relations
% % \begin{align}\label{eq:systemraw}
% % 	\begin{cases}
% % 		\begin{aligned}
% % 			v_{x} 
% % 			 & 
% % 			\stackrel{\eqref{eq:vwdefined}}{=} u_{xx}
% % 			\stackrel{\eqref{eq:wave1D}}{=} \frac{1}{\alpha^2} u_{tt}
% % 			\stackrel{\eqref{eq:vwdefined}}{=} \frac{1}{\alpha^2} w_{t}, \\
% % 			w_{x}
% % 			 & 
% % 			\stackrel{\eqref{eq:vwdefined}}{=} u_{tx}
% % 			\stackrel{\eqref{eq:vwdefined}}{=} u_{xt}
% % 			\stackrel{\eqref{eq:vwdefined}}{=} v_{t},
% % 		\end{aligned}
% % 	\end{cases}
% % 	\Leftrightarrow
% % 	\begin{cases}
% % 		v_{t} - w_{x} = 0, \\
% % 		w_{t} - \alpha^2 v_{x} = 0.
% % 	\end{cases}
% % \end{align}
% % Note in passing that $u_{tx}=u_{xt}$ since $u\in C^1(\mathbb{R}\times\mathbb{R^+})$.
% % The later system in \eqref{eq:systemraw} can be rewritten in form of a system, whose derivative term
% % w.r.t. time $t$ and space $x$ follows with each other
% % \begin{align}\label{eq:system1}
% % 	\begin{pmatrix} v\\ w\end{pmatrix}_{t}
% % 	+
% % 	\begin{pmatrix} -w \\ -\alpha^2 v \end{pmatrix}_{x}
% % 	= 0.
% % \end{align}
% % In order to mimic explicitly the form of a system of conservation laws, the system in \eqref{eq:system1}
% % is recast into the following system
% % \begin{align}\label{eq:system2}
% % 	\begin{pmatrix} v\\ w\end{pmatrix}_{t}
% % 	+
% % 	\begin{pmatrix} 0 & -1 \\ -\alpha^2 & 0 \end{pmatrix}
% % 	\begin{pmatrix} v\\ w\end{pmatrix}_{x}
% % 	= 0.
% % \end{align}
% % The initial condition becomes
% % \begin{align}
% % 	\begin{cases}
% % 		\begin{aligned}
% % 			v(x,0) & = u'_{0}(x), \\
% % 			w(x,0) & = u_{1}(x).  \\
% % 		\end{aligned}
% % 	\end{cases}
% % \end{align}
% % By defining the new variables
% % \begin{align}
% % 	U := \begin{pmatrix} v\\w\end{pmatrix}
% % 	\quad\text{and}\quad
% % 	A := \begin{pmatrix} 0 &-1\\-\alpha^2 &0 \end{pmatrix},
% % \end{align}
% % we now write \eqref{eq:system2} as 
% % \begin{equation}\label{eq:system3}
% % 	U_{t} + AU_{x} = 0
% % \end{equation}
% % Next, we realize that the system \eqref{eq:system2} is not ready to solve since the two PDEs are 
% % interlocked with each other. The approach for solution to \eqref{eq:system3} goes as follows
% % \begin{enumerate}
% % 	\item Eigenvalue problem for matrix $A$ yields
% % 	      \begin{align}\label{eq:eigenvalueproblem}
% % 		      A
% % 		      % \begin{pmatrix} v^{A}_{1} \\ v^{A}_{2} \end{pmatrix}
% % 		      v^{A}
% % 		      =\lambda
% % 		      v^{A}
% % 		      \Leftrightarrow
% % 		      (A-\lambda I)v^{A}=0
% % 		      \Leftrightarrow
% % 		      \begin{pmatrix} -\lambda &-1\\-\alpha^2 & -\lambda \end{pmatrix}
% % 		      \begin{pmatrix} v^{A}_{1} \\ v^{A}_{2} \end{pmatrix}
% % 		      =
% % 		      \begin{pmatrix} 0 \\ 0 \end{pmatrix}
% % 	      \end{align}
% % 	      where $\lambda$ is eigenvalue and $v^{A} = (v^{A}_{1},v^{A}_{2})^{\top}$ eigenvector.
% % 	\item Characteristic polynomial
% % 	      \begin{align}
% % 		      \det(A-\lambda I) = 0 
% % 		      \Leftrightarrow
% % 		      \lambda^2 - \alpha^2 = 0 
% % 		      \Leftrightarrow 
% % 		      \begin{cases}
% % 			      \lambda_{1}   = +\alpha \\
% % 			      \lambda_{2}   = -\alpha
% % 		      \end{cases}
% % 	      \end{align}
% % 	      \begin{enumerate}
% % 		      \item For the $1^{st}$ eigenvalue $\lambda_{1} = +\alpha$, we obtain its corresponding eigenvector
% % 		            \begin{align}
% % 			            \eqref{eq:eigenvalueproblem}
% % 			            \Leftrightarrow
% % 			            \begin{pmatrix} -\alpha &-1\\-\alpha^2 & -\alpha \end{pmatrix}
% % 			            \begin{pmatrix} v^{A}_{1} \\ v^{A}_{2} \end{pmatrix}
% % 			            =
% % 			            \begin{pmatrix} 0 \\ 0 \end{pmatrix}
% % 		            \end{align}
% % 		            \emph{Gaussian}-elimination leads to
% % 		            \begin{align}
% % 			            \begin{pmatrix} -\alpha &-1\\-\alpha^2 & -\alpha \end{pmatrix}
% % 			            %   \lelem{s_{1} \leftrightarrow s_{2},\\ s_{2} \leftrightarrow s_{4},\\ s_{3} \leftrightarrow s_{4}}
% % 			             & \lelem{r_{1} \leftrightarrow r_{2}}
% % 			            \begin{pmatrix} -\alpha^2 & -\alpha\\ -\alpha &-1 \end{pmatrix}\notag            \\
% % 			             & \lelem{r_{1}:=-\frac{1}{\alpha^2} r_{1}}
% % 			            \begin{pmatrix} 1 &1/\alpha\\-\alpha & -1 \end{pmatrix}
% % 			            \lelem{r_{2}:=\alpha r_{1}+r_{2} }
% % 			            \begin{pmatrix} 1 &1/\alpha\\0 & 0 \end{pmatrix}
% % 		            \end{align}
% % 		            Hence, eigenvector for $\lambda_{1} = +\alpha$ reads
% % 		            \begin{align}
% % 			            v^{A(\lambda_{1})}
% % 			            =\begin{pmatrix} v^{A(\lambda_{1})}_{1}\\v^{A(\lambda_{1})}_{2} \end{pmatrix}
% % 			            =\begin{pmatrix} 1\\ -\alpha \end{pmatrix}
% % 		            \end{align}

% % 		      \item Similarly, for the $2^{nd}$ eigenvalue $\lambda_{1} = -\alpha$,
% % 		            we obtain its corresponding eigenvector as follows
% % 		            \begin{align}
% % 			            \eqref{eq:eigenvalueproblem}
% % 			            \Leftrightarrow
% % 			            \begin{pmatrix} +\alpha &-1\\-\alpha^2 & +\alpha \end{pmatrix}
% % 			            \begin{pmatrix} v^{A}_{1} \\ v^{A}_{2} \end{pmatrix}
% % 			            =
% % 			            \begin{pmatrix} 0 \\ 0 \end{pmatrix}
% % 		            \end{align}
% % 		            \emph{Gaussian}-elimination leads to
% % 		            \begin{align}
% % 			            \begin{pmatrix} +\alpha &-1\\-\alpha^2 & +\alpha \end{pmatrix}
% % 			             & \lelem{r_{1} \leftrightarrow r_{2}}
% % 			            \begin{pmatrix} -\alpha^2 & +\alpha\\ +\alpha &-1 \end{pmatrix}\notag            \\
% % 			             & \lelem{r_{1}:=-\frac{1}{\alpha^2} r_{1}}
% % 			            \begin{pmatrix} 1 &-1/\alpha\\+\alpha & -1 \end{pmatrix}
% % 			            \lelem{r_{2}:=-\alpha r_{1}+r_{2} }
% % 			            \begin{pmatrix} 1 &-1/\alpha\\0 & 0 \end{pmatrix}
% % 		            \end{align}
% % 		            Hence, eigenvector for $\lambda_{1} = -\alpha$ reads
% % 		            \begin{align}
% % 			            v^{A(\lambda_{2})}
% % 			            =\begin{pmatrix} v^{A(\lambda_{2})}_{1}\\v^{A(\lambda_{2})}_{2} \end{pmatrix}
% % 			            =\begin{pmatrix} 1\\ \alpha \end{pmatrix}
% % 		            \end{align}
% % 	      \end{enumerate}
% % 	\item Diagonalizable
% % 	      \begin{align}
% % 		      A & = T\Lambda T^{-1}\notag            \\
% % 		        & =
% % 		      \begin{pmatrix}\begin{array}{c|c} v^{A(\lambda_{1})} & v^{A(\lambda_{2})}\end{array}\end{pmatrix}
% % 		      \begin{pmatrix} \lambda_{1} &0 \\ 0&\lambda_{2}\end{pmatrix}
% % 		      \begin{pmatrix}\begin{array}{c|c} v^{A(\lambda_{1})} & v^{A(\lambda_{2})}\end{array}\end{pmatrix}^{-1}\notag \\
% % 		        & =
% % 		      \begin{pmatrix}\begin{array}{c|c} 1 & 1\\ -\alpha&\alpha \end{array}\end{pmatrix}
% % 		      \begin{pmatrix} +\alpha &0 \\ 0&-\alpha\end{pmatrix}
% % 		      \begin{pmatrix}\begin{array}{c|c} 1 & 1\\ -\alpha&\alpha \end{array}\end{pmatrix}^{-1}\notag \\
% % 		        & =
% % 		      \begin{pmatrix} 1 & 1\\ -\alpha&\alpha \end{pmatrix}
% % 		      \begin{pmatrix} +\alpha &0 \\ 0&-\alpha\end{pmatrix}
% % 		      \begin{pmatrix} 1 & 1\\ -\alpha&\alpha \end{pmatrix}^{-1}\notag 
% % 	      \end{align}
% % 	\item Derivation
% % 	      \begin{align}\label{eq:system4}
% % 		      U_{t} + AU_{x} = 0 
% % 		       & \Leftrightarrow
% % 		      T^{-1}U_{t} + T^{-1}AU_{x} = 0\notag               \\
% % 		       & \Leftrightarrow
% % 		      T^{-1}U_{t} + T^{-1}T\Lambda T^{-1}U_{x} = 0\notag \\
% % 		       & \Leftrightarrow
% % 		      T^{-1}U_{t} + \Lambda T^{-1}U_{x} = 0\notag        \\
% % 		       & \Leftrightarrow
% % 		      W_{t} + \Lambda W_{x} = 0\notag                    \\
% % 		       & \Leftrightarrow
% % 		      \begin{pmatrix} \xi \\ \eta \end{pmatrix}_{t}
% % 		      + \Lambda
% % 		      \begin{pmatrix} \xi \\ \eta \end{pmatrix}_{x}
% % 		      = 0 \notag                                         \\
% % 		       & \Leftrightarrow
% % 		      \begin{pmatrix} \xi \\ \eta \end{pmatrix}_{t}
% % 		      + 
% % 		      \begin{pmatrix} +\alpha &0 \\ 0&-\alpha\end{pmatrix}
% % 		      \begin{pmatrix} \xi \\ \eta \end{pmatrix}_{x}
% % 		      = 0 \notag                                         \\
% % 		       & \Leftrightarrow
% % 		      \begin{cases}
% % 			      \xi_{t} + \alpha\xi_{x} = 0   \\
% % 			      \eta_{t} - \alpha\eta_{x} = 0 \\
% % 		      \end{cases}
% % 		      \Leftrightarrow
% % 		      \begin{cases}
% % 			      \xi(x,t) = \xi_{0}(x-\alpha t)   \\
% % 			      \eta(x,t) = \eta_{0}(x+\alpha t) \\
% % 		      \end{cases}
% % 	      \end{align}
% % 	      Note in passing that 
% % 	      \begin{align}
% % 		      \begin{pmatrix} \xi \\ \eta \end{pmatrix} =   
% % 		      W := T^{-1}U 
% % 		       & = \begin{pmatrix} 1 & 1\\ -\alpha&\alpha \end{pmatrix}^{-1}
% % 		      \begin{pmatrix} v\\w\end{pmatrix}           \\
% % 		       & = \frac{1}{\alpha-(-\alpha)}
% % 		      \begin{pmatrix} \alpha & -1\\ \alpha&1 \end{pmatrix}
% % 		      \begin{pmatrix} v\\w\end{pmatrix}\notag     \\
% % 		       & =
% % 		      \begin{pmatrix} 1/2 & -1/(2\alpha)\\ 1/2&1/(2\alpha) \end{pmatrix}
% % 		      \begin{pmatrix} v\\w\end{pmatrix} \notag    \\
% % 		       & = 
% % 		      \begin{pmatrix} (1/2) v - 1/(2\alpha) w \\ (1/2) v + 1/(2\alpha) w \end{pmatrix}.
% % 	      \end{align}
% % 	      The initial conditions are as follows
% % 	      \begin{align}
% % 		      \xi_{0}(x)
% % 		      = \xi(x,0)
% % 		       & = \frac{1}{2}v(x,0)-\frac{1}{2\alpha}w(x,0)
% % 		      = \frac{1}{2}u'_{0}(x)-\frac{1}{2\alpha}u_{1}(x) \\
% % 		      \eta_{0}(x) = \eta(x,0)
% % 		       & = \frac{1}{2}v(x,0)+\frac{1}{2\alpha}w(x,0) 
% % 		      = \frac{1}{2}u'_{0}(x)+\frac{1}{2\alpha}u_{1}(x)
% % 	      \end{align}
% % 	      Likewise,
% % 	      \begin{align}
% % 		      U = TW 
% % 		       & \Leftrightarrow 
% % 		      \begin{pmatrix} v\\w\end{pmatrix}
% % 		      =
% % 		      \begin{pmatrix} 1 & 1\\ -\alpha&\alpha \end{pmatrix}
% % 		      \begin{pmatrix} \xi\\\eta \end{pmatrix} \notag \\
% % 		       & \Leftrightarrow 
% % 		      \begin{cases}
% % 			      \begin{aligned}
% % 				      v(x,t) & = \xi(x,t) + \eta(x,t)              \\
% % 				      w(x,t) & = -\alpha\xi(x,t) + \alpha\eta(x,t)
% % 			      \end{aligned}
% % 		      \end{cases}       \\
% % 		       & \Leftrightarrow 
% % 		      \begin{cases}
% % 			      \begin{aligned}
% % 				      v(x,t) & = \xi_{0}(x-\alpha\,t) + \eta_{0}(x+\alpha\,t)              \\
% % 				      w(x,t) & = -\alpha\xi_{0}(x-\alpha\,t) + \alpha\eta_{0}(x+\alpha\,t)
% % 			      \end{aligned}
% % 		      \end{cases}
% % 	      \end{align}
% % 	      Therefore, the final solutions explicitly reads
% % 	      \begin{align}
% % 		      \begin{cases}
% % 			      \begin{aligned}
% % 				      v(x,t) & = 
% % 				      \frac{1}{2}u'_{0}(x-\alpha\,t)-\frac{1}{2\alpha}u_{1}(x-\alpha\,t)
% % 				      +
% % 				      \frac{1}{2}u'_{0}(x+\alpha\,t)+\frac{1}{2\alpha}u_{1}(x+\alpha\,t) \\
% % 				      w(x,t) & = 
% % 				      -\frac{\alpha}{2}u'_{0}(x-\alpha\,t)+\frac{1}{2}u_{1}(x-\alpha\,t) 
% % 				      + 
% % 				      \frac{\alpha}{2}u'_{0}(x+\alpha\,t)+\frac{1}{2}u_{1}(x+\alpha\,t)
% % 			      \end{aligned}
% % 		      \end{cases}
% % 	      \end{align}
% % \end{enumerate}

% % \begin{example}
% % 	Consider the \textbf{linearized shallow water equations}
% % 	\begin{align}
% % 		\begin{pmatrix} u\\ \varphi\end{pmatrix}_{t}
% % 		+
% % 		\begin{pmatrix} \bar{u} & 1 \\ \bar{\varphi} & \bar{u} \end{pmatrix}
% % 		\begin{pmatrix} u \\ \varphi\end{pmatrix}_{x}
% % 		= 0,
% % 	\end{align}
% % 	where $\bar{u}, \bar{\varphi} \in \mathbb{R}$ are given, 
% % 	and sufficient initial conditions for $u$ and $\varphi$ are provided.
% % \end{example}
% % \clearpage
% % %------------------------------------------------------------------------------
% % \begin{example}
% % 	Consider
% % 	\begin{align*}
% % 		u_t + Au_x = 0 \qquad\text{for}\quad (x,t)\in \mathbb{R} \times \mathbb{R}^+,
% % 	\end{align*}
% % 	where the unknown vector $u(x,t)$ is defined, and the matrix $A$ is given as follows
% % 	\begin{align*}
% % 		\begin{array}{l}
% % 			u(x, t) :=
% % 			\begin{pmatrix} u_1(x,t) \\ u_2(x,t) \end{pmatrix},
% % 			\quad\text{and}\quad
% % 			A = \begin{pmatrix} 2 & -1 \\ -1 & 2 \end{pmatrix}, \\
% % 		\end{array}
% % 	\end{align*}
% % 	and the initial conditions are given by
% % 	\begin{align*}
% % 		\begin{cases}
% % 			u_1(x, 0) & = 1, \\
% % 			u_2(x, 0) & =
% % 			\begin{cases}
% % 				1,  & x < 0, \\
% % 				-1, & x > 0.
% % 			\end{cases}
% % 		\end{cases}
% % 	\end{align*}
% % \end{example}
% % Approach: Eigenvalue decomposition of matrix $A$ is given as follows
% % \begin{align*}
% % 	A = R \Lambda R^{-1}, \quad
% % 	R =
% % 	\begin{pmatrix}
% % 		-1 & -1 \\ -1 & 1
% % 	\end{pmatrix}, \quad
% % 	R^{-1} = \frac{1}{2} R, \quad
% % 	\Lambda =
% % 	\begin{pmatrix}
% % 		1 & 0 \\ 0 & 3
% % 	\end{pmatrix}.
% % \end{align*}
% % Hence, we can now change variable $u$ to a new variable $w$ as follows
% % \begin{align*}
% % 	w := R^{-1}u = \frac{1}{2} R u, \quad\text{so}\quad u = Rw.
% % \end{align*}
% % Then, the initial problem can be rewritten as
% % \begin{align*}
% % 	w_t + \Lambda w_x = 0
% % 	\Leftrightarrow
% % 	\begin{cases}
% % 		(w_{1})_{t} +  \lambda_{1}(w_{1})_{x} = 0, \\
% % 		(w_{2})_{t} +  \lambda_{2}(w_{2})_{x} = 0,
% % 	\end{cases}
% % \end{align*}
% % with the diagonal matrix $\Lambda$. This step simply means that variables $w_1$ and
% % $w_2$ are independent of each other, i.e. the original system has been fully decoupled.
% % Next, let us find the initial condition in terms
% % of $w$ as follows
% % \begin{align*}
% % 	w(x, 0)
% % 	= \frac{1}{2} R u(x, 0)
% % 	= \frac{1}{2}
% % 	\begin{pmatrix}
% % 		-1 & -1 \\ -1 & 1
% % 	\end{pmatrix}
% % 	\begin{pmatrix}
% % 		u_{1}(x,0) \\ u_{2}(x,0)
% % 	\end{pmatrix}
% % 	=
% % 	\begin{pmatrix}
% % 		-\left(u_1(x, 0) + u_2(x, 0)\right)/2 \\
% % 		-\left(u_1(x, 0) - u_2(x, 0)\right)/2
% % 	\end{pmatrix},
% % \end{align*}
% % which yields the following relation
% % \begin{align*}
% % 	\begin{pmatrix}
% % 		w_{1}(x,0) \\ w_{2}(x,0)
% % 	\end{pmatrix}
% % 	=
% % 	\begin{pmatrix}
% % 		-\left(u_1(x, 0) + u_2(x, 0)\right)/2 \\
% % 		-\left(u_1(x, 0) - u_2(x, 0)\right)/2
% % 	\end{pmatrix}.
% % \end{align*}
% % Therefore, the initial conditions for the transformed variable $w_1$ read
% % \begin{align*}
% % 	w_1(x, 0) =
% % 	\begin{cases}
% % 		-(1+1)/2=-1, & x < 0, \\
% % 		-(1-1)/2=0,  & x > 0,
% % 	\end{cases}
% % \end{align*}
% % and for the variable $w_2$ read
% % \begin{align*}
% % 	w_2(x, 0) =
% % 	\begin{cases}
% % 		-(1-1)/2=0,     & x < 0, \\
% % 		-(1-(-1))/2=-1, & x > 0.
% % 	\end{cases}
% % \end{align*}
% % Since we have the advection problem with speeds
% % $a_1=1$ and $a_2=3$, which are the diagonal values of the diagonal matrix $\Lambda$, for both $w_1$ and $w_2$,
% % we can derive explicit solution to each equation. The explicit solution $w_1$ reads
% % \begin{align*}
% % 	w_1(x, t)
% % 	= w_1(x-t, 0)
% % 	=
% % 	\begin{cases}
% % 		-1, & x-t < 0, \\
% % 		0,  & x-t > 0,
% % 	\end{cases}
% % 	\Leftrightarrow
% % 	w_1(x, t) =
% % 	\begin{cases}
% % 		-1, & x<t, \\
% % 		0,  & x>t,
% % 	\end{cases}
% % \end{align*}
% % and, likewise, the explicit solution $w_2$ reads
% % \begin{align*}
% % 	w_2(x, t)
% % 	= w_2(x-3t,0)
% % 	=
% % 	\begin{cases}
% % 		0,  & x-3t < 0, \\
% % 		-1, & x-3t > 0,
% % 	\end{cases}
% % 	\Leftrightarrow
% % 	w_2(x, t) =
% % 	\begin{cases}
% % 		0,  & x<3t, \\
% % 		-1, & x>3t.
% % 	\end{cases}
% % \end{align*}
% % Now we trace back to variable $u$ from variable $w$ by taking into the consideration of the relation
% % \begin{align*}
% % 	u(x,t) = Rw(x,t) =
% % 	\begin{pmatrix}
% % 		-1 & -1 \\ -1 & 1
% % 	\end{pmatrix}
% % 	\begin{pmatrix}
% % 		w_{1}(x,t) \\ w_{2}(x,t)
% % 	\end{pmatrix}
% % 	=
% % 	\begin{pmatrix}
% % 		-\left(w_{1}(x,t) + w_{2}(x,t)\right) \\
% % 		-\left(w_{1}(x,t) - w_{2}(x,t)\right)
% % 	\end{pmatrix},
% % \end{align*}
% % which means the following relation holds
% % \begin{align*}
% % 	\begin{pmatrix}
% % 		u_{1}(x,t) \\ u_{2}(x,t)
% % 	\end{pmatrix}
% % 	=
% % 	\begin{pmatrix}
% % 		-\left(w_{1}(x,t) + w_{2}(x,t)\right) \\
% % 		-\left(w_{1}(x,t) - w_{2}(x,t)\right)
% % 	\end{pmatrix}.
% % \end{align*}
% % Therefore, the solution for $u_{1}(x,t)$ is computed as follows
% % \begin{align*}
% % 	u_1(x, t)
% % 	=
% % 	\begin{cases}
% % 		-(-1+0)=1,   & x < t,      \\
% % 		-(0+0)=0,    & t < x < 3t, \\
% % 		-(0+(-1))=1, & x > 3t,
% % 	\end{cases}
% % \end{align*}
% % and, likewise, the solution for $u_{2}(x,t)$ is
% % \begin{align*}
% % 	u_2(x, t)
% % 	=
% % 	\begin{cases}
% % 		-(-1-0)=1,    & x < t,      \\
% % 		-(0-0)=0,     & t < x < 3t, \\
% % 		-(0-(-1))=-1, & x > 3t.
% % 	\end{cases}
% % \end{align*}
% % Finally, the solution to $u_{1}(x,t)$ and $u_{2}(x,t)$ read
% % \begin{equation*}
% % 	\therefore\quad
% % 	\boxed{
% % 		\begin{cases}
% % 			\begin{aligned}
% % 				u_1(x, t)
% % 				=
% % 				\begin{cases}
% % 					1, & x < t,      \\
% % 					0, & t < x < 3t, \\
% % 					1, & x > 3t,
% % 				\end{cases}
% % 			\end{aligned} \\
% % 			\begin{aligned}
% % 				u_2(x, t)
% % 				=
% % 				\begin{cases}
% % 					1,  & x < t,      \\
% % 					0,  & t < x < 3t, \\
% % 					-1, & x > 3t.
% % 				\end{cases}
% % 			\end{aligned}
% % 		\end{cases}
% % 	}
% % \end{equation*}
% % \clearpage
% % %------------------------------------------------------------------------------
% % \section{Remark on step-by-step derivation: Weak formulation of conservation laws}
% % \begin{example}
% % 	Consider the conservation law
% % 	\begin{align}\label{eq:conser}
% % 		u_{t} + f(u)_{x} = 0\quad\text{for}\quad(x,t)\in\mathbb{R}\times\mathbb{R}^{+},
% % 	\end{align}
% % 	with initial condition given by 
% % 	\begin{align*}
% % 		u(x,0) = u_{0}(x).
% % 	\end{align*}
% % 	Find the weak solution.
% % \end{example}
% % Approach: The weak formulation is obtained by multiplying both sides of \eqref{eq:conser}
% % with a test function 
% % $\varphi \in C^{1}_{0}(\mathbb{R}\times\mathbb{R})$,
% % and, then, integrating both sides over $\mathbb{R}\times\mathbb{R}^{+}$
% % \begin{align}\label{eq:conweak}
% % 	\int_{\mathbb{R}^{+}\times\mathbb{R}} \varphi \left(u_{t} + f(u)_{x}\right)dxdt = 0 .
% % \end{align}
% % The derivation goes as follows
% % \begin{align*}
% % 	\eqref{eq:conweak}
% % 	 & \Leftrightarrow
% % 	\int_{0}^{\mathscr{T}}\int_{\mathscr{D}} \varphi \left(u_{t} + f(u)_{x}\right)dxdt = 0 \\
% % 	 & \Leftrightarrow
% % 	\int_{0}^{\mathscr{T}}\int_{\mathscr{D}} \varphi \left(u_{t}\right)dxdt+ 
% % 	\int_{0}^{\mathscr{T}}\int_{\mathscr{D}} \varphi \left(f(u)_{x}\right)dxdt
% % 	= 0                                                                                    \\
% % 	 & \Leftrightarrow
% % 	\int_{\mathscr{D}}\int_{0}^{\mathscr{T}} \varphi \left(u_{t}\right)dtdx + 
% % 	\int_{0}^{\mathscr{T}}\int_{\mathscr{D}} \varphi \left(f(u)_{x}\right)dxdt
% % 	= 0                                                                                    \\
% % 	 & \Leftrightarrow
% % 	\left[
% % 	\int_{\mathscr{D}} \left. \varphi\, u \right|_{t=0}^{t=+\infty} dx -
% % 	\int_{\mathscr{D}}\int_{0}^{\mathscr{T}} \varphi_{t}\,u\, dtdx
% % 	\right]                                                                                \\
% % 	 & \quad +
% % 	\left[
% % 	\int_{0}^{\mathscr{T}} \left. \varphi\,f(u)\right|_{x=-\infty}^{x=+\infty} dxdt -
% % 	\int_{0}^{\mathscr{T}}\int_{\mathscr{D}} \varphi_{x}\, f(u)\,dxdt
% % 	\right]
% % 	= 0                                                                                    \\
% % 	 & \Leftrightarrow
% % 	\left[
% % 	\int_{\mathscr{D}} \left(
% % 	\varphi(x,\stackrel{t\rightarrow +\infty}{\mathscr{T}})
% % 	u(x,\stackrel{t\rightarrow+\infty}{\mathscr{T}})-\varphi(x,0)u(x,0)\right) dx  -
% % 	\int_{\mathscr{D}}\int_{0}^{\mathscr{T}} \varphi_{t}\,u\, dtdx
% % 	\right]                                                                                \\ 
% % 	 & \quad +
% % 	\left[
% % 	\int_{0}^{\mathscr{T}} \left(
% % 	\varphi(\stackrel{x \rightarrow +\infty}{\mathscr{D}},t)\,
% % 	f(u(\stackrel{x \rightarrow +\infty}{\mathscr{D}},t)) 
% % 	- \varphi(\underset{x \rightarrow -\infty}{\mathscr{D}},t)\,
% % 	f(u(\underset{x \rightarrow -\infty}{\mathscr{D}},t)) 
% % 	\right)dt \right.                                                                      \\
% % 	 & \qquad\qquad\qquad\qquad\qquad\qquad\qquad\qquad\qquad\qquad - 
% % 	\left. 
% % 	\int_{0}^{\mathscr{T}}\int_{\mathscr{D}} \varphi_{x}\, f(u)\,dxdt
% % 	\right]
% % 	= 0                                                                                    \\
% % 	 & \Leftrightarrow
% % 	\left[
% % 	\int_{\mathscr{D}} \left(
% % 	\cancelto{0}{\varphi(x,\stackrel{t\rightarrow +\infty}{\mathscr{T}})}
% % 	u(x,\stackrel{t\rightarrow+\infty}{\mathscr{T}})-\varphi(x,0)u(x,0)\right) dx  -
% % 	\int_{\mathscr{D}}\int_{0}^{\mathscr{T}} \varphi_{t}\,u\, dtdx
% % 	\right]                                                                                \\ 
% % 	 & \quad +
% % 	\left[
% % 	\int_{0}^{\mathscr{T}} \left(
% % 	\cancelto{0}{\varphi(\stackrel{x \rightarrow +\infty}{\mathscr{D}},t)}\,
% % 	f(u(\stackrel{x \rightarrow +\infty}{\mathscr{D}},t)) 
% % 	- \cancelto{0}{\varphi(\underset{x \rightarrow -\infty}{\mathscr{D}},t)}\,
% % 	f(u(\underset{x \rightarrow -\infty}{\mathscr{D}},t)) 
% % 	\right)dt \right.                                                                      \\
% % 	 & \qquad\qquad\qquad\qquad\qquad\qquad\qquad\qquad\qquad\qquad - 
% % 	\left. 
% % 	\int_{0}^{\mathscr{T}}\int_{\mathscr{D}} \varphi_{x}\, f(u)\,dxdt
% % 	\right]
% % 	= 0                                                                                    \\
% % 	%  & \Leftrightarrow
% % 	% \left[
% % 	% \int_{\mathscr{D}} \left(\cancelto{0}{
% % 	% 	\varphi(x,\mathscr{T}^{(t\rightarrow +\infty)})}
% % 	% u(x,\mathscr{T}^{(x\rightarrow+\infty)})-\varphi(x,0)u(x,0)\right) dx  -
% % 	% \int_{\mathscr{D}}\int_{0}^{\mathscr{T}} \varphi_{t}\,u\, dtdx
% % 	% \right]                                                                                \\ 
% % 	%  & \quad +
% % 	% \left[
% % 	% \int_{0}^{\mathscr{T}} \left(\cancelto{0}{
% % 	% 	\varphi(\mathscr{D}^{\left(x \rightarrow +\infty\right)},t)}\,
% % 	% f(u(\mathscr{D}^{\left(x \rightarrow +\infty\right)},t)) 
% % 	% - \cancelto{0}{
% % 	% 	\varphi(\mathscr{D}_{\left(x \rightarrow -\infty\right)},t)}\,
% % 	% f(u(\mathscr{D}_{\left(x \rightarrow -\infty\right)},t)) 
% % 	% \right)dt \right.                                                                      \\
% % 	%  & \qquad\qquad\qquad\qquad\qquad\qquad\qquad\qquad\qquad\qquad - 
% % 	% \left. 
% % 	% \int_{0}^{\mathscr{T}}\int_{\mathscr{D}} \varphi_{x}\, f(u)\,dxdt
% % 	% \right]
% % 	% = 0                                                                                    \\
% % 	 & \Leftrightarrow
% % 	-\int_{\mathscr{D}}\varphi(x,0)u(x,0)dx
% % 	-\int_{\mathscr{D}}\int_{0}^{\mathscr{T}}\varphi_{t}u\,dtdx
% % 	-\int_{0}^{\mathscr{T}}\int_{\mathscr{D}} \varphi_{x}\, f(u)\,dxdt
% % 	=0                                                                                     \\
% % 	 & \Leftrightarrow
% % 	-\int_{\mathscr{D}}\varphi(x,0)u_{0}(x)dx
% % 	-\int_{0}^{\mathscr{T}}\int_{\mathscr{D}}\varphi_{t}u\,dxdt
% % 	-\int_{0}^{\mathscr{T}}\int_{\mathscr{D}} \varphi_{x}\, f(u)\,dxdt
% % 	=0.
% % \end{align*}
% % Therefore, the weak formulation reads
% % \begin{equation*}
% % 	\therefore\quad
% % 	\boxed{
% % 	\int_{0}^{\mathscr{T}}\int_{\mathscr{D}}
% % 	\left(
% % 	u\varphi_{t} + f(u)\varphi_{x}\right) dxdt
% % 	+\int_{\mathscr{D}}\varphi(x,0)u_{0}(x)dx
% % 	=0.
% % 	}
% % \end{equation*}
% % Note in passing that the derivation above has taken
% % into consideration the swapping order of integral limits between time $t\in[0,+\infty]$
% % and space $x\in[-\infty,+\infty]$, which 
% % results in indifference as shown in 
% % \rFig{orderintegral}.
% % \inputfig{floats/orderintegral}{orderintegral}
% % \clearpage
% % %------------------------------------------------------------------------------
% % \section{Remark on step-by-step derivation: Weak solution of inviscid Burgers' equation in Riemann's problem}
% % \inputfig{floats/orderintegraluLR}{orderintegraluLR}
% % \inputfig{floats/orderintegraluR}{orderintegraluR}
% % \begin{example}
% % 	Given Burgers' equation with initial condition 
% % 	\begin{align*}
% % 		u_{0}(x) = 
% % 		\begin{cases}
% % 			u_{L},\quad x<0, \\
% % 			u_{R},\quad x>0.
% % 		\end{cases}
% % 	\end{align*}
% % 	Then
% % 	\begin{align*}
% % 		u(x,t) = 
% % 		\begin{cases}
% % 			u_{L},\quad x<st, \\
% % 			u_{R},\quad x>st,
% % 		\end{cases}
% % 	\end{align*}
% % 	is the weak solution to Burgers' equation.
% % \end{example}
% % \begin{proof}
% % 	RHS term
% % 	\begin{align*}
% % 		RHS & = -\int_{\mathscr{D}}\varphi(x,0)u_{0}(x)dx \\
% % 		    & = -u_{L}\int_{-\infty}^{0}\varphi(x,0)dx 
% % 		-u_{R}\int^{+\infty}_{0}\varphi(x,0)dx
% % 	\end{align*}
% % 	LHS term
% % 	\begin{align*}
% % 		LHS 
% % 		 & = 
% % 		\int_{0}^{\mathscr{T}}\int_{-\infty}^{st}\varphi_{t}u\,dxdt
% % 		+\int_{0}^{\mathscr{T}}\int_{st}^{+\infty}\varphi_{t}u\,dxdt                            \\
% % 		 & \quad +\int_{0}^{\mathscr{T}}\int_{-\infty}^{st}\varphi_{x}f(u)\,dxdt
% % 		+\int_{0}^{\mathscr{T}}\int_{st}^{+\infty}\varphi_{x}f(u)\,dxdt                         \\
% % 		 & =
% % 		u_{L}\int_{0}^{\mathscr{T}}\int_{-\infty}^{st}\varphi_{t}\,dxdt
% % 		+u_{R}\int_{0}^{\mathscr{T}}\int_{st}^{+\infty}\varphi_{t}\,dxdt                        \\
% % 		 & \quad + \frac{u_{L}^2}{2}\int_{0}^{\mathscr{T}}\int_{-\infty}^{st}\varphi_{x}\,dxdt
% % 		+\frac{u_{R}^2}{2}\int_{0}^{\mathscr{T}}\int_{st}^{+\infty}\varphi_{x}\,dxdt            \\
% % 		 & =
% % 		\left(
% % 		u_{L}\int_{-\infty}^{0}\int_{0}^{+\infty}\varphi_{t}\,dtdx
% % 		+u_{L}\int_{0}^{+\infty}\int_{x/s}^{+\infty}\varphi_{t}\,dtdx
% % 		\right)                                                                                 \\
% % 		 & \quad +u_{R}\int_{0}^{+\infty}\int_{0}^{x/s}\varphi_{t}\,dtdx                        \\
% % 		 & \quad + \frac{u_{L}^2}{2}\int_{0}^{\mathscr{T}}\int_{-\infty}^{st}\varphi_{x}\,dxdt
% % 		+ \frac{u_{R}^2}{2}\int_{0}^{\mathscr{T}}\int_{st}^{+\infty}\varphi_{x}\,dxdt           \\
% % 		 & =      
% % 		u_{L}\int_{-\infty}^{0} \left. \varphi\right|_{t=0}^{t=+\infty}dx
% % 		+u_{L}\int_{0}^{+\infty} \left. \varphi\right|_{t=x/s}^{t=+\infty}dx
% % 		+u_{R}\int_{0}^{+\infty} \left. \varphi\right|_{t=0}^{t=x/s}dx                          \\
% % 		 & \quad +\frac{u_{L}^2}{2}\int_{0}^{+\infty}\left. \varphi\right|_{x=-\infty}^{x=st}dt
% % 		+\frac{u_{R}^2}{2}\int_{0}^{+\infty}\left. \varphi\right|_{x=st}^{x=+\infty}dt          \\
% % 		 & =
% % 		u_{L}\int_{-\infty}^{0} \left(-\varphi(x,0)\right) dx 
% % 		+ u_{L}\int_{0}^{+\infty} \left(-\varphi(x,x/s)\right) dx                               \\
% % 		 & \quad + u_{R}\int_{0}^{+\infty}\left(\varphi(x,x/s)-\varphi(x,0)\right) dx           \\
% % 		 & \quad + \frac{u_{L}^2}{2}\int_{0}^{+\infty} \varphi(st,t)dt 
% % 		+ \frac{u_{R}^2}{2}\int_{0}^{+\infty} \left(-\varphi(st,t)\right)dt                     \\
% % 		 & = -u_{L}\int_{-\infty}^{0}\varphi(x,0)dx 
% % 		-u_{R}\int^{+\infty}_{0}\varphi(x,0)dx                                                  \\
% % 		 & \quad 
% % 		- \left(u_{L}-u_{R}\right)\int_{0}^{+\infty}\varphi(x,x/s)dx
% % 		+ \left(u_{L}-u_{R}\right)\int_{0}^{+\infty}\varphi(w,w/s)dw                            \\
% % 		 & = -u_{L}\int_{-\infty}^{0}\varphi(x,0)dx 
% % 		-u_{R}\int^{+\infty}_{0}\varphi(x,0)dx      
% % 		= RHS
% % 	\end{align*}
% % 	Note in passing that we have used the following relation
% % 	\begin{align*}
% % 		\frac{u_{L}^2}{2}\int_{0}^{+\infty} \varphi(st,t)dt 
% % 		+ \frac{u_{R}^2}{2}\int_{0}^{+\infty} \left(-\varphi(st,t)\right)dt       
% % 		 & = \frac{u_{L}^2-u_{R}^2}{2}\int_{0}^{+\infty} \varphi(st,t)\,dt \\
% % 		 & = \frac{u_{L}^2-u_{R}^2}{2s}\int_{0}^{+\infty}\varphi(w,w/s)dw  \\
% % 		 & = \left(u_{L}-u_{R}\right)\int_{0}^{+\infty}\varphi(w,w/s)dw
% % 	\end{align*}
% % 	where $w:=st$ and $dw=sdt$, and the RH condition for \emph{Burgers}' equation $s=1/2(u_{L}+u_{R})$.
% % \end{proof}
% % \clearpage
% % %------------------------------------------------------------------------------
% % \section{Consistency order}
% % % \begin{example}
% % % 	Order of consistency for Lax-Wendroff. 
% % % \end{example}
% % \begin{example}
% % 	Leapfrog scheme for the linear advection equation reads
% % 	\begin{align*}
% % 		% \text{Lax-Wendroff:}\qquad u_j^{n+1} & = u_j^n + \frac{a \Delta t}{2\Delta x} \left( u_{j-1}^n - u_{j+1}^n \right) + \frac{a^2 (\Delta t)^2}{2(\Delta x)^2} \left( u_{j+1}^n -2 u_j^n + u_{j-1}^n \right) \\
% % 		u_j^{n+1}  = u_j^{n-1} + \frac{a \Delta t}{\Delta x} \left( u_{j-1}^n - u_{j+1}^n \right).
% % 	\end{align*}
% % 	Show that the local consistency error $L_{\Delta t}$
% % 	\begin{align*}
% % 		% \text{Lax-Wendroff:}\qquad \|L_{\Delta t}(\cdot,t)\| & \leq C (\Delta t)^2\qquad \text{f\"ur } \Delta t \to 0 \\
% % 		\|L_{\Delta t}(\cdot,t)\| & \leq C (\Delta t)^2\qquad \text{for } \Delta t \to 0,
% % 	\end{align*}
% % 	hold for a linear solution and $\Delta t/\Delta x=const.$
% % \end{example}

% % % Zeigen Sie, dass im Fall glatter L\"osungen und $\Delta t/\Delta x=const.$  
% % % f\"ur die lokalen Abschneidefehler $L_{\Delta t}$
% % % gilt und somit beide Verfahren Konsistenzordnung zwei haben.



% % % Der lokale Abschneidefehler ist definiert als
% % % \begin{align*}
% % % 	L_{\Delta t} (x,t) = \frac{1}{\Delta t} \left[ u(x, t + \Delta t) - H_{\Delta t} (u(\cdot,t),x) \right],
% % % \end{align*}
% % % wobei $u(x,t)$ eine gen"ugend glatte L"osung von $u_t + a u_x = 0$ ist. \\

% % % Wir betrachten zuerst das \underline{Lax-Wendroff} Verfahren. F\"ur dieses ist der lokale Abschneidefehler gegeben durch
% % % \begin{align*}
% % % 	L_{\Delta t} (x,t) = \frac{1}{\Delta t} \Big[ u(x, t + \Delta t) & - \Big(u(x,t) + \frac{a \Delta t}{2\Delta x} \Big( u(x-\Delta x,t) - u(x+\Delta x,t) \Big)                   \\
% % % 	                                                                 & + \frac{a^2 (\Delta t)^2}{2(\Delta x)^2} \Big( u(x+\Delta x,t) -2u(x,t) + u(x-\Delta x,t) \Big) \Big) \Big].
% % % \end{align*}
% % % Folgend approximieren wir $u(x,t+\Delta t)$ durch eine Taylorentwicklung und verwenden dann, dass $u$ eine L\"osung der Advektionsgleichung ist und somit  $u_t = -au_x, u_{tt}=a^2u_{xx}$ gilt
% % % \begin{align*}
% % % 	u(x, t + \Delta t)
% % % 	 & = u(x,t) + u_t(x,t)\Delta t + u_{tt}(x,t) \frac{(\Delta t)^2}{2} + \mathcal{O}((\Delta t)^3)      \\
% % % 	 & = u(x,t) -au_x(x,t)\Delta t + a^2 u_{xx}(x,t) \frac{(\Delta t)^2}{2} + \mathcal{O}((\Delta t)^3).
% % % \end{align*}
% % % F\"ur $u(x\pm\Delta x,t)$ erhalten wir durch Taylorentwicklung
% % % \begin{align*}
% % % 	u(x+\Delta x,t)
% % % 	 & = u(x,t) + u_x(x,t)\Delta x + u_{xx}(x,t) \frac{(\Delta x)^2}{2} + \mathcal{O}((\Delta x)^3)  \\
% % % 	u(x-\Delta x,t)
% % % 	 & = u(x,t) - u_x(x,t)\Delta x + u_{xx}(x,t) \frac{(\Delta x)^2}{2} + \mathcal{O}((\Delta x)^3).
% % % \end{align*}
% % % Mit $\mathcal{O}((\Delta t)^n)=\mathcal{O}((\Delta x)^n)$, da $\Delta t/\Delta x=const.$ folgt f\"ur den lokalen Abschneidefehler
% % % \begin{align*}
% % % 	L_{\Delta t} (x,t) 
% % % 	 & = \frac{1}{\Delta t} \Big[ u(x,t) -au_x(x,t)\Delta t + a^2 u_{xx}(x,t) \frac{(\Delta t)^2}{2} + \mathcal{O}((\Delta t)^3) - \Big(u(x,t)                        \\
% % % 	 & + \frac{a \Delta t}{2\Delta x} \Big( \Big(u(x,t) - u_x(x,t)\Delta x + u_{xx}(x,t) \frac{(\Delta x)^2}{2} + \mathcal{O}((\Delta x)^3) \Big)                     \\
% % % 	 & \qquad\quad- \Big( u(x,t) + u_x(x,t)\Delta x + u_{xx}(x,t) \frac{(\Delta x)^2}{2} + \mathcal{O}((\Delta x)^3) \Big) \Big)                                      \\
% % % 	 & + \frac{a^2 (\Delta t)^2}{2(\Delta x)^2} \Big( \Big( u(x,t) + u_x(x,t)\Delta x + u_{xx}(x,t) \frac{(\Delta x)^2}{2} + \mathcal{O}((\Delta x)^3) \Big) -2u(x,t) \\
% % % 	 & \qquad\qquad +\Big(u(x,t) - u_x(x,t)\Delta x + u_{xx}(x,t) \frac{(\Delta x)^2}{2} + \mathcal{O}((\Delta x)^3) \Big) \Big) \Big) \Big]                          \\
% % % 	 & = \frac{1}{\Delta t} \Big[-au_x(x,t)\Delta t + a^2 u_{xx}(x,t) \frac{(\Delta t)^2}{2} + \mathcal{O}((\Delta t)^3) - \Big(                                      \\
% % % 	 & + \frac{a \Delta t}{2\Delta x} \Big( -2u_x(x,t)\Delta x + \mathcal{O}((\Delta x)^3) \Big)                                                                      \\
% % % 	 & + \frac{a^2 (\Delta t)^2}{2(\Delta x)^2} \Big( 2u_{xx}(x,t) \frac{(\Delta x)^2}{2} + \mathcal{O}((\Delta x)^3) \Big) \Big) \Big]                               \\
% % % 	 & = \frac{1}{\Delta t} \Big[-au_x(x,t)\Delta t + a^2 u_{xx}(x,t) \frac{(\Delta t)^2}{2} + \mathcal{O}((\Delta t)^3)                                              \\
% % % 	 & \quad - \Big( -au_x(x,t)\Delta t + a^2 u_{xx}(x,t) \frac{(\Delta t)^2}{2} + \mathcal{O}((\Delta x)^3) \Big) \Big) \Big]                                        \\
% % % 	 & = \frac{1}{\Delta t} \mathcal{O}((\Delta t)^3) = \mathcal{O}((\Delta t)^2)
% % % \end{align*}
% % % Somit gilt f\"ur den lokalen Abschneidefehler 
% % % \begin{align*}
% % % 	\|L_{\Delta t}(\cdot,t)\| & \leq C (\Delta t)^2\qquad \text{f\"ur } \Delta t \to 0.
% % % \end{align*}
% % \inputfig{floats/leapfrog}{leapfrog}
% % Consider the local truncation error
% % \begin{align*}
% % 	L_{\Delta t} (x,t) = \frac{1}{\Delta t} \left[ u(x, t + \Delta t) - H_{\Delta t} (u(\cdot,t),x) \right],
% % \end{align*}
% % which reads for the considered problem
% % \begin{align*}
% % 	L_{\Delta t} (x,t) = \frac{1}{\Delta t} \Big[ u(x, t + \Delta t)  - \Big(u(x,t-\Delta t) + \frac{a \Delta t}{\Delta x} \left( u(x-\Delta x,t) - u(x+\Delta x,t) \right) \Big) \Big].
% % \end{align*}
% % % Analog zu $u(x,t+\Delta t)$ oben erhalten wir f\"ur $u(x,t-\Delta t)$
% % Applying Taylor's expansion for $u(x,t+\Delta t)$
% % \begin{align*}
% % 	u(x, t + \Delta t)
% % 	 & = u(x,t) + u_t(x,t)\Delta t + u_{tt}(x,t) \frac{(\Delta t)^2}{2} + \mathcal{O}((\Delta t)^3)      \\
% % 	 & = u(x,t) -au_x(x,t)\Delta t + a^2 u_{xx}(x,t) \frac{(\Delta t)^2}{2} + \mathcal{O}((\Delta t)^3).
% % \end{align*}
% % Applying Taylor's expansion for $u(x,t-\Delta t)$
% % \begin{align*}
% % 	u(x, t - \Delta t)
% % 	 & = u(x,t) - u_t(x,t)\Delta t + u_{tt}(x,t) \frac{(\Delta t)^2}{2} + \mathcal{O}((\Delta t)^3)       \\
% % 	 & = u(x,t) + au_x(x,t)\Delta t + a^2 u_{xx}(x,t) \frac{(\Delta t)^2}{2} + \mathcal{O}((\Delta t)^3).
% % \end{align*}
% % Furthermore, applying Taylor's expansion for $u(x+\Delta x,t)$ and $u(x-\Delta x,t)$
% % \begin{align*}
% % 	u(x+\Delta x,t)
% % 	 & = u(x,t) + u_x(x,t)\Delta x + u_{xx}(x,t) \frac{(\Delta x)^2}{2} + \mathcal{O}((\Delta x)^3)  \\
% % 	u(x-\Delta x,t)
% % 	 & = u(x,t) - u_x(x,t)\Delta x + u_{xx}(x,t) \frac{(\Delta x)^2}{2} + \mathcal{O}((\Delta x)^3).
% % \end{align*}
% % % Durch einsetzen der Taylorreihen erhalten wir f\"ur den lokalen Abschneidefehler
% % Local truncation error 
% % \begin{align*}
% % 	L_{\Delta t} (x,t) = \frac{1}{\Delta t} & \Big[ u(x,t) -au_x(x,t)\Delta t + a^2 u_{xx}(x,t) \frac{(\Delta t)^2}{2} + \mathcal{O}((\Delta t)^3)                                                  \\ &- \Big( u(x,t) + au_x(x,t)\Delta t + a^2 u_{xx}(x,t) \frac{(\Delta t)^2}{2} + \mathcal{O}((\Delta t)^3)\\
% % 	                                        & + \frac{a \Delta t}{\Delta x} \Big( \Big(  u(x,t) - u_x(x,t)\Delta x + u_{xx}(x,t) \frac{(\Delta x)^2}{2} + \mathcal{O}((\Delta x)^3) \Big)           \\
% % 	                                        & \qquad\quad - \Big( u(x,t) + u_x(x,t)\Delta x + u_{xx}(x,t) \frac{(\Delta x)^2}{2} + \mathcal{O}((\Delta x)^3) \Big) \Big) \Big) \Big]                \\
% % 	= \frac{1}{\Delta t}                    & \Big[ -2au_x(x,t)\Delta t + \mathcal{O}((\Delta t)^3) - \frac{a \Delta t}{\Delta x} \Big( - 2u_x(x,t)\Delta x + \mathcal{O}((\Delta x)^3) \Big) \Big] \\
% % 	                                        & = \frac{1}{\Delta t} \mathcal{O}((\Delta t)^3)                                                                                                        \\
% % 	                                        & = \mathcal{O}((\Delta t)^2).
% % \end{align*}
% % % Somit gilt auch f\"ur das Leapfrog Verfahren 
% % Therefore, order of consistency for Leapfrog scheme is of order 2
% % \begin{align*}
% % 	\|L_{\Delta t}(\cdot,t)\| & \leq C (\Delta t)^2\qquad \text{for } \Delta t \to 0.
% % \end{align*}
% % Note in passing that $u_{t} = -au_{x}$ and $u_{tt} = a^2u_{xx}$ are from the linear advection equation.

% % \section{Breaking time for an arbitrary flux function}
% % Consider arbitrary convex scalar equation of conservation laws 
% % \begin{equation}
% % 	u_{t} + f(u)_{x} = 0, 
% % \end{equation}
% % where $f(u)$ is convex, i.e. $f''(u) > 0$. Then, smooth flux function $f(u)$ leads to
% % \begin{equation}
% % 	u_{t} + f'(u) u_{x} = 0.
% % \end{equation}
% % Arbitrarily consider one characteristic line passing the point with initial condition in space and time $x_{0}(t=0)$. 
% % % Likewise, the neighborhood of this point reads $x_{0} + \delta x$.\\
% % The first characteristic line passing point $(x_{0_{1}}, 0)$ reads
% % \begin{equation}\label{eq:firstline}
% % 	x =  x_{0_{1}} + f'(u_{0}(x_{0_{1}}))\,t                                                              
% % \end{equation}
% % The second characteristic line passing point $(x_{0_{2}}, 0)$  reads
% % \begin{equation}\label{eq:secondline}
% % 	x =  x_{0_{2}} + f'(u_{0}(x_{0_{2}}))\,t                                                              
% % \end{equation}
% % % \begin{align}\label{eq:secondline}
% % % 	x & = (x_{0} + \delta x) + f'(u_{0}(x_{0} + \delta x))\,t                                       \notag \\
% % % 	  & = (x_{0} + \delta x) + f'(u_{0}(x_{0}) + u'_{0}(x_{0})\delta x + \mathcal{O}(\delta x^2))t \notag  \\
% % % 	  & = (x_{0} + \delta x) + f'(u_{0}(x_{0}) + u'_{0}(x_{0})\delta x + \mathcal{O}(\delta x^2))t
% % % \end{align}
% % Equalizing the two characteristic lines from \eqref{eq:firstline} and \eqref{eq:secondline} yields
% % % \begin{align*}
% % % 	x_{0} + f'(u_{0}(x_{0})) \, t = 
% % % 	(x_{0} + \delta x) + f'(u_{0}(x_{0}) + u'_{0}(x_{0})\delta x + \mathcal{O}(\delta x^2)) \, t
% % % \end{align*}
% % \begin{align}
% % 	x_{0_{1}} + f'(u_{0}(x_{0_{1}}))\,t = x_{0_{2}} + f'(u_{0}(x_{0_{2}}))\,t  
% % \end{align}
% % which leads to
% % \begin{equation}
% % 	t=-\frac{x_{0_{1}} - x_{0_{2}}}{f'(u_{0}(x_{0_{1}})-f'(u_{0}(x_{0_{2}})}
% % \end{equation}
% % The breaking time requires the following condition
% % \begin{align}
% % 	t & =-\frac{1}{\displaystyle \frac{f'(u_{0}(x_{0_{1}})-f'(u_{0}(x_{0_{2}})}{x_{0_{1}} - x_{0_{2}}}} \notag \\
% % 	  & =-\frac{1}{\displaystyle \frac{d}{dx}f'(u_{0}(x))}
% % \end{align}
% % The \emph{first} breaking time yields
% % \begin{equation}
% % 	\therefore\quad
% % 	\boxed{
% % 		T_{b}=-\frac{1}{\displaystyle \min_{x\in\mathbb{R}} \frac{d}{dx}f'(u_{0}(x))}
% % 	}
% % \end{equation}
% % \clearpage
% % \begin{example}
% % 	Compute the breaking time of Burgers' equation given different initial conditions as follows.
% % \end{example}
% % \inputfig{floats/characsin}{characsin}
% % \clearpage
% % %------------------------------------------------------------------------------
% % \section{Riemann's problem}
% % \begin{example}
% % 	Non-zero initial condition applied on $u_{L}$ and $u_{R}$.
% % \end{example}
% % Consider the two cases as follows
% % \begin{align}
% % 	\begin{cases}
% % 		\begin{aligned}
% % 			u_{L} = \alpha, \qquad x<0, \\
% % 			u_{R} = \beta, \qquad x>0,   
% % 		\end{aligned}
% % 	\end{cases}
% % \end{align}
% % and
% % \begin{align}
% % 	\begin{cases}
% % 		\begin{aligned}
% % 			u_{L} = \beta, \qquad x<0, \\
% % 			u_{R} = \alpha, \qquad x>0,   
% % 		\end{aligned}
% % 	\end{cases}
% % \end{align}
% % where $\alpha,\beta \in \mathbb{R}\backslash\left\{0\right\}$.
% % \inputfig{floats/charac12182181}{charac12182181}
% % \clearpage
% % \begin{example}
% % 	Zero initial condition applied either on $u_{L}$ or $u_{R}$.
% % \end{example}
% % Consider the two cases as follows
% % \begin{align}
% % 	\begin{cases}
% % 		\begin{aligned}
% % 			u_{L} = \alpha, \qquad x<0, \\
% % 			u_{R} = 0, \qquad x>0,   
% % 		\end{aligned}
% % 	\end{cases}
% % \end{align}
% % and
% % \begin{align}
% % 	\begin{cases}
% % 		\begin{aligned}
% % 			u_{L} = 0, \qquad x<0, \\
% % 			u_{R} = \alpha, \qquad x>0,   
% % 		\end{aligned}
% % 	\end{cases}
% % \end{align}
% % where $\alpha \in \mathbb{R}\backslash\left\{0\right\}$.
% % \inputfig{floats/charac02082080}{charac02082080}
% % % %------------------------------------------------------------------------------
% % % \section{Riemann's problem}
% % % %------------------------------------------------------------------------------
% % % \section{Krushkov}
% % \clearpage
% % %------------------------------------------------------------------------------
% % \section{Total variation (TV)}
% % (Useful later on for structure of FVM, i.e. Total-variation-diminishing (TVD))
% % \begin{example}
% % 	Let $f$ be defined as follows
% % 	\begin{align}
% % 		f:
% % 		\begin{cases}
% % 			\mathbb{R} \rightarrow \mathbb{R}, \\
% % 			x \mapsto
% % 			f(x) := 
% % 			\begin{cases}
% % 				x, \quad   & \quad  0<x<1,                 \\
% % 				2, \quad   & \quad  1\leq x < 2,           \\
% % 				1+x, \quad & \quad  2\leq x < 4,           \\
% % 				0, \quad   & \quad \text{everywhere else}.
% % 			\end{cases}
% % 		\end{cases}
% % 	\end{align}
% % 	Find the total variation of $f$.

% % \end{example}
% % \inputfig{floats/totalvariation}{totalvariation}

% % The total variation of $f$ is computed as follows
% % \begin{equation}
% % 	\begin{aligned}
% % 		TV(f)   = \int_{0}^1|f^{'}|dx                                                                
% % 		 & + \lim_{h\to 0} \int_{1-h}^1\frac{|f(x+h) - f(x)|}{h}dx
% % 		+\int_{1}^2|f^{'}|dx                                                                    \\
% % 		 & + \lim_{h\to 0} \int_{2-h}^{2}\frac{|f(x+h) - f(x)|}{h}dx  +  \int_{2}^{4}|f^{'}|dx  \\
% % 		 & + \lim_{h\to 0} \int_{4-h}^{4}\frac{|f(x+h) - f(x)|}{h}dx                           
% % 		= 1 + 1 + 0 + 1 + 2 + 5  = 10 \notag
% % 	\end{aligned}
% % \end{equation}
% % \begin{equation}
% % 	\therefore\quad
% % 	\boxed{
% % 		TV(f) = 10.
% % 	}
% % \end{equation}
% % %------------------------------------------------------------------------------
% % \begin{example}
% % 	Let $f$ be defined as follows
% % 	\begin{align}
% % 		f:
% % 		\begin{cases}
% % 			\mathbb{R} \rightarrow \mathbb{R}, \\
% % 			x \mapsto
% % 			f(x) := 
% % 			\begin{cases}
% % 				-x+1, \quad & \quad  0<x<1,                 \\
% % 				2, \quad    & \quad  1\leq x < 2,           \\
% % 				-x+7, \quad & \quad  2\leq x < 4,           \\
% % 				0, \quad    & \quad \text{everywhere else}.
% % 			\end{cases}
% % 		\end{cases}
% % 	\end{align}
% % 	Find the total variation of $f$.
% % \end{example}
% % \inputfig{floats/totalvariation_inv}{totalvariation_inv}

% % The total variation of $f$ is computed as follows
% % \begin{equation}
% % 	\begin{aligned}
% % 		TV(f)  = \dots
% % 		%  & = \int_{0}^1|f^{'}|dx                                                               \\
% % 		%  & + \lim_{h\to 0} \int_{1-h}^1\frac{|f(x+h) - f(x)|}{h}dx
% % 		% +\int_{1}^2|f^{'}|dx                                                                   \\
% % 		%  & + \lim_{h\to 0} \int_{2-h}^{2}\frac{|f(x+h) - f(x)|}{h}dx  +  \int_{2}^{4}|f^{'}|dx \\
% % 		%  & + \lim_{h\to 0} \int_{4-h}^{4}\frac{|f(x+h) - f(x)|}{h}dx                           \\
% % 		%  & = 1 + 1 + 2 + 3 + 2 + 3 = 12.
% % 	\end{aligned}
% % \end{equation}
% % \newline
% % \newline
% % \newline
% % \newline
% % \newline
% % \newline
% % \newline
% % \newline
% % \newline
% % \begin{equation}
% % 	\therefore\quad
% % 	\boxed{
% % 		TV(f) = 12
% % 	}
% % \end{equation}
% % %------------------------------------------------------------------------------
% % \begin{example}
% % 	Let $f$ be defined as follows
% % 	\begin{align}
% % 		f:
% % 		\begin{cases}
% % 			\mathbb{R} \rightarrow \mathbb{R}, \\
% % 			x \mapsto
% % 			f(x) := 
% % 			\begin{cases}
% % 				-x+1, \quad           & \quad  0<x<1,                 \\
% % 				2, \quad              & \quad  1\leq x < 2,           \\
% % 				\sin(-\pi x)+4, \quad & \quad  2\leq x < 4,           \\
% % 				0, \quad              & \quad \text{everywhere else}.
% % 			\end{cases}
% % 		\end{cases}
% % 	\end{align}
% % 	Find the total variation of $f$.
% % \end{example}
% % \inputfig{floats/totalvariation_sin}{totalvariation_sin}

% % The total variation of $f$ is computed as follows
% % \begin{equation}
% % 	\begin{aligned}
% % 		TV(f)  = \dots
% % 		% 	 & = \int_{0}^1|f^{'}|dx                                                               \\
% % 		% 	 & + \lim_{h\to 0} \int_{1-h}^1\frac{|f(x+h) - f(x)|}{h}dx
% % 		% 	+\int_{1}^2|f^{'}|dx                                                                   \\
% % 		% 	 & + \lim_{h\to 0} \int_{2-h}^{2}\frac{|f(x+h) - f(x)|}{h}dx  +  \int_{2}^{4}|f^{'}|dx \\
% % 		% 	 & + \lim_{h\to 0} \int_{4-h}^{4}\frac{|f(x+h) - f(x)|}{h}dx                           \\
% % 		% 	 & = 1 + 1 + 2 + 0 + 2 + 1 + 2 + 1 + 4 = 14.
% % 	\end{aligned}
% % \end{equation}
% % \newline
% % \newline
% % \newline
% % \newline
% % \newline
% % \newline
% % \newline
% % \newline
% % \newline
% % \begin{equation}
% % 	\therefore\quad
% % 	\boxed{
% % 		TV(f) = 14
% % 	}
% % \end{equation}
% % %------------------------------------------------------------------------------
% % % \section{Domain of dependence}
% % \clearpage
% % %------------------------------------------------------------------------------
% % \section{Manipulating conservation laws}
% % \emph{Manipulation of conservation laws} by transforming the differential form into another differential equation, 
% % which appears lately as equivalently as the original one, \textbf{may not} result in an equivalent differential
% % equation with respect to weak solutions.
% % \begin{example}
% % 	Manipulating conservation laws
% % \end{example}
% % Consider the Burgers' equation
% % \begin{equation}\label{eq:Burgers}
% % 	u_{t} + \left(\frac{1}{2}u^2\right)_{x} = 0.
% % \end{equation}
% % \emph{Manipulation} of this PDE by multiplying \eqref{eq:Burgers} by $2u$ we obtain
% % \begin{equation}\label{eq:Burgers1}
% % 	2uu_{t} + 2u\left(\frac{1}{2}u^2\right)_{x} = 0,
% % \end{equation}
% % which can be recast into
% % \begin{equation}\label{eq:Burgers2}
% % 	w_{t} + \left(\frac{2}{3}w^{\frac{3}{2}}\right)_{x} = 0
% % \end{equation}
% % where $w:=u^2$, and the flux function become $\displaystyle f(w) = \frac{2}{3}w^{\frac{3}{2}}$. 
% % The solutions to \eqref{eq:Burgers} and \eqref{eq:Burgers2}, respectively, read
% % \begin{align}
% % 	u(x,t) & = g_{0}(x-ut),       \\
% % 	w(x,t) & = g_{0}(x-w^{1/2}t),
% % \end{align}
% % which are precisely the same. Note in passing that $w^{1/2} = u$. Although \eqref{eq:Burgers} and \eqref{eq:Burgers2}
% % have the same solution, their weak solutions are different from each other,
% % i.e. by considering the \emph{Riemann}'s problem with $u_{L}>u_{R}$
% % together with checking the \emph{Rankine-Hugoniot} condition, we obtain unique weak solution 
% % with a shock travelling at speeds computed as follows
% % \begin{equation}
% % 	s_{\text{Burgers}} = \frac{[f(u)]}{[u]} = \dots
% % \end{equation}
% % Likewise, for the \emph{manipulated Burgers} we obtain
% % \begin{equation}
% % 	s_{\text{manipulated Burgers}} = \frac{[f(w)]}{[w]} = \dots
% % \end{equation}
% % \clearpage
% % %------------------------------------------------------------------------------
% % \section{Linear hyperbolic systems}
% % \begin{example}
% % 	Consider the natural 1D second order wave equation 
% % 	\begin{equation}
% % 		u_{tt} = \alpha^2 u_{xx},\quad x\in\mathbb{R}, 
% % 	\end{equation}
% % 	with initial condition
% % 	\begin{align}
% % 		\begin{cases}
% % 			\begin{aligned}
% % 				u(x,0)     & = u_{0}(x), \\
% % 				u_{t}(x,0) & = u_{1}(x). \\
% % 			\end{aligned}
% % 		\end{cases}
% % 	\end{align}
% % \end{example}
% % Approach: By introducing 
% % \begin{align}
% % 	\begin{cases}
% % 		\begin{aligned}
% % 			v & = u_{x}, \\
% % 			w & = u_{t},
% % 		\end{aligned}
% % 	\end{cases}
% % \end{align}
% % one obtains 
% % \begin{align}
% % 	\begin{pmatrix} v\\ w\end{pmatrix}_{t}
% % 	+
% % 	\begin{pmatrix} -w \\ -\alpha^2 v \end{pmatrix}_{x}
% % 	= 0,
% % \end{align}
% % which can be recast into
% % \begin{align}
% % 	\begin{pmatrix} v\\ w\end{pmatrix}_{t}
% % 	+
% % 	\begin{pmatrix} 0 & -1 \\ -\alpha^2 & 0 \end{pmatrix}
% % 	\begin{pmatrix} v\\ w\end{pmatrix}_{x}
% % 	= 0.
% % \end{align}
% % The initial condition becomes
% % \begin{align}
% % 	\begin{cases}
% % 		\begin{aligned}
% % 			v(x,0) & = u'_{0}(x), \\
% % 			w(x,0) & = u_{1}(x).  \\
% % 		\end{aligned}
% % 	\end{cases}
% % \end{align}
% % %------------------------------------------------------------------------------
% % \clearpage
% % \begin{example}
% % 	Consider the \textbf{linearized shallow water equations}
% % 	\begin{align}
% % 		\begin{pmatrix} u\\ \varphi\end{pmatrix}_{t}
% % 		+
% % 		\begin{pmatrix} \bar{u} & 1 \\ \bar{\varphi} & \bar{u} \end{pmatrix}
% % 		\begin{pmatrix} u \\ \varphi\end{pmatrix}_{x}
% % 		= 0,
% % 	\end{align}
% % 	where $\bar{u}, \bar{\varphi} \in \mathbb{R}$ are given, 
% % 	and sufficient initial conditions for $u$ and $\varphi$ are provided.
% % \end{example}

% % % Integral form
% % % \begin{equation}
% % % 	\frac{d}{dt}\int_{\Omega}u\,d\Omega 
% % % 	= -\int_{\partial\Omega}f(u)\cdot n\,d\partial\Omega
% % % \end{equation}
% % % Differential form
% % % \begin{equation}
% % % 	u_{t} + \nabla\cdot f(u) = 0 
% % % \end{equation}
% % \emph{Master balance principle}: due to the fact that the mathematical structure of the fundamental balance relations,
% % namely of mass, linear momentum, moment of momentum, energy and entropy, is in principle identical, they can be 
% % formulated within the concise shape of a master balance. To begin with, let $\Psi$ and $\BPsi$ be volume specific
% % scalar and vector value densities of a physical quantity to be balanced, respectively. Then, the general balance relations take 
% % the global \textbf{integral form} as follows 
% % \begin{align}
% % 	\frac{d}{dt}\int_{\Omega} \Psi \,d\Omega  & = 
% % 	-\int_{\partial\Omega}\Bphi\cdot\Bn\,d\partial\Omega 
% % 	+ \int_{\Omega}\sigma\,d\Omega
% % 	+ \int_{\Omega}\hat{\Psi}\,d\Omega             \\
% % 	\frac{d}{dt}\int_{\Omega} \BPsi \,d\Omega & = 
% % 	-\int_{\partial\Omega}\BPhi\cdot\Bn\,d\partial\Omega 
% % 	+ \int_{\Omega}\Bsigma\,d\Omega
% % 	+ \int_{\Omega}\hat{\BPsi}\,d\Omega
% % \end{align}
% % where $(\Bphi\cdot\Bn)$ and $(\BPhi\cdot\Bn)$ describe action at the vicinity; $\sigma$ and 
% % $\Bsigma$ present for action from a distance; and $\hat{\Psi}$ and $\hat{\BPsi}$ for production or nucleation.
% % \begin{example}
% % 	Conservation of mass: A derivation from integral to differential form.
% % \end{example}
% % \inputfig{floats/massx1x2}{massx1x2}
% % Let $x$ represent the distance along the tube and let $\rho(x,t)$ be the mass density of the 
% % fluid at point $x$ and time $t$. Then, the total mass $M$ in the section 
% % $[x_{1},x_{2}]$ at time $t$ is defined as follows
% % \begin{equation}
% % 	M := \int_{x_{1}}^{x_{2}} \rho(x,t)\,dx.
% % \end{equation}
% % Assume that the walls of the tube are impermeable and mass is neither created nor destroyed, 
% % this total mass $M$ in this interval $[x_1,x_2]$ varies over time only due to the fluid flowing
% % across at the boundaries $x=x_{1}$ and $x=x_{2}$. 
% % % Besides, the flow rate (also called \emph{rate of flow}, or \emph{flux}) is denoted as $f(\rho)$.
% % Now let $v(x,t)$ be the measured velocity of flow at point $x$ and time $t$.
% % Then, the flow rate (also called \emph{rate of flow}, or \emph{flux})
% % passing the end points of the interval $[x_{1},x_{2}]$ at time $t$ is given by
% % \begin{equation}
% % 	\rho(x_{1},t)v(x_{1},t)-\rho(x_{2},t)v(x_{2},t).
% % \end{equation}
% % The rate of change of mass in $[x_{1},x_{2}]$ given by the difference of fluxes at 
% % $x_1$ and $x_2$ flow rate has to be equal with the time rate of change of total mass
% % \begin{equation}\label{eq:relation0}
% % 	\boxed{
% % 		\frac{d}{dt}\int_{x_{1}}^{x_{2}} \rho(x,t)\,dx 
% % 		% \stackrel{!}{=} f(\rho(x_{1},t))-f(\rho(x_{2},t))
% % 		= \rho(x_{1},t)v(x_{1},t)-\rho(x_{2},t)v(x_{2},t)
% % 	}
% % \end{equation}
% % which is the \textbf{integral form} of the conservation law. Next, note that the following relation holds
% % \begin{equation}\label{eq:relation1}
% % 	\int_{x_{1}}^{x_{2}} \frac{\partial}{\partial x}f(\rho)\,dx 
% % 	= f(\rho(x_{2},t))-f(\rho(x_{1},t)).
% % \end{equation}
% % Substitution of \eqref{eq:relation1} into \eqref{eq:relation0} leads to
% % \begin{equation}\label{eq:relation2}
% % 	\frac{d}{dt}\int_{x_{1}}^{x_{2}} \rho(x,t)\,dx 
% % 	= -\int_{x_{1}}^{x_{2}}\frac{\partial}{\partial x}f(\rho)\,dx.
% % \end{equation}
% % If $\rho(x,t)$ is sufficiently smooth, time derivative on the LHS of \eqref{eq:relation2}
% % can be brought inside the integral. Then, an arrangement of \eqref{eq:relation2} yields
% % \begin{equation}\label{eq:relation3}
% % 	\int_{x_{1}}^{x_{2}}
% % 	\left(\frac{\partial}{\partial t}\rho(x,t) + \frac{\partial}{\partial x}f(\rho)\right)\,dx 
% % 	= 0.
% % \end{equation}
% % Since \eqref{eq:relation3} holds for any choice of $x_{1}$ and $x_{2}$, the integrant must be vanished
% % everywhere. Therefore, one obtains the \textbf{differential form} of the conservation law
% % \begin{equation}\label{eq:deri}
% % 	\therefore \quad
% % 	\boxed{\rho_{t} + f(\rho)_{x} = 0}
% % \end{equation}
% % % \begin{enumerate}
% % % 	\item gas dynamics
% % % 	\item oil-water mixtures
% % % 	\item plasmas
% % % 	\item shallow water flows
% % % 	\item meteology
% % % 	\item traffic
% % % 	\item ...
% % % \end{enumerate}
% % % \begin{definition}
% % % 	Cauchy problem
% % % \end{definition}
% % % Cauchy problem\\
% % % Riemann problem\\
% % % \begin{equation}
% % % 	u_{t} + f(u)_{x} = 0 
% % % \end{equation}
% % \begin{example}
% % 	Different definitions of flux function yields different applications
% % \end{example}
% % \begin{enumerate}
% % 	\item Advection equation: \eqref{eq:deri} with flux function
% % 	      $\displaystyle  f(\rho) =  a\rho, \ a\in \mathbb{R^+}$, taking form
% % 	      \begin{equation}
% % 		      \boxed{
% % 			      \rho_{t} + a\rho_{x} = 0
% % 		      }
% % 	      \end{equation}
% % 	\item \emph{Inviscid} Burgers' equation: \eqref{eq:deri} with flux function
% % 	      $\displaystyle  f(\rho) =  \frac{1}{2}\rho^2$, yielding
% % 	      \begin{equation}
% % 		      \boxed{
% % 			      \rho_{t} + \rho\rho_{x} = 0
% % 		      }
% % 	      \end{equation}
% % 	      which is used to illustrate the distortion of waveform in simple waves.
% % 	      Meanwhile, \emph{viscous} Burgers' equation is a scalar parabolic PDE, taking the form
% % 	      $$\rho_t + \rho\rho_{x} = \varepsilon\rho_{xx}$$
% % 	      which is the simplest differential model for a fluid flow.
% % 	\item Lighthill-Whitham-Richards (LWR) equation: \eqref{eq:deri} with flux function
% % 	      $$\displaystyle  f(\rho) =  u_{max}\,\rho\left(1-\frac{\rho}{\rho_{max}}\right)$$
% % 	      which is used to model traffic flow; $u_{max}$ is the given maximal speed of vehicle to be allowed, 
% % 	      $\rho_{max}$ the given maximal density of vehicle; taking the form
% % 	      \begin{equation}
% % 		      \rho_{t} + \left(u_{max}-\frac{2u_{max}}{\rho_{max}}\rho\right)\rho_{x} = 0
% % 	      \end{equation}
% % 	\item (Typical) Buckley-Leverett petroleum equation: \eqref{eq:deri} with flux function
% % 	      $$\displaystyle f(\rho) =  \frac{\rho^2}{\rho^2 + (1-\rho)^2\mu_{\text{water}}/\mu_{\text{oil}}}$$
% % 	      which is a one-dimensional model for a two-phase flow; $\rho$ stands for water saturation
% % 	      and takes the value in the interval $[0,1]$.
% % 	\item \emph{Euler} equations of gas dynamics in 1D
% % 	      %   \begin{align}
% % 	      %       \begin{cases}
% % 	      % 	      \begin{aligned}
% % 	      % 		      \rho_{t}     & + (\rho v)_x & = 0 \\
% % 	      % 		      (\rho v)_{t} & + (\rho v^2 + p)_x & = 0 \\
% % 	      % 		      E_{t}        & + (v(E+p))_x & = 0
% % 	      % 	      \end{aligned}
% % 	      %       \end{cases}
% % 	      %   \end{align}
% % 	      \begin{align}
% % 		      \begin{pmatrix} \rho\\\rho v \\ E \end{pmatrix}_{t}
% % 		      +
% % 		      \begin{pmatrix} \rho v\\\rho v^2 + p \\ v(E+p) \end{pmatrix}_{x}
% % 		      =
% % 		      \begin{pmatrix} 0\\ 0 \\ 0 \end{pmatrix}
% % 	      \end{align}
% % 	      $\rightarrow$ The pressure $p$ should be specified as a given function of 
% % 	      mass density $\rho$, linear momentum $\rho v$ and/or energy $E$ in order to 
% % 	      fulfill the \emph{closure} problem.
% % 	      Such additional equation is called \emph{equation of state} (normally in fluid), or 
% % 	      \emph{constitutive equation} (normally in solid). 
% % 	\item \emph{Euler} equations of gas dynamics in 2D
% % 	      \begin{align}
% % 		      \begin{pmatrix} \rho\\ \rho u\\ \rho v \\ E \end{pmatrix}_{t}
% % 		      +
% % 		      \begin{pmatrix} \rho u\\ \rho u^2 + p \\ \rho uv \\ u(E+p) \end{pmatrix}_{x}
% % 		      + 
% % 		      \begin{pmatrix} \rho v\\ \rho uv \\ \rho v^2 + p \\ v(E+p) \end{pmatrix}_{y}
% % 		      =
% % 		      \begin{pmatrix} 0 \\ 0 \\ 0 \\ 0 \end{pmatrix}
% % 	      \end{align}
% % \end{enumerate}
% % % \begin{recall}
% % % 	Convexity
% % % \end{recall}
% % \begin{definition}
% % 	Cauchy problem and Riemann problem
% % \end{definition}
% % Let $u$ be a conserved unknown quantity to be modelled and defined as follows
% % \begin{align}\label{eq:conservedquantity}
% % 	u:
% % 	\begin{cases}
% % 		\mathbb{R}\times\mathbb{R}^+ \rightarrow \mathbb{R}, \\
% % 		(x,t) \mapsto u(x,t).
% % 	\end{cases}
% % \end{align}
% % \emph{Cauchy's problem} is simply the pure initial value problem (IVP), e.g. 
% % find a function $u$ holding \eqref{eq:conservedquantity} that is a solution of
% % $\eqref{eq:cauchy}_1$ satisfying the initial condition $\eqref{eq:cauchy}_2$:
% % \begin{equation}\label{eq:cauchy}
% % 	\boxed{
% % 		\begin{aligned}
% % 			\partial_{t}u + \partial_{x}f(u) & = 0,        \\
% % 			u(x,0)                           & = u_{0}(x).
% % 		\end{aligned}
% % 	}
% % \end{equation}
% % \emph{Riemann's problem} is simply the conservation law together with particular initial data
% % consisting of two constant states separated by a single discontinuity, e.g.
% % find a function $u$ holding \eqref{eq:conservedquantity} that is a solution of
% % $\eqref{eq:riemann}_1$ satisfying the initial condition $\eqref{eq:riemann}_2$:
% % \begin{equation}\label{eq:riemann}
% % 	\boxed{
% % 		\begin{aligned}
% % 			\partial_{t}u + \partial_{x}f(u) & = 0,        \\
% % 			u(x,0)                           & = u_{0}(x)=
% % 			\begin{cases}
% % 				u_{L}, \quad x<0, \\
% % 				u_{R}, \quad x>0.
% % 			\end{cases}
% % 		\end{aligned}
% % 	}
% % \end{equation}
% % % \begin{definition}
% % % 	Hyperbolic
% % % \end{definition}
% % % Let $\Omega$ be 
% % % \inputfig{floats/comparehyperbolic}{comparehyperbolic}

% % Let $f:\mathbb{R}\rightarrow\mathbb{R}$ be a $C^1$ function. We consider the \emph{Cauchy} problem
% % \begin{align}\label{eq:charac}
% % 	\frac{\partial u}{\partial t} + \frac{\partial}{\partial x}f(u) & = 0,        \qquad  x\in\mathbb{R}, t>0, \\
% % 	u(x,0)                                                          & = u_{0}(x), \qquad  x\in\mathbb{R}.
% % \end{align}
% % By setting $a(u) = f'(u)$, and letting $u$ be a classical solution of \eqref{eq:charac},
% % one obtains from \eqref{eq:charac} the \emph{non-conservative} form as follows
% % \begin{equation}
% % 	\frac{\partial u}{\partial t} + a(u)\frac{\partial u}{\partial x} = 0.
% % \end{equation}
% % The characteristic curves associated with \eqref{eq:charac} are defined as the integral curves
% % of the differential equation
% % \begin{equation}
% % 	\frac{dx}{dt} = a(u(x(t),t)).
% % \end{equation}
% % \begin{proposition}
% % 	Assume that u is a smooth solution of the Cauchy problem. The characteristic curves 
% % 	are straight lines along which u is constant.
% % \end{proposition}
% % \begin{proof}
% % 	Consider a characteristic curve passing through the point $(x_{0},0)$, i.e.
% % 	a solution of the ordinary differential system
% % 	\begin{align}
% % 		\begin{cases}\displaystyle
% % 			\frac{dx}{dt} & = a(u(x(t),t)), \\
% % 			x(0)          & = x_{0}.
% % 		\end{cases}
% % 	\end{align}
% % 	It exists at least on a small time interval $[0,t_{0})$. 
% % 	Along such a curve, u is constant since
% % 	\begin{align*}
% % 		\frac{d}{dt}u(x(t),t) 
% % 		 & = \frac{\partial u}{\partial t}(x(t),t)
% % 		+ \frac{\partial u}{\partial x}(x(t),t) \frac{dx}{dt}(t)                                    \\
% % 		 & =\left(\frac{\partial u}{\partial t} + a(u)\frac{\partial u}{\partial x}\right)(x(t),t) 
% % 		= 0.
% % 	\end{align*}
% % 	Therefore, the characteristic curves are straight lines whose constant slopes depend
% % 	on the initial data. As a result, the characteristic straight line passing through
% % 	the point $(x_{0},0)$ is defined by the equation
% % 	\begin{equation}
% % 		\boxed{
% % 			x = x_{0} + ta(u_{0}(x_{0}))
% % 		}
% % 	\end{equation}
% % \end{proof}
% % \begin{example}
% % 	Determine characteristics of the inviscid Burgers' equation with initial conditions given as
% % 	\begin{equation}
% % 		u(x,0) = u_{0}(x) =
% % 		\begin{cases}
% % 			\begin{aligned}
% % 				 & 1,   & x\leq 0,        &                        \\
% % 				 & 1-x, & 0\leq x \leq 1, & \quad x\in \mathbb{R}, \\
% % 				 & 0,   & x\geq 1.        & 
% % 			\end{aligned}
% % 		\end{cases}
% % 	\end{equation}
% % \end{example}
% % By using the method of characteristics, the solution can be solved up to the time when
% % those characteristics first intersect with each other, i.e. so-called breaking time or shock.
% % Since the $f'(u) = u$ in the inviscid Burgers' equation, it yields the characteristics 
% % passing through the point $(x_{0},t)$
% % \begin{equation}
% % 	x = x_{0} + t\,u_{0}(x_{0})
% % \end{equation}
% % which leads to
% % \begin{equation}
% % 	x(x_{0},t) = 
% % 	\begin{cases}
% % 		\begin{aligned}
% % 			 & x_{0}+t,          & x_{0}\leq 0,          \\
% % 			 & x_{0}+t(1-x_{0}), & 0\leq x_{0} \leq 1,   \\
% % 			 & x_{0},            & x_{0}\geq 1.         
% % 		\end{aligned}
% % 	\end{cases}
% % \end{equation}
% % (Sketch)
% % %------------------------------------------------------------------------------
% % % \section{General strategy for solving the first-order PDE}
% % % Supposed 
% % %------------------------------------------------------------------------------
% % % \section{Rankine-Hugoniot (RH) jump condition}

% % % Characteristic curves
% % % \begin{equation}
% % % 	\dot{x} = f'(u)\\
% % % \end{equation}
% % % \begin{align*}
% % % 	\frac{d}{dt}u(x(t),t) 
% % % 	= \frac{\partial u}{\partial t}(x(t),t)
% % % 	+ \frac{\partial u}{\partial x}(x(t),t) \frac{dx}{dt}(t)
% % % \end{align*}


% % % \inputfig{floats/multistructures}{multistructures}

\end{document}

